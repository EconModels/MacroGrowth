% This article has been prepared for publication in Energy Economics in RStudio with knitr.
% According to http://www.elsevier.com/author-schemas/the-elsarticle-latex-document-class, we should be using the
% elsarticle.cls file.
% According to http://cdn.elsevier.com/assets/pdf_file/0006/109392/journal_refstyles.pdf, we should be using
% elsarticle-template-2-harv.tex as the template for the text.
% Furthermore, we should be using model2-names.bst for the bibliographic references.
% The approach here is to load the frontmatter and backmatter from elsarticle-template-2-harv.tex
% both ahead of and behind the text for our paper.
% -- Matthew Kuperus Heun, 2013-01-18

%% This is file `elsarticle-template-2-harv.tex',
%%
%% Copyright 2009 Elsevier Ltd
%%
%% This file is part of the 'Elsarticle  Bundle'.
%% ---------------------------------------------
%%
%% It may be distributed under the conditions of the LaTeX Project Public
%% License, either version 1.2 of this license or (at your option) any
%% later version.  The latest version of this license is in
%%    http://www.latex-project.org/lppl.txt
%% and version 1.2 or later is part of all distributions of LaTeX
%% version 1999/12/01 or later.
%%
%% The list of all files belonging to the 'Elsarticle Bundle' is
%% given in the file `manifest.txt'.
%%
%% Template article for Elsevier's document class `elsarticle'
%% with harvard style bibliographic references
%%
%% $Id: elsarticle-template-2-harv.tex 155 2009-10-08 05:35:05Z rishi $
%% $URL: http://lenova.river-valley.com/svn/elsbst/trunk/elsarticle-template-2-harv.tex $
%%
\documentclass[preprint,authoryear,12pt]{elsarticle}\usepackage{graphicx, color}
%% maxwidth is the original width if it is less than linewidth
%% otherwise use linewidth (to make sure the graphics do not exceed the margin)
\makeatletter
\def\maxwidth{ %
  \ifdim\Gin@nat@width>\linewidth
    \linewidth
  \else
    \Gin@nat@width
  \fi
}
\makeatother

\IfFileExists{upquote.sty}{\usepackage{upquote}}{}
\definecolor{fgcolor}{rgb}{0.2, 0.2, 0.2}
\newcommand{\hlnumber}[1]{\textcolor[rgb]{0,0,0}{#1}}%
\newcommand{\hlfunctioncall}[1]{\textcolor[rgb]{0.501960784313725,0,0.329411764705882}{\textbf{#1}}}%
\newcommand{\hlstring}[1]{\textcolor[rgb]{0.6,0.6,1}{#1}}%
\newcommand{\hlkeyword}[1]{\textcolor[rgb]{0,0,0}{\textbf{#1}}}%
\newcommand{\hlargument}[1]{\textcolor[rgb]{0.690196078431373,0.250980392156863,0.0196078431372549}{#1}}%
\newcommand{\hlcomment}[1]{\textcolor[rgb]{0.180392156862745,0.6,0.341176470588235}{#1}}%
\newcommand{\hlroxygencomment}[1]{\textcolor[rgb]{0.43921568627451,0.47843137254902,0.701960784313725}{#1}}%
\newcommand{\hlformalargs}[1]{\textcolor[rgb]{0.690196078431373,0.250980392156863,0.0196078431372549}{#1}}%
\newcommand{\hleqformalargs}[1]{\textcolor[rgb]{0.690196078431373,0.250980392156863,0.0196078431372549}{#1}}%
\newcommand{\hlassignement}[1]{\textcolor[rgb]{0,0,0}{\textbf{#1}}}%
\newcommand{\hlpackage}[1]{\textcolor[rgb]{0.588235294117647,0.709803921568627,0.145098039215686}{#1}}%
\newcommand{\hlslot}[1]{\textit{#1}}%
\newcommand{\hlsymbol}[1]{\textcolor[rgb]{0,0,0}{#1}}%
\newcommand{\hlprompt}[1]{\textcolor[rgb]{0.2,0.2,0.2}{#1}}%

\usepackage{framed}
\makeatletter
\newenvironment{kframe}{%
 \def\at@end@of@kframe{}%
 \ifinner\ifhmode%
  \def\at@end@of@kframe{\end{minipage}}%
  \begin{minipage}{\columnwidth}%
 \fi\fi%
 \def\FrameCommand##1{\hskip\@totalleftmargin \hskip-\fboxsep
 \colorbox{shadecolor}{##1}\hskip-\fboxsep
     % There is no \\@totalrightmargin, so:
     \hskip-\linewidth \hskip-\@totalleftmargin \hskip\columnwidth}%
 \MakeFramed {\advance\hsize-\width
   \@totalleftmargin\z@ \linewidth\hsize
   \@setminipage}}%
 {\par\unskip\endMakeFramed%
 \at@end@of@kframe}
\makeatother

\definecolor{shadecolor}{rgb}{.97, .97, .97}
\definecolor{messagecolor}{rgb}{0, 0, 0}
\definecolor{warningcolor}{rgb}{1, 0, 1}
\definecolor{errorcolor}{rgb}{1, 0, 0}
\newenvironment{knitrout}{}{} % an empty environment to be redefined in TeX

\usepackage{alltt}

%% Use the option review to obtain double line spacing
%% \documentclass[authoryear,preprint,review,12pt]{elsarticle}

%% Use the options 1p,twocolumn; 3p; 3p,twocolumn; 5p; or 5p,twocolumn
%% for a journal layout:
%% \documentclass[final,authoryear,1p,times]{elsarticle}
%% \documentclass[final,authoryear,1p,times,twocolumn]{elsarticle}
%% \documentclass[final,authoryear,3p,times]{elsarticle}
%% \documentclass[final,authoryear,3p,times,twocolumn]{elsarticle}
%% \documentclass[final,authoryear,5p,times]{elsarticle}
%% \documentclass[final,authoryear,5p,times,twocolumn]{elsarticle}

%% if you use PostScript figures in your article
%% use the graphics package for simple commands
%% \usepackage{graphics}
%% or use the graphicx package for more complicated commands
%% \usepackage{graphicx}
%% or use the epsfig package if you prefer to use the old commands
%% \usepackage{epsfig}

%% The amssymb package provides various useful mathematical symbols
\usepackage{amssymb}
%% The amsthm package provides extended theorem environments
%% \usepackage{amsthm}

%% The lineno packages adds line numbers. Start line numbering with
%% \begin{linenumbers}, end it with \end{linenumbers}. Or switch it on
%% for the whole article with \linenumbers after \end{frontmatter}.
%% \usepackage{lineno}

%% natbib.sty is loaded by default. However, natbib options can be
%% provided with \biboptions{...} command. Following options are
%% valid:

%%   round  -  round parentheses are used (default)
%%   square -  square brackets are used   [option]
%%   curly  -  curly braces are used      {option}
%%   angle  -  angle brackets are used    <option>
%%   semicolon  -  multiple citations separated by semi-colon (default)
%%   colon  - same as semicolon, an earlier confusion
%%   comma  -  separated by comma
%%   authoryear - selects author-year citations (default)
%%   numbers-  selects numerical citations
%%   super  -  numerical citations as superscripts
%%   sort   -  sorts multiple citations according to order in ref. list
%%   sort&compress   -  like sort, but also compresses numerical citations
%%   compress - compresses without sorting
%%   longnamesfirst  -  makes first citation full author list
%%
%% \biboptions{longnamesfirst,comma}

% \biboptions{}

\journal{Energy Economics}

\begin{document}

\begin{frontmatter}

%% Title, authors and addresses

%% use the tnoteref command within \title for footnotes;
%% use the tnotetext command for the associated footnote;
%% use the fnref command within \author or \address for footnotes;
%% use the fntext command for the associated footnote;
%% use the corref command within \author for corresponding author footnotes;
%% use the cortext command for the associated footnote;
%% use the ead command for the email address,
%% and the form \ead[url] for the home page:
%%
%% \title{Title\tnoteref{label1}}
%% \tnotetext[label1]{}
%% \author{Name\corref{cor1}\fnref{label2}}
%% \ead{email address}
%% \ead[url]{home page}
%% \fntext[label2]{}
%% \cortext[cor1]{}
%% \address{Address\fnref{label3}}
%% \fntext[label3]{}

\title{Empirical Analysis of the Role of Energy in Economic Growth}

%% use optional labels to link authors explicitly to addresses:
%% \author[label1,label2]{<author name>}
%% \address[label1]{<address>}
%% \address[label2]{<address>}

\author[Calvin]{Caleb Reese}
\author[Calvin]{Lucas Timmer}
\author[Calvin]{Matthew Kuperus Heun\corref{cor1}}
\ead{mkh2@calvin.edu, tel: +1 (616) 526-6663, fax: +1 (616) 526-6501}

\cortext[cor1]{Corresponding author}
\address[Calvin]{Engineering Department, Calvin College, Grand Rapids, MI 49546, USA}

\begin{abstract}
%% Text of abstract
*********** Add abstract ***********
\end{abstract}

\begin{keyword}
%% keywords here, in the form: keyword \sep keyword
economic growth \sep energy \sep cobb-douglas \sep CES \sep LINEX
%% MSC codes here, in the form: \MSC code \sep code
%% or \MSC[2008] code \sep code (2000 is the default)
\end{keyword}

\end{frontmatter}

% \linenumbers
%% main text

Caleb, put your LaTeX code here.












\section{Cobb-Douglas Without Energy}







\begin{knitrout}
\definecolor{shadecolor}{rgb}{0.969, 0.969, 0.969}\color{fgcolor}\begin{kframe}
\begin{alltt}
createCDParamsGraph <- \hlfunctioncall{function}()\{
\hlcomment{  # Create the data table that we want. This table has the following columns}
\hlcomment{  # -95% CI, value, +95% CI, country abbrev, parameter (lambda, alpha, or beta)}
  dataTable <- \hlfunctioncall{do.call}(\hlstring{"rbind"}, \hlfunctioncall{lapply}(countryAbbrevs, cobbDouglasCountryRowsForParamsGraph))
  
  \hlfunctioncall{print}(dataTable)
  
  graph <- \hlfunctioncall{segplot}(value ~ country | parameter, centers=value, data=dataTable)  
  \hlfunctioncall{return}(graph)

  
    
\}

\hlfunctioncall{createCDParamsGraph}()
\end{alltt}


{\ttfamily\noindent\itshape\color{messagecolor}{Waiting for profiling to be done...}}

{\ttfamily\noindent\itshape\color{messagecolor}{Waiting for profiling to be done...}}

{\ttfamily\noindent\itshape\color{messagecolor}{Waiting for profiling to be done...}}

{\ttfamily\noindent\itshape\color{messagecolor}{Waiting for profiling to be done...}}

{\ttfamily\noindent\itshape\color{messagecolor}{Waiting for profiling to be done...}}

{\ttfamily\noindent\itshape\color{messagecolor}{Waiting for profiling to be done...}}

{\ttfamily\noindent\itshape\color{messagecolor}{Waiting for profiling to be done...}}

{\ttfamily\noindent\itshape\color{messagecolor}{Waiting for profiling to be done...}}

{\ttfamily\noindent\itshape\color{messagecolor}{Waiting for profiling to be done...}}\begin{verbatim}
      - 95% CI                value                 
 [1,] "0.00867713811541703"   "0.0101554649771947"  
 [2,] "0.213128596976903"     "0.274182451792623"   
 [3,] "0.664533917792339"     "0.725817548207377"   
 [4,] "-0.0104339533284891"   "0.0097166097229806"  
 [5,] "-0.245055231216114"    "0.444076431841734"   
 [6,] "-0.126474527059773"    "0.555923568158266"   
 [7,] "0.00214925937049534"   "0.00517407901182134" 
 [8,] "0.437044349900555"     "0.515630717914427"   
 [9,] "0.405869151146723"     "0.484369282085573"   
[10,] "-0.0405221051255718"   "0.0187921739305594"  
[11,] "0.108505956239368"     "0.712431541207713"   
[12,] "-0.319614564678439"    "0.287568458792287"   
[13,] "-0.000717427211566538" "0.000771177746585204"
[14,] "0.461441497491662"     "0.597466553596658"   
[15,] "0.264697791730982"     "0.402533446403342"   
[16,] "-0.0159263027425623"   "-0.0123103576408377" 
[17,] "0.214820434186193"     "0.448455179197524"   
[18,] "0.319930296839117"     "0.551544820802476"   
[19,] "0.0031544365568454"    "0.00385069982960034" 
[20,] "0.49113172479408"      "0.596672406132716"   
[21,] "0.297079321606455"     "0.403327593867284"   
[22,] "-0.00391419988823917"  "0.00149948729754192" 
[23,] "0.504166911618199"     "0.726578989422795"   
[24,] "0.0490171189358977"    "0.273421010577205"   
[25,] "0.0217845209774593"    "0.0249136301557912"  
[26,] "1.24947924633067"      "1.41002169368921"    
[27,] "-0.57145350717947"     "-0.410021693689214"  
      + 95% CI               country parameter
 [1,] "0.0116267632506195"   "US"    "lambda" 
 [2,] "0.335458612778305"    "US"    "alpha"  
 [3,] "0.787101178622416"    "US"    "beta"   
 [4,] "0.0302750369133842"   "UK"    "lambda" 
 [5,] "1.119149013819"       "UK"    "alpha"  
 [6,] "1.23832166337631"     "UK"    "beta"   
 [7,] "0.0081926858312227"   "JP"    "lambda" 
 [8,] "0.594141371854355"    "JP"    "alpha"  
 [9,] "0.562869413024423"    "JP"    "beta"   
[10,] "0.0779058376028465"   "CN"    "lambda" 
[11,] "1.31814921490484"     "CN"    "alpha"  
[12,] "0.894751482263012"    "CN"    "beta"   
[13,] "0.00222325751849382"  "ZA"    "lambda" 
[14,] "0.733549304597543"    "ZA"    "alpha"  
[15,] "0.540369101075701"    "ZA"    "beta"   
[16,] "-0.00873591015395777" "SA"    "lambda" 
[17,] "0.68279397709528"     "SA"    "alpha"  
[18,] "0.783159344765835"    "SA"    "beta"   
[19,] "0.00453844391610371"  "IR"    "lambda" 
[20,] "0.702639779772847"    "IR"    "alpha"  
[21,] "0.509575866128113"    "IR"    "beta"   
[22,] "0.00678367569691732"  "TZ"    "lambda" 
[23,] "0.951643458591323"    "TZ"    "alpha"  
[24,] "0.497824902218513"    "TZ"    "beta"   
[25,] "0.0280398287326144"   "ZM"    "lambda" 
[26,] "1.5728851355369"      "ZM"    "alpha"  
[27,] "-0.248589880198959"   "ZM"    "beta"   
\end{verbatim}


{\ttfamily\noindent\bfseries\color{errorcolor}{Error: invalid 'envir' argument}}\begin{alltt}

\end{alltt}
\end{kframe}
\end{knitrout}





\section{Cobb-Douglas With Energy}

We can force $\alpha$, $\beta$, and $\gamma$ to be in $[0,1]$ by a reparameterization:

\[ a \in[0,1], b \in [0,1], \alpha=\min(a,b), \beta=|b-a|, \gamma = 1-\max(a,b) \]




\subsection{Cobb-Douglas with $Q$}

\begin{knitrout}
\definecolor{shadecolor}{rgb}{0.969, 0.969, 0.969}\color{fgcolor}\begin{kframe}
\begin{alltt}
\hlcomment{# Note that the anlaysis of ZA is taking a long time here. Need to figure out why.}
CDqTables <- \hlfunctioncall{lapply}(countryAbbrevs, cobbDouglasEnergyTable, energyType=\hlstring{"Q"})
\end{alltt}
\end{kframe}
\end{knitrout}


\begin{kframe}
\begin{alltt}
\hlfunctioncall{print}(CDqTables[[\hlstring{"US"}]], caption.placement=\hlstring{"top"})
\hlfunctioncall{print}(CDqTables[[\hlstring{"ZA"}]], caption.placement=\hlstring{"top"})
\hlcomment{# According to http://cran.r-project.org/web/packages/xtable/vignettes/xtableGallery.pdf, Section 3.1, I should }
\hlcomment{# be able to use the "sanitize.text.function" parameter to allow markup in column headers. But this next}
\hlcomment{# line is not working at the present time. --MKH, 18 Jan 2012.}
\hlcomment{# print(tableCDe, sanitize.text.function = function(x)\{x\})}

\hlcomment{#print(tableAll, caption.placement="top")}
\end{alltt}
\end{kframe}


\subsection{Cobb-Douglas With $X$}

\begin{knitrout}
\definecolor{shadecolor}{rgb}{0.969, 0.969, 0.969}\color{fgcolor}\begin{kframe}
\begin{alltt}
\hlcomment{# Note that the anlaysis of ZA is taking a long time here. Need to figure out why.}
CDxTables <- \hlfunctioncall{lapply}(countryAbbrevs, cobbDouglasEnergyTable, energyType=\hlstring{"X"})
\end{alltt}
\end{kframe}
\end{knitrout}


\begin{kframe}
\begin{alltt}
\hlfunctioncall{print}(CDxTables[[\hlstring{"US"}]], caption.placement=\hlstring{"top"})
\hlfunctioncall{print}(CDxTables[[\hlstring{"ZA"}]], caption.placement=\hlstring{"top"})
\end{alltt}
\end{kframe}


\subsection{Cobb-Douglas With $U$}

\begin{knitrout}
\definecolor{shadecolor}{rgb}{0.969, 0.969, 0.969}\color{fgcolor}\begin{kframe}
\begin{alltt}
CDuTables <- \hlfunctioncall{lapply}(countryAbbrevs, cobbDouglasEnergyTable, energyType=\hlstring{"U"})
\end{alltt}
\end{kframe}
\end{knitrout}


\begin{kframe}
\begin{alltt}
\hlfunctioncall{print}(CDuTables[[\hlstring{"US"}]], caption.placement=\hlstring{"top"})
\hlfunctioncall{print}(CDuTables[[\hlstring{"ZA"}]], caption.placement=\hlstring{"top"})
\end{alltt}
\end{kframe}


\section{CES}

\begin{knitrout}
\definecolor{shadecolor}{rgb}{0.969, 0.969, 0.969}\color{fgcolor}\begin{kframe}
\begin{alltt}
cesData <- \hlfunctioncall{function}(countryName, energyType)\{
  energyColumnName <- \hlfunctioncall{paste}(\hlstring{"i"}, energyType, sep=\hlstring{""})
\hlcomment{  # Load the data that we need.}
  dataTable <- \hlfunctioncall{loadData}(countryName)
    
\hlcomment{  # Establish guess values for phi beta, zeta, lambda_L and lambda_E.}
  phiGuess <- -20
  betaGuess <- 0.5 \hlcomment{# a typical value for \hlfunctioncall{beta} (exponent on labor)}
  zetaGuess <- 0.0004 \hlcomment{# a small value}
  lambda_LGuess <- 0.007 \hlcomment{#assuming no technical progress on the labor-capital portion of the function}
  lambda_EGuess <- 0.008 \hlcomment{#assuming no technical progress on the energy portion of the function}
  
\hlcomment{  # Runs a non-linear least squares fit to the data with constraints}
  modelCES <- \hlfunctioncall{nls}(iGDP ~ ((1-zeta) * (\hlfunctioncall{exp}(lambda_L*iYear) * iCapStk^(1-beta) * iLabor^beta)^phi 
                           + zeta*(\hlfunctioncall{exp}(lambda_E*iYear) * iQ)^phi)^(1/phi), 
                   algorithm = \hlstring{"port"},
                   control = \hlfunctioncall{nls.control}(maxiter = 500, tol = 1e-06, minFactor = 1/1024, 
                                         printEval = FALSE, warnOnly = FALSE),
                   start = \hlfunctioncall{list}(phi=phiGuess, beta=betaGuess, zeta=zetaGuess, lambda_L=lambda_LGuess, 
                                lambda_E=lambda_EGuess),
                   lower = \hlfunctioncall{list}(phi=-Inf, beta=0, zeta=0, lambda_L=-Inf, lambda_E=-Inf),
                   upper = \hlfunctioncall{list}(phi=0, beta=1, zeta=1, lambda_L=Inf, lambda_E=Inf),
                   data=dataTable)

  aicCES <- \hlfunctioncall{AIC}(modelCES, k=2) \hlcomment{# Checks validity of the model. AIC stands for Akaike's Information Criterion}
  \hlfunctioncall{print}(aicCES)

\hlcomment{  # Gives the nls summary table}
  summaryCES <- \hlfunctioncall{summary}(modelCES) \hlcomment{# Gives the nls summary table}
  \hlfunctioncall{print}(summaryCES)
  
\hlcomment{  # Provides confidence intervals on phi, beta, zeta, lambda_L, and lambda_E. But, we need the CI on alpha.}
  ciCES <- \hlfunctioncall{confint}(modelCES, level = ciLevel)
  \hlfunctioncall{print}(ciCES)
  
\hlcomment{  # Get the estimate for alpha}
  beta <- \hlfunctioncall{as.numeric}(\hlfunctioncall{coef}(modelCES)[\hlstring{"beta"}])
  alpha <- 1.0 - beta
  alpha.est <- \hlfunctioncall{deltaMethod}(modelCES, \hlstring{"1 - beta"}) # Estimates alpha and its standard \hlfunctioncall{error} (SE).
  \hlfunctioncall{print}(alpha.est) 
  
\hlcomment{  # Now calculate a confidence interval on alpha}
  dofCES <- summaryCES$df[2]
  \hlfunctioncall{print}(dofCES) \hlcomment{# Gives the degrees of freedom for the model.}
  tvalCES <- \hlfunctioncall{qt}(ciHalfLevel, df = dofCES); tvalCES
\hlcomment{  # Get confidence intervals for each parameter in the model}
  alphaCICES <- \hlfunctioncall{with}(alpha.est, Estimate + \hlfunctioncall{c}(-1.0, 1.0) * tvalCES * SE) \hlcomment{# CI on alpha.}
  \hlfunctioncall{print}(alphaCICES) 

\hlcomment{  # Assemble the data into data frames for the table.}
  estCES <- \hlfunctioncall{data.frame}(phi = \hlfunctioncall{coef}(modelCES)[\hlstring{"phi"}], alpha = alpha, 
                       beta = \hlfunctioncall{coef}(modelCES)[\hlstring{"beta"}], zeta = \hlfunctioncall{coef}(modelCES)[\hlstring{"zeta"}], 
                       lambda_L = \hlfunctioncall{coef}(modelCES)[\hlstring{"lambda_L"}], lambda_E = \hlfunctioncall{coef}(modelCES)[\hlstring{"lambda_E"}])
  \hlfunctioncall{row.names}(estCES) <- \hlfunctioncall{paste}(\hlstring{"CES with "}, energyType, sep=\hlstring{""})
\hlcomment{  #print(estCES)  }
\hlcomment{  # The [1] subscripts pick off the lower confidence interval}
  lowerCES <- \hlfunctioncall{data.frame}(phi = ciCES[\hlstring{"phi"},\hlstring{"2.5%"}], alpha = alphaCICES[1], 
                         beta = ciCES[\hlstring{"beta"}, \hlstring{"2.5%"}], zeta = ciCES[\hlstring{"zeta"}, \hlstring{"2.5%"}],
                         lambda_L = ciCES[\hlstring{"lambda_L"}, \hlstring{"2.5%"}], lambda_E = ciCES[\hlstring{"lambda_E"}, \hlstring{"2.5%"}])
  \hlfunctioncall{row.names}(lowerCES) <- \hlstring{"- 95% CI"}
\hlcomment{  # The [2] subscripts pick off the lower confidence interval}
  upperCES <- \hlfunctioncall{data.frame}(phi = ciCES[\hlstring{"phi"},\hlstring{"97.5%"}], alpha = alphaCICES[2], 
                         beta = ciCES[\hlstring{"beta"}, \hlstring{"97.5%"}], zeta = ciCES[\hlstring{"zeta"}, \hlstring{"97.5%"}],
                         lambda_L = ciCES[\hlstring{"lambda_L"}, \hlstring{"97.5%"}], lambda_E = ciCES[\hlstring{"lambda_E"}, \hlstring{"97.5%"}])
  \hlfunctioncall{row.names}(upperCES) <- \hlstring{"+ 95% CI"}
  
\hlcomment{  # Now create the data for a table.}
  dataCES <- \hlfunctioncall{rbind}(upperCES, estCES, lowerCES)
  \hlfunctioncall{print}(dataCES)
  \hlfunctioncall{return}(dataCES)

\hlcomment{  #xyplot( resid(modelCESQ) ~ fitted(modelCESQ) )}
\hlcomment{  #histogram( ~resid(modelCESQ) )}
\hlcomment{  #qqmath( ~resid(modelCESQ) )}
\}

\hlcomment{####################################}
\hlcomment{# Creates a LaTeX printable table from the CES data. This function first calls cesData.}
\hlcomment{#}
\hlcomment{# countryName is a string containint the 2-letter abbreviation for the country, e.g. "US" or "CN"}
\hlcomment{# energyType is a string to be used in table captions reprsenting the type of energy. Typically, "Q", "X", or "U"}
\hlcomment{#}
\hlcomment{# returns a printable LaTeX table from xtable.}
\hlcomment{##}
cesTable <- \hlfunctioncall{function}(countryName, energyType)\{
  dataCESe <- \hlfunctioncall{cesData}(countryName, energyType)
  tableCESq <- \hlfunctioncall{xtable}(dataCESe, caption=\hlfunctioncall{paste}(countryName, \hlstring{", 1980-2011."}, sep=\hlstring{""}), digit = \hlfunctioncall{c}(4, 1, 2, 2, 6, 5, 5))
\}
\end{alltt}
\end{kframe}
\end{knitrout}


\subsection{CES with $Q$}
\begin{knitrout}
\definecolor{shadecolor}{rgb}{0.969, 0.969, 0.969}\color{fgcolor}\begin{kframe}
\begin{alltt}
countryName <- \hlstring{"US"}
energyType <- \hlstring{"Q"}
tableCESq <- \hlfunctioncall{cesTable}(countryName, energyType)

\hlcomment{#CESqTables <- lapply(countryAbbrevs, cesTable, energyType="Q")}
\end{alltt}
\end{kframe}
\end{knitrout}


\begin{kframe}
\begin{alltt}
\hlfunctioncall{print}(tableCESq, caption.placement=\hlstring{"top"})

\hlcomment{#print(CESqTables[["US"]], caption.placement="top")}
\hlcomment{#print(CESqTables[["ZA"]], caption.placement="top")}
\end{alltt}
\end{kframe}


%% The Appendices part is started with the command \appendix;
%% appendix sections are then done as normal sections
%% \appendix

%% \section{}
%% \label{}

%% References
%%
%% Following citation commands can be used in the body text:
%%
%%  \citet{key}  ==>>  Jones et al. (1990)
%%  \citep{key}  ==>>  (Jones et al., 1990)
%%
%% Multiple citations as normal:
%% \citep{key1,key2}         ==>> (Jones et al., 1990; Smith, 1989)
%%                            or  (Jones et al., 1990, 1991)
%%                            or  (Jones et al., 1990a,b)
%% \cite{key} is the equivalent of \citet{key} in author-year mode
%%
%% Full author lists may be forced with \citet* or \citep*, e.g.
%%   \citep*{key}            ==>> (Jones, Baker, and Williams, 1990)
%%
%% Optional notes as:
%%   \citep[chap. 2]{key}    ==>> (Jones et al., 1990, chap. 2)
%%   \citep[e.g.,][]{key}    ==>> (e.g., Jones et al., 1990)
%%   \citep[see][pg. 34]{key}==>> (see Jones et al., 1990, pg. 34)
%%  (Note: in standard LaTeX, only one note is allowed, after the ref.
%%   Here, one note is like the standard, two make pre- and post-notes.)
%%
%%   \citealt{key}          ==>> Jones et al. 1990
%%   \citealt*{key}         ==>> Jones, Baker, and Williams 1990
%%   \citealp{key}          ==>> Jones et al., 1990
%%   \citealp*{key}         ==>> Jones, Baker, and Williams, 1990
%%
%% Additional citation possibilities
%%   \citeauthor{key}       ==>> Jones et al.
%%   \citeauthor*{key}      ==>> Jones, Baker, and Williams
%%   \citeyear{key}         ==>> 1990
%%   \citeyearpar{key}      ==>> (1990)
%%   \citetext{priv. comm.} ==>> (priv. comm.)
%%   \citenum{key}          ==>> 11 [non-superscripted]
%% Note: full author lists depends on whether the bib style supports them;
%%       if not, the abbreviated list is printed even when full requested.
%%
%% For names like della Robbia at the start of a sentence, use
%%   \Citet{dRob98}         ==>> Della Robbia (1998)
%%   \Citep{dRob98}         ==>> (Della Robbia, 1998)
%%   \Citeauthor{dRob98}    ==>> Della Robbia


%% References with bibTeX database:

\bibliographystyle{model2-names}
\bibliography{<your-bib-database>}

%% Authors are advised to submit their bibtex database files. They are
%% requested to list a bibtex style file in the manuscript if they do
%% not want to use model2-names.bst.

%% References without bibTeX database:

% \begin{thebibliography}{00}

%% \bibitem must have one of the following forms:
%%   \bibitem[Jones et al.(1990)]{key}...
%%   \bibitem[Jones et al.(1990)Jones, Baker, and Williams]{key}...
%%   \bibitem[Jones et al., 1990]{key}...
%%   \bibitem[\protect\citeauthoryear{Jones, Baker, and Williams}{Jones
%%       et al.}{1990}]{key}...
%%   \bibitem[\protect\citeauthoryear{Jones et al.}{1990}]{key}...
%%   \bibitem[\protect\astroncite{Jones et al.}{1990}]{key}...
%%   \bibitem[\protect\citename{Jones et al., }1990]{key}...
%%   \harvarditem[Jones et al.]{Jones, Baker, and Williams}{1990}{key}...
%%

% \bibitem[ ()]{}

% \end{thebibliography}

\end{document}

%%
%% End of file `elsarticle-template-2-harv.tex'.
