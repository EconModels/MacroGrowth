%% This is file `elsarticle-template-2-harv.tex',
%%
%% Copyright 2009 Elsevier Ltd
%%
%% This file is part of the 'Elsarticle  Bundle'.
%% ---------------------------------------------
%%
%% It may be distributed under the conditions of the LaTeX Project Public
%% License, either version 1.2 of this license or (at your option) any
%% later version.  The latest version of this license is in
%%    http://www.latex-project.org/lppl.txt
%% and version 1.2 or later is part of all distributions of LaTeX
%% version 1999/12/01 or later.
%%
%% The list of all files belonging to the 'Elsarticle Bundle' is
%% given in the file `manifest.txt'.
%%
%% Template article for Elsevier's document class `elsarticle'
%% with harvard style bibliographic references
%%
%% $Id: elsarticle-template-2-harv.tex 155 2009-10-08 05:35:05Z rishi $
%% $URL: http://lenova.river-valley.com/svn/elsbst/trunk/elsarticle-template-2-harv.tex $
%%
\documentclass[preprint,authoryear,12pt]{elsarticle}

%% Use the option review to obtain double line spacing
%% \documentclass[authoryear,preprint,review,12pt]{elsarticle}

%% Use the options 1p,twocolumn; 3p; 3p,twocolumn; 5p; or 5p,twocolumn
%% for a journal layout:
%% \documentclass[final,authoryear,1p,times]{elsarticle}
%% \documentclass[final,authoryear,1p,times,twocolumn]{elsarticle}
%% \documentclass[final,authoryear,3p,times]{elsarticle}
%% \documentclass[final,authoryear,3p,times,twocolumn]{elsarticle}
%% \documentclass[final,authoryear,5p,times]{elsarticle}
%% \documentclass[final,authoryear,5p,times,twocolumn]{elsarticle}

%% if you use PostScript figures in your article
%% use the graphics package for simple commands
%% \usepackage{graphics}
%% or use the graphicx package for more complicated commands
%% \usepackage{graphicx}
%% or use the epsfig package if you prefer to use the old commands
%% \usepackage{epsfig}

%% The amssymb package provides various useful mathematical symbols
\usepackage{amssymb}
%% The amsthm package provides extended theorem environments
%% \usepackage{amsthm}

\usepackage{wrapfig}    % Allows wrapping of text around figures
\usepackage{float}      % Allows precise positioning of tables and figures within the text
\usepackage{amsmath}    % Allows \begin{equation*} \end{equation*} for unnumbered equations
\usepackage{multirow}   % To create multirow tables
\usepackage{hyperref}   % To create hyperlinks in the paper
\usepackage{microtype}  % Produces hanging punctuation and beautiful type
\usepackage{booktabs}   % For nice tables
\usepackage{nicefrac}   % For beautiful fractions


% From http://economics.utoronto.ca/osborne/latex/BIBTEX.HTM
\newcommand{\citeapos}[1]{\citeauthor{#1}'s (\citeyear{#1})} % Posessive citations. 

%% The lineno packages adds line numbers. Start line numbering with
%% \begin{linenumbers}, end it with \end{linenumbers}. Or switch it on
%% for the whole article with \linenumbers after \end{frontmatter}.
%% \usepackage{lineno}

%% natbib.sty is loaded by default. However, natbib options can be
%% provided with \biboptions{...} command. Following options are
%% valid:

%%   round  -  round parentheses are used (default)
%%   square -  square brackets are used   [option]
%%   curly  -  curly braces are used      {option}
%%   angle  -  angle brackets are used    <option>
%%   semicolon  -  multiple citations separated by semi-colon (default)
%%   colon  - same as semicolon, an earlier confusion
%%   comma  -  separated by comma
%%   authoryear - selects author-year citations (default)
%%   numbers-  selects numerical citations
%%   super  -  numerical citations as superscripts
%%   sort   -  sorts multiple citations according to order in ref. list
%%   sort&compress   -  like sort, but also compresses numerical citations
%%   compress - compresses without sorting
%%   longnamesfirst  -  makes first citation full author list
%%
%% \biboptions{longnamesfirst,comma}

% \biboptions{}

\journal{}

\begin{document}

\begin{frontmatter}

%% Title, authors and addresses

%% use the tnoteref command within \title for footnotes;
%% use the tnotetext command for the associated footnote;
%% use the fnref command within \author or \address for footnotes;
%% use the fntext command for the associated footnote;
%% use the corref command within \author for corresponding author footnotes;
%% use the cortext command for the associated footnote;
%% use the ead command for the email address,
%% and the form \ead[url] for the home page:
%%
%% \title{Title\tnoteref{label1}}
%% \tnotetext[label1]{}
%% \author{Name\corref{cor1}\fnref{label2}}
%% \ead{email address}
%% \ead[url]{home page}
%% \fntext[label2]{}
%% \cortext[cor1]{}
%% \address{Address\fnref{label3}}
%% \fntext[label3]{}

\title{Primer on Production Functions and Economic Growth}

%% use optional labels to link authors explicitly to addresses:
%% \author[label1,label2]{<author name>}
%% \address[label1]{<address>}
%% \address[label2]{<address>}

% \author[BenWarr]{Benjamin Warr}
% \author[IST]{Tiago Domingos}
% \author[Leeds]{Paul Brockway}
% \author[Leeds]{Julia Steinberger}
\author[CalvinEngr]{Matthew Kuperus Heun\corref{cor1}}
\ead{mkh2@calvin.edu}

% \address[BenWarr]{Ben Warr's address, Zambia}
% \address[IST]{IST, Lisbon, Portugal}
% \address[Leeds]{Leeds University, Leeds, UK}
\cortext[cor1]{Corresponding author}
\address[CalvinEngr]{Engineering Department, Calvin College, Grand Rapids, MI 49546, USA}

\begin{abstract}
%% Text of abstract
**** Add abstract ****
\end{abstract}

\begin{keyword}
%% keywords here, in the form: keyword \sep keyword
production function \sep economic growth \sep energy \sep Cobb-Douglas \sep CES \sep LinEx
%% MSC codes here, in the form: \MSC code \sep code
%% or \MSC[2008] code \sep code (2000 is the default)
\end{keyword}

\end{frontmatter}

% \linenumbers
%% main text


%%%%%%%%%%%%%%%%%%%%%%%%%%%%%%%
\section{Introduction} 
\label{sec:Introduction}
%%%%%%%%%%%%%%%%%%%%%%%%%%%%%%%

This primer on production functions 
provides background on several production functions
for the purpose of accounting for the role of energy
in economic growth.


%%%%%%%%%%%%%%%%%%%%%%%%%%%%%%%
\section{Differential Economic Growth} 
\label{sec:diff_growth}
%%%%%%%%%%%%%%%%%%%%%%%%%%%%%%%

The differential equation for economic growth~($y$) with 
capital~($k$), labor~($l$), and energy~($e$) factors of production, 
i.e., $y = y(k, l, e; t)$ is

\begin{equation} \label{eq:diff_growth}
  \frac{1}{y}\frac{\mathrm{d}y}{\mathrm{d}t} 
      = \lambda \frac{1}{t - t_0}
      + \alpha \frac{1}{k}\frac{\mathrm{d}k}{\mathrm{d}t} 
      + \beta \frac{1}{l}\frac{\mathrm{d}l}{\mathrm{d}t} 
      + \gamma \frac{1}{e}\frac{\mathrm{d}e}{\mathrm{d}t} \; ,
\end{equation}
%
where 
$y \equiv Y/Y_{0}$,
$t$ (time) is often measured in years,
$\lambda$ represents the time dependence of the change of $y$,
$k \equiv K/K_0$, 
$l \equiv L/L_0$, 
$e \equiv E/E_0$.
$Y$ (economic output) is measured by GDP in currency units, 
$K$ (capital) is expressed in currency units, 
$L$ (labor) is expressed in workers or work-hours/year, and
the subscript 0 indicates values at an initial year.
Constant returns to scale are represented by the constraint
$\alpha + \beta + \gamma = 1$. 

The output elasticities $\alpha$, $\beta$, and $\gamma$ are defined as

\begin{equation} \label{eq:alaph_def}
  \alpha \equiv \frac{k}{y} \frac{\partial k / \partial t}{\partial y / \partial t} \; , \;
  \beta  \equiv \frac{l}{y} \frac{\partial k / \partial t}{\partial y / \partial t} \; , \;
  \gamma \equiv \frac{e}{y} \frac{\partial e / \partial t}{\partial y / \partial t} \; .
\end{equation}


%%%%%%%%%%%%%%%%%%%%%%%%%%%%%%%
\section{Leontief Production Function} 
\label{sec:Leontief}
%%%%%%%%%%%%%%%%%%%%%%%%%%%%%%%

The Leontief production function is given by 
%
\begin{equation} \label{eq:Leontief}
  y = \min \left[ a k, b l, c e\right]
\end{equation}
%
where $a$, $b$, and $c$ are technologically-determined constants. 
The Leontief function is a recipe for production.

An example of a Leontief system is the production of automobiles.
If $Y$ is the production rate of cars (cars/year),
$W$ is wheels (wheels/year), 
$S$ is steering wheels (steering wheels/year), and
$R$ is rear-view mirrors (rear-view mirrors/year), 
a Leontief production function is
%
\begin{equation} \label{eq:cars}
  Y = \min \left[ a W, \; b S, \; c R\right] \; ,
\end{equation}
%
where $a$ = 1 car/4 wheels, $b$ = 1 car/steering wheel, and $c$ = 1 car/rear-view mirror.

The Leontief production function indicates that 
oversupplying a factor of production~($W$, $S$, or $R$) will not increase output~($Y$).
Put another way, wheels~($W$), steering wheels~($S$), and rear-view mirrors~($R$)
are not substitutable. 

In the Leontief world, there is always at least one limiting factor of production.
With 8 wheels, 10 steering wheels, and 10 rear-view mirrors,
an automaker can produce only 2 cars.
Under these conditions, wheels are the limiting factor of production, 
and there will be 8 unused steering wheels and 8 unused rear-view mirrors, 
a non-optimal production system indeed.
The optimal (profit-maximizing) production strategy to make 2 cars would be to supply
8 wheels, 2 steering wheels, and 2 rear-view mirrors.

In modern economies, 
constraints exist in terms of capital~($k$), labor~($l$), and energy~($e$).~\citep{Kummel:2010vz}
%
\begin{itemize}

  \item Given a fully-utilized machine, throwing more energy at it will \emph{not} increase output~($y$).%
  \footnote{
  In fact, adding additional energy to the machine may cause damage.
  }
  Capital~($k$) constrains energy~($e$).
  
  \item Given a state of automation of the economy at an instant in time, 
  and assuming full utilization of the capital~($k$)
  in terms of both labor~($l$) and energy~($e$) for operation,
  adding labor to the economy doesn't increase output~($y$). 
  Capital~($k$) and energy~($e$) constrain labor~($l$).

\end{itemize}


%%%%%%%%%%%%%%%%%%%%%%%%%%%%%%%
\section{Linear Production Function} 
\label{sec:linear}
%%%%%%%%%%%%%%%%%%%%%%%%%%%%%%%

The linear production function is given by
%
\begin{equation} \label{eq:linear}
  y = a k + b l + c e \; .
\end{equation}
%
In Equation~\ref{eq:linear}, $a$, $b$, and $c$ are positive constants.
In the linear world, factors of production are perfect substitutes.

Continuing the cooking analogy, consider producing toast ($Y$) 
with a fatty substance~(butter, $B$, or margarine, $M$).
%
\begin{equation} \label{eq:toast}
  Y = a B + b M
\end{equation}
%
The linear model (Equation~\ref{eq:toast}) indicates that butter~($B$) and margarine~($M$) 
can be used interchangably as the fatty substance.
Butter~($B$) and margarine~($M$) are perfect substitutes.%
  \footnote{
  Butter and margarine are perfect substitutes for all but the most discerning toast consumers.
  }%



%%%%%%%%%%%%%%%%%%%%%%%%%%%%%%%
\section{Cobb-Douglas Production Function} 
\label{sec:CD}
%%%%%%%%%%%%%%%%%%%%%%%%%%%%%%%

The Cobb-Douglas production function is given by
%
\begin{equation} \label{eq:CD}
  y = A k^\alpha l^\beta e^\gamma \; ; \; A \equiv \mathrm{e}^{\lambda(t-t_0)} \; .
\end{equation}
%
$A$ is known as total factor productivity,
and $\lambda$ is the Solow residual.

In the Cobb-Douglas world, factors of production are substitutable, 
eg., a reduction of labor~($l$) can be compensated by an increase of capital~($k$)
without affecting output~($y$).%
  \footnote{
  In fact, the capital-for-labor substitution has a name: automation.
  }
Furthermore, an increase in one factor of production (say, $k$)
without a concomitant increase in the others ($l$ or $e$) 
will result in an increase of output ($y$), provided that the output elasticity 
($\alpha$) is non-zero.


%%%%%%%%%%%%%%%%%%%%%%%%%%%%%%%
\section{Constant Elasticity of Substitution Production Function (CES)} 
\label{sec:CES}
%%%%%%%%%%%%%%%%%%%%%%%%%%%%%%%

The Constant Elasticity of Substitution production function
provides smooth transition between perfectly complementary (Leontief) 
and perfectly substitutable (linear) factors of production.


%++++++++++++++++++++++++++++++
\subsection{CES with Two Factors of Production} 
\label{sec:CES-2}
%++++++++++++++++++++++++++++++

With two factors of production (capital, $k$ and labor, $l$), 
the CES production function is given by
%
\begin{equation} \label{eq:CES}
  y = A \: \left[\delta_1 k^{-\rho_1} 
      + (1-\delta_1)l^{-\rho_1} \right]^{-1/\rho_1} \; ; \;
      A = \mathrm{e}^{\lambda (t-t_0)}   \; .
\end{equation}
%
In Equation~\ref{eq:CES}, capital~($k$) and labor~($l$) are said to be ``nested''
together in the production function.

$\delta_1$ describes the importance of capital ($k$)
relative to labor ($l$).
As $\delta_1 \rightarrow 1$, only capital~($k$) is necessary for production.
As $\delta_1 \rightarrow 0$, only labor~($l$) is necessary for production.

$\rho_1$ indicates the elasticity 
of substitution~($\sigma_1$) between capital~($k$) and labor~($l$).
The elasticity of substitution~($\sigma_1$)
between capital~($k$) and labor~($l$) is related to $\rho_1$ and given by 
%
\begin{equation} \label{eq:sigma_1}
 \sigma_1 = \frac{1}{1+\rho_1} \; .
\end{equation}
%
Solving for $\rho_1$ gives
%
\begin{equation} \label{eq:rho_1}
  \rho_1 = \frac{1 - \sigma_1}{\sigma_1} \; .
\end{equation}

Constraints on the parameters include 
$\delta_1 \in [0,1]$ and
$\rho_1 \in (-1,0) \cup (0,\infty)$.



%++++++++++++++++++++++++++++++
\subsection{CES with Three Factors of Production} 
\label{sec:CES-3}
%++++++++++++++++++++++++++++++

A doubly-nested form of the CES production function is required
if a third factor of production~($e$) is to be included.
%
\begin{equation} \label{eq:CES-kle}
  y = A \: \left\{\delta \left[\delta_1 k^{-\rho_1} 
      + (1-\delta_1)l^{-\rho_1} \right]^{\rho/\rho_1} 
      + (1-\delta) e^{-\rho} \right\}^{-1/\rho} \; ; \; 
      A = \mathrm{e}^{\lambda (t-t_0)}
\end{equation}
%
In Equation~\ref{eq:CES-kle}, $\delta$ 
describes the importance of the capital-labor ($kl$) nesting 
relative to energy~($e$).

Important questions arise: why nest capital~($k$) and labor~($l$) together 
and separate energy~($e$)
in the~($kl$)($e$) nest structure?
Why not nest, e.g., labor~($l$) and energy~($e$) in a~($le$)($k$) structure?%
  \footnote{
  Note that the $\rho$ (and, therefore, $\sigma$) and $\delta$ parameters have different
  meanings depending upon how the factors of production are nested.
  }%
%
\begin{equation} \label{eq:CES-lek}
  y = A \: \left\{\delta \left[\delta_1 l^{-\rho_1} 
      + (1-\delta_1) e^{-\rho_1} \right]^{\rho/\rho_1} 
      + (1-\delta) k^{-\rho} \right\}^{-1/\rho} \; ; \; 
      A = \mathrm{e}^{\lambda (t-t_0)}
\end{equation}
%
Or nest energy and capital in a~($ek$)($l$) structure?
%
\begin{equation} \label{eq:CES-ekl}
  y = A \: \left\{\delta \left[\delta_1 e^{-\rho_1} 
      + (1-\delta_1) k^{-\rho_1} \right]^{\rho/\rho_1} 
      + (1-\delta) l^{-\rho} \right\}^{-1/\rho} \; ; \; 
      A = \mathrm{e}^{\lambda (t-t_0)}
\end{equation}


%%%%%%%%%%%%%%%%%%%%%%%%%%%%%%%
\section{Relating CES to Other Production Functions} 
\label{sec:CES-others}
%%%%%%%%%%%%%%%%%%%%%%%%%%%%%%%

Equation~\ref{eq:CES} simplifies to the other production functions 
with different values of $\sigma$.
For the two-factor CES production function, 
Table~\ref{tab:ces_simplifications} describes the options.

\begin{table}
\caption[CES Simplifications]{CES Simplifications.}
\begin{center}
 \begin{tabular}{ c c c c }
\toprule 
$\sigma$ & $\rho$ & CES simplifies to \dots & Factors of production are \dots \\
\midrule
0        & $\infty$ & Leontief              & perfect complements             \\
% \midrule
1        & 0        & Cobb-Douglas          & substitutable                   \\
% \midrule
$\infty$ & $-1$     & linear                & perfect substitutes             \\
\bottomrule
\end{tabular}
\end{center}
\label{tab:ces_simplifications}
\end{table}

Figure~\ref{fig:CES_isoquants} shows the simplifications from Table~\ref{tab:ces_simplifications}
graphically.
%
\begin{figure}[!ht]
\centering\
\includegraphics[width=0.8\linewidth]{figure/CESIsoquants.png}
\caption[CES isoquants]{CES isoquants (lines of constant output, $Y$). 
$K$ is capital and $N$ is labor. From \citet{Klump:2011aa}.}
\label{fig:CES_isoquants}
\end{figure}


%%%%%%%%%%%%%%%%%%%%%%%%%%%%%%%
\section{LinEx Production Function} 
\label{sec:linex}
%%%%%%%%%%%%%%%%%%%%%%%%%%%%%%%

An additional production function, the Linear Exponential (LinEx%
    \footnote{
    The name ``LinEx'' comes from the form of the equation:
    production is \emph{linear} in energy~($e$) and \emph{exponential}
    in ratios of the factors of production~($k$, $l$, and $e$).
    }%
) function is given by K\"{u}mmel in several places~\citep{Kummel:1980wx,Kummel:1982vy,
Kummel:1985vz,Kummel:1989tf,Kummel:2002tx,Kummel:2010vz} as
%
\begin{equation} \label{eq:LinEx-kummel}
  y = e \exp{\left[ a \left( 2 - \frac{l + e}{k} \right)  + a c \left( \frac{l}{e} - 1 \right)\right]} \; .
\end{equation}
%
An alternative presentation that affords easier interpretation is
%
\begin{equation} \label{eq:LinEx}
  y = A e \; ; \; A \equiv \mathrm{e}^{a\left[2 \left(1 - \frac{1}{\rho_k} \right) 
    + c \left(\vphantom{\frac{1}{\rho_k}}\rho_l - 1 \right)\right]} \; .
\end{equation}
%
The LinEx function was derived from thermodynamic considerations. 
A key feature of the derivation involves
constraints among capital~($k$), labor~($l$), and energy~($e$).
These constraints were discussed in Section~\ref{sec:Leontief} above.%
  \footnote{
  As an aside, I question whether the constraints 
  described in Section~\ref{sec:Leontief} are the right ones.
  }%

In contrast to both the Cobb-Douglas model (Equation~\ref{eq:CD}) 
and the CES model~(Section~\ref{sec:CES}), 
the LiNex model (Equation~\ref{eq:LinEx}) does not include a generic, 
time-dependent augmentation term~($A$).
The LinEx model assumes energy~($e$) is the only factor of production.

The LinEx model does, however, include terms that represent
efficiencies among capital, labor, and energy, in the form of 
the ratios $\rho_k$ and $\rho_l$ which are defined by
%
\begin{equation} \label{eq:rho_k}
  \rho_k \equiv \frac{k}{(\nicefrac{1}{2})(l+e)}
\end{equation}
%
and
%
\begin{equation} \label{eq:rho_l}
  \rho_l \equiv \frac{l}{e}
\end{equation}
%
and represent 
%
(\ref{eq:rho_k}) capital stock increase relative to the average of labor and energy (capital deepening) and 
(\ref{eq:rho_l}) labor increase relative to energy.
The effect of energy ($e$) on output ($y$) is enhanced when $A > 1$,
which occurs if either
%
\begin{itemize}
\item{capital stock ($k$) has increased more than the average 
      of labor ($l$) and energy ($e$), i.e. if $\rho_k >$ 1 or}
\item{labor ($l$) has increased more than energy ($e$), i.e. if $\rho_l >$ 1.}
\end{itemize}
%
An economy that increases 
capital stock ($k$) and labor ($l$) at the same rate as energy ($e$) will have
capital and labor efficiency ratios ($\rho_k$ and $\rho_l$, respectively) of unity.
In that case, economic output ($y$) increases at the same rate as energy consumption ($e$).
According to the LiNex model, an economy that increases 
capital stock ($k$) without a commensurate increase in the average of labor
and energy [$(\nicefrac{1}{2})(l+e)$] will experience an increase in output ($y$) in excess 
of its increase of energy consumption ($e$), because $\rho_k > 1$ and
%
\begin{equation}
    \mathrm{e}^{a \left[2 \left(1 - \frac{1}{\rho_k} \right) 
    + c \left(\vphantom{\frac{1}{\rho_k}}\rho_l - 1 \right)\right]}
	> 1 \; .
\end{equation}
%
Similarly, an economy benefits by increasing labor ($l$) 
without a commensurate increase in energy ($e$), because $\rho_l > 1$. 
Thus, $\rho_k$ is an 
efficiency of using additional labor~($l$) and energy~($e$) to make 
additional capital stock~($k$) available to the economy, 
and $\rho_l$ is an efficiency of using additional energy~($e$) to make 
additional labor~($l$) available to the economy. 
Increases in $\rho_k$ and $\rho_l$ provide upward pressure on economic output~($y$),
as the only factor of production~($e$) is used more efficiently.

The parameter $a$ represents the overall importance of efficiencies $\rho_k$ and $\rho_l$, 
and the parameter $c$ represents the relative importance 
of labor efficiency~($\rho_l$) compared to capital stock efficiency~($\rho_k$). 

\citet{Kummel:2010vz} use primary exergy for $e$ and find that 
both $a$ and $c$ vary with time.
\citet{Warr:2012cg} use useful work for $e$ and find that 
constant values of $a$ and $c$ are sufficient to describe economic growth 
over the long run. 

For the Leontief, linear, Cobb-Douglas, and CES production functions, 
output elasticities are constant with respect to time.
However, for the LinEx function, 
the values of coefficients $a$ and $c$ imply time-dependent
values of $\alpha$, $\beta$, and $\gamma$ as given by the following equations~\citep{Warr:2012cg}:
%
\begin{equation} \label{eq:LINEX_alpha}
  \alpha \equiv \frac{k}{y}\frac{\partial y}{\partial k} = a \left( \frac{l + e}{k} \right) \;,
\end{equation}

\begin{equation} \label{eq:LINEX_beta}
  \beta \equiv \frac{l}{y} \frac{\partial y}{\partial l} = a \left[ c \left( \frac{l}{e}\right) - \frac{l}{k} \right] \;,
\end{equation}
%
\noindent{}and
%
\begin{equation} \label{eq:LINEX_beta_2}
  \gamma \equiv \frac{e}{y} \frac{\partial y}{\partial e} 
         = 1 - a \left[ \frac{e}{k} + c \frac{l}{e} \right] 
         = 1 - \alpha - \beta \;.
\end{equation}
%
The LinEx model assumes constant returns to scale.


%%%%%%%%%%%%%%%%%%%%%%%%%%%%%%%
\section{Conclusion}
\label{sec:Conclusion}
%%%%%%%%%%%%%%%%%%%%%%%%%%%%%%%


%%%%%%%%%%%%%%%%%%%%%%%%%%%%%%%
\section{Future Work}
\label{sec:FutureWork}
%%%%%%%%%%%%%%%%%%%%%%%%%%%%%%%




%%%%%%%%%%%%%%%%%%%%%%%%%%%%%%%
\section*{Acknowledgements}
\label{sec:Acknowledgements}
%%%%%%%%%%%%%%%%%%%%%%%%%%%%%%%






%% References
%%
%% Following citation commands can be used in the body text:
%%
%%  \citet{key}  ==>>  Jones et al. (1990)
%%  \citep{key}  ==>>  (Jones et al., 1990)
%%
%% Multiple citations as normal:
%% \citep{key1,key2}         ==>> (Jones et al., 1990; Smith, 1989)
%%                            or  (Jones et al., 1990, 1991)
%%                            or  (Jones et al., 1990a,b)
%% \cite{key} is the equivalent of \citet{key} in author-year mode
%%
%% Full author lists may be forced with \citet* or \citep*, e.g.
%%   \citep*{key}            ==>> (Jones, Baker, and Williams, 1990)
%%
%% Optional notes as:
%%   \citep[chap. 2]{key}    ==>> (Jones et al., 1990, chap. 2)
%%   \citep[e.g.,][]{key}    ==>> (e.g., Jones et al., 1990)
%%   \citep[see][pg. 34]{key}==>> (see Jones et al., 1990, pg. 34)
%%  (Note: in standard LaTeX, only one note is allowed, after the ref.
%%   Here, one note is like the standard, two make pre- and post-notes.)
%%
%%   \citealt{key}          ==>> Jones et al. 1990
%%   \citealt*{key}         ==>> Jones, Baker, and Williams 1990
%%   \citealp{key}          ==>> Jones et al., 1990
%%   \citealp*{key}         ==>> Jones, Baker, and Williams, 1990
%%
%% Additional citation possibilities
%%   \citeauthor{key}       ==>> Jones et al.
%%   \citeauthor*{key}      ==>> Jones, Baker, and Williams
%%   \citeyear{key}         ==>> 1990
%%   \citeyearpar{key}      ==>> (1990)
%%   \citetext{priv. comm.} ==>> (priv. comm.)
%%   \citenum{key}          ==>> 11 [non-superscripted]
%% Note: full author lists depends on whether the bib style supports them;
%%       if not, the abbreviated list is printed even when full requested.
%%
%% For names like della Robbia at the start of a sentence, use
%%   \Citet{dRob98}         ==>> Della Robbia (1998)
%%   \Citep{dRob98}         ==>> (Della Robbia, 1998)
%%   \Citeauthor{dRob98}    ==>> Della Robbia


%% References with bibTeX database:

%%%%%%%%%%%%%%%%%%%%%%%%%%%%%%%
\section*{References}
%%%%%%%%%%%%%%%%%%%%%%%%%%%%%%%

\bibliographystyle{model2-names}
%%\bibliography{<your-bib-database>}
\bibliography{ProdFuncPrimer.bib}

%% Authors are advised to submit their bibtex database files. They are
%% requested to list a bibtex style file in the manuscript if they do
%% not want to use model2-names.bst.

%% References without bibTeX database:

% \begin{thebibliography}{00}

%% \bibitem must have one of the following forms:
%%   \bibitem[Jones et al.(1990)]{key}...
%%   \bibitem[Jones et al.(1990)Jones, Baker, and Williams]{key}...
%%   \bibitem[Jones et al., 1990]{key}...
%%   \bibitem[\protect\citeauthoryear{Jones, Baker, and Williams}{Jones
%%       et al.}{1990}]{key}...
%%   \bibitem[\protect\citeauthoryear{Jones et al.}{1990}]{key}...
%%   \bibitem[\protect\astroncite{Jones et al.}{1990}]{key}...
%%   \bibitem[\protect\citename{Jones et al., }1990]{key}...
%%   \harvarditem[Jones et al.]{Jones, Baker, and Williams}{1990}{key}...
%%

% \bibitem[ ()]{}

% \end{thebibliography}


%% The Appendices part is started with the command \appendix;
%% appendix sections are then done as normal sections
%% \appendix
%% \section{}
%% \label{}


%%%%%%%%%%%%%%%%%%%%%%%%%%%%%%%
\appendix
%%%%%%%%%%%%%%%%%%%%%%%%%%%%%%%

\section{Appendix}

Include data tables here?


\end{document}

%%
%% End of file `elsarticle-template-2-harv.tex'.
