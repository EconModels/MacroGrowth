% This article has been prepared for publication in RStudio with knitr.
% According to http://www.elsevier.com/author-schemas/the-elsarticle-latex-document-class, we should be using the
% elsarticle.cls file.
% According to http://cdn.elsevier.com/assets/pdf_file/0006/109392/journal_refstyles.pdf, we should be using
% elsarticle-template-2-harv.tex as the template for the text.
% Furthermore, we should be using model2-names.bst for the bibliographic references.
% The approach here is to load the frontmatter and backmatter from elsarticle-template-2-harv.tex
% both ahead of and behind the text for our paper.
% -- Matthew Kuperus Heun, 2014-08-01

%% This is file `elsarticle-template-2-harv.tex',
%%
%% Copyright 2009 Elsevier Ltd
%%
%% This file is part of the 'Elsarticle  Bundle'.
%% ---------------------------------------------
%%
%% It may be distributed under the conditions of the LaTeX Project Public
%% License, either version 1.2 of this license or (at your option) any
%% later version.  The latest version of this license is in
%%    http://www.latex-project.org/lppl.txt
%% and version 1.2 or later is part of all distributions of LaTeX
%% version 1999/12/01 or later.
%%
%% The list of all files belonging to the 'Elsarticle Bundle' is
%% given in the file `manifest.txt'.
%%
%% Template article for Elsevier's document class `elsarticle'
%% with harvard style bibliographic references
%%
%% $Id: elsarticle-template-2-harv.tex 155 2009-10-08 05:35:05Z rishi $
%% $URL: http://lenova.river-valley.com/svn/elsbst/trunk/elsarticle-template-2-harv.tex $
%%
\documentclass[preprint,authoryear,12pt]{elsarticle}\usepackage[]{graphicx}\usepackage[]{color}
%% maxwidth is the original width if it is less than linewidth
%% otherwise use linewidth (to make sure the graphics do not exceed the margin)
\makeatletter
\def\maxwidth{ %
  \ifdim\Gin@nat@width>\linewidth
    \linewidth
  \else
    \Gin@nat@width
  \fi
}
\makeatother

\definecolor{fgcolor}{rgb}{0.345, 0.345, 0.345}
\newcommand{\hlnum}[1]{\textcolor[rgb]{0.686,0.059,0.569}{#1}}%
\newcommand{\hlstr}[1]{\textcolor[rgb]{0.192,0.494,0.8}{#1}}%
\newcommand{\hlcom}[1]{\textcolor[rgb]{0.678,0.584,0.686}{\textit{#1}}}%
\newcommand{\hlopt}[1]{\textcolor[rgb]{0,0,0}{#1}}%
\newcommand{\hlstd}[1]{\textcolor[rgb]{0.345,0.345,0.345}{#1}}%
\newcommand{\hlkwa}[1]{\textcolor[rgb]{0.161,0.373,0.58}{\textbf{#1}}}%
\newcommand{\hlkwb}[1]{\textcolor[rgb]{0.69,0.353,0.396}{#1}}%
\newcommand{\hlkwc}[1]{\textcolor[rgb]{0.333,0.667,0.333}{#1}}%
\newcommand{\hlkwd}[1]{\textcolor[rgb]{0.737,0.353,0.396}{\textbf{#1}}}%

\usepackage{framed}
\makeatletter
\newenvironment{kframe}{%
 \def\at@end@of@kframe{}%
 \ifinner\ifhmode%
  \def\at@end@of@kframe{\end{minipage}}%
  \begin{minipage}{\columnwidth}%
 \fi\fi%
 \def\FrameCommand##1{\hskip\@totalleftmargin \hskip-\fboxsep
 \colorbox{shadecolor}{##1}\hskip-\fboxsep
     % There is no \\@totalrightmargin, so:
     \hskip-\linewidth \hskip-\@totalleftmargin \hskip\columnwidth}%
 \MakeFramed {\advance\hsize-\width
   \@totalleftmargin\z@ \linewidth\hsize
   \@setminipage}}%
 {\par\unskip\endMakeFramed%
 \at@end@of@kframe}
\makeatother

\definecolor{shadecolor}{rgb}{.97, .97, .97}
\definecolor{messagecolor}{rgb}{0, 0, 0}
\definecolor{warningcolor}{rgb}{1, 0, 1}
\definecolor{errorcolor}{rgb}{1, 0, 0}
\newenvironment{knitrout}{}{} % an empty environment to be redefined in TeX

\usepackage{alltt}

%% Use the option review to obtain double line spacing
%% \documentclass[authoryear,preprint,review,12pt]{elsarticle}

%% Use the options 1p,twocolumn; 3p; 3p,twocolumn; 5p; or 5p,twocolumn
%% for a journal layout:
%% \documentclass[final,authoryear,1p,times]{elsarticle}
%% \documentclass[final,authoryear,1p,times,twocolumn]{elsarticle}
%% \documentclass[final,authoryear,3p,times]{elsarticle}
%% \documentclass[final,authoryear,3p,times,twocolumn]{elsarticle}
%% \documentclass[final,authoryear,5p,times]{elsarticle}
%% \documentclass[final,authoryear,5p,times,twocolumn]{elsarticle}

%% if you use PostScript figures in your article
%% use the graphics package for simple commands
%% \usepackage{graphics}
%% or use the graphicx package for more complicated commands
%% \usepackage{graphicx}
%% or use the epsfig package if you prefer to use the old commands
%% \usepackage{epsfig}

%% The amssymb package provides various useful mathematical symbols
\usepackage{amssymb}
%% The amsthm package provides extended theorem environments
%% \usepackage{amsthm}

\usepackage{wrapfig}    % Allows wrapping of text around figures
\usepackage{soul}       % Provides strikethrough text
\usepackage{float}      % Allows precise positioning of tables and figures within the text
\usepackage{amsmath}    % Allows \begin{equation*} \end{equation*} for unnumbered equations
\usepackage{multirow}   % To create multirow tables
\usepackage{hyperref}   % To create hyperlinks in the paper
\usepackage{microtype}  % produces hanging punctuation and beautiful type
\usepackage{booktabs}   % for beautiful tables
\usepackage{dcolumn}    % aligns table columns at decimal places


% From http://economics.utoronto.ca/osborne/latex/BIBTEX.HTM
\newcommand{\citeapos}[1]{\citeauthor{#1}'s (\citeyear{#1})} % Posessive citations. 

%% The lineno packages adds line numbers. Start line numbering with
%% \begin{linenumbers}, end it with \end{linenumbers}. Or switch it on
%% for the whole article with \linenumbers after \end{frontmatter}.
%% \usepackage{lineno}

%% natbib.sty is loaded by default. However, natbib options can be
%% provided with \biboptions{...} command. Following options are
%% valid:

%%   round  -  round parentheses are used (default)
%%   square -  square brackets are used   [option]
%%   curly  -  curly braces are used      {option}
%%   angle  -  angle brackets are used    <option>
%%   semicolon  -  multiple citations separated by semi-colon (default)
%%   colon  - same as semicolon, an earlier confusion
%%   comma  -  separated by comma
%%   authoryear - selects author-year citations (default)
%%   numbers-  selects numerical citations
%%   super  -  numerical citations as superscripts
%%   sort   -  sorts multiple citations according to order in ref. list
%%   sort&compress   -  like sort, but also compresses numerical citations
%%   compress - compresses without sorting
%%   longnamesfirst  -  makes first citation full author list
%%
%% \biboptions{longnamesfirst,comma}

% \biboptions{}

\journal{Ecological Economics}
\IfFileExists{upquote.sty}{\usepackage{upquote}}{}
\begin{document}




\begin{frontmatter}

%% Title, authors and addresses

%% use the tnoteref command within \title for footnotes;
%% use the tnotetext command for the associated footnote;
%% use the fnref command within \author or \address for footnotes;
%% use the fntext command for the associated footnote;
%% use the corref command within \author for corresponding author footnotes;
%% use the cortext command for the associated footnote;
%% use the ead command for the email address,
%% and the form \ead[url] for the home page:
%%
%% \title{Title\tnoteref{label1}}
%% \tnotetext[label1]{}
%% \author{Name\corref{cor1}\fnref{label2}}
%% \ead{email address}
%% \ead[url]{home page}
%% \fntext[label2]{}
%% \cortext[cor1]{}
%% \address{Address\fnref{label3}}
%% \fntext[label3]{}

\title{Effect of quality-adjusted production function inputs}

%% use optional labels to link authors explicitly to addresses:
%% \author[label1,label2]{<author name>}
%% \address[label1]{<address>}
%% \address[label2]{<address>}

\author[IST]{Jo\~{a}o dos Santos}
\author[Leeds]{Paul Brockway}
\author[CalvinEngr]{Matthew Kuperus Heun\corref{cor1}}
\ead{mkh2@calvin.edu}
\author[Leeds]{Marco Sekai}
\author[IST]{Tiago Domingos}
\author[Leeds]{Julia Steinberger}

\address[IST]{IST, Lisbon, Portugal}
\address[Leeds]{Leeds University, Leeds, UK}
\address[CalvinEngr]{Engineering Department, Calvin College, Grand Rapids, MI 49546, USA}
\cortext[cor1]{Corresponding author}

\begin{abstract}
%% Text of abstract
**** Add abstract ****
\end{abstract}

\begin{keyword}
%% keywords here, in the form: keyword \sep keyword
economic growth \sep exergy \sep energy \sep Cobb-Douglas \sep CES
%% MSC codes here, in the form: \MSC code \sep code
%% or \MSC[2008] code \sep code (2000 is the default)
\end{keyword}

\end{frontmatter}

% \linenumbers
%% main text


%%%%%%%%%%%%%%%%%%%%%%%%%%%%%%%
\section{Introduction} 
\label{sec:Introduction}
%%%%%%%%%%%%%%%%%%%%%%%%%%%%%%%

This is a study of the effects of quality-adjusted production function inputs
on the importance of energy in the production function.


%%%%%%%%%%%%%%%%%%%%%%%%%%%%%%%
\section{Coordinates of Analysis} 
\label{sec:Coordinates}
%%%%%%%%%%%%%%%%%%%%%%%%%%%%%%%

This section describes the coordinates of analysis and briefly reviews 
literature related to each.


%++++++++++++++++++++++++++++++
\subsection{Mathematical Forms of the Energy-augmented Production Function} 
\label{sec:Prod_Func_Forms}
%++++++++++++++++++++++++++++++

In this paper, we assess two prominent energy-augmented production functions 
that appear in the literature: 
Cobb-Douglas (CD) and
Constant Elasticity of Substitution (CES).
These production functions are assessed relative to 
a model of exponential growth only.
The following subsections describe each.


%------------------------------
\subsubsection{Cobb-Douglas Production Function} 
\label{sec:CDe}
%------------------------------

The Cobb-Douglas production function can be expressed as
%
\begin{equation} \label{eq:CD}
  y = \theta A k^{\alpha_1} l^{\alpha_2} \; ; \; A \equiv \mathrm{e}^{\lambda(t-t_0)} \; ,
\end{equation}
%
where 
$y \equiv Y/Y_{0}$,
$\theta$ is a scale parameter,
e is the base of the natural logarithm, 
$\lambda$ is represents the pace of technological progress,
$t$ (time) is measured in years,
$k \equiv K/K_{0}$, 
$l \equiv L/L_{0}$, 
$Y$ (economic output) is represented by GDP, 
$K$ (capital) is expressed in currency units, 
$L$ (labor) is expressed in workers or work-hours/year, and
the 0 subscript indicates values at an initial year.%
  \footnote{Dimensionless, indexed quantities are represented by 
  lower-case symbols 
  ($y$, $k$, $l$, $e$, $q$, $x$, and $u$), and dimensional 
  quantities are represented by upper-case symbols 
  ($Y$, $K$, $L$, $E$, $Q$, $X$, and $U$). 
  Model parameters are represented by Greek letters
  ($\alpha_1$, $\alpha_2$, $\lambda$, $\theta$).
  }
Constant returns to scale are represented by the constraint
$\alpha_1 + \alpha_2 = 1$.

The capital-labor Cobb-Douglas production function shown in Equation~\ref{eq:CD}
can be augmented to include an energy term:
%
\begin{equation} \label{eq:CDe}
  y = \theta A k^{\alpha_1} l^{\alpha_2} e^{\alpha_3} \; ; \; A \equiv \mathrm{e}^{\lambda(t-t_0)} \; ,
\end{equation}
%
where $e \equiv E/E_0$, and $E$ is in units of energy per time, typically TJ/year.
The energy-augmented Cobb-Douglas production function 
is often assumed to have constant returns to scale for the three factors 
of production: $\alpha_1 + \alpha_2 + \alpha_3 = 1$.
The term $A$ is known as total factor productivity,
and $\lambda$ is the Solow residual. 


%------------------------------
\subsubsection{Constant Elasticity of Substitution Production Function (CES)} 
\label{sec:CES}
%------------------------------

Other energy economists use an energy-augmented 
Constant Elasticity of Substitution (CES) production function to 
describe economic growth. The \texttt{R} 
% \citep{R} 
package \texttt{micEconCES} 
% \citep{Henningsen:2011td} 
estimates CES production 
functions of the following form
%
\begin{equation} \label{eq:CESgeneric}
  y = \gamma \: A \: \left\{\delta \left[\delta_1 x_1^{-\rho_1} 
      + (1-\delta_1)x_2^{-\rho_1} \right]^{\rho/\rho_1} 
      + (1-\delta) x_3^{-\rho} \right\}^{-1/\rho}; 
      A \equiv \mathrm{e}^{\lambda (t-t_0)},
\end{equation}
%
where $x_1$, $x_2$, and $x_3$ are factors of production
and permutations of capital~($k$), labor~($l$), and energy~($e$).

The CES model without energy is given in Equation~\ref{eq:CESkl}.
%
\begin{equation} \label{eq:CESkl}
  y = \gamma \: A \: \left[\delta_1 k^{-\rho_1} 
      + (1-\delta_1)l^{-\rho_1} \right]^{-1/\rho_1}; 
      A \equiv \mathrm{e}^{\lambda (t-t_0)} . 
\end{equation}
%
Equation~\ref{eq:CESkle} augments Equation~\ref{eq:CESkl} with energy 
using a ($kl$)($e$) nesting structure, as is typical in the literature. 
Equation~\ref{eq:CESkl} is a degenerate form of Equation~\ref{eq:CESkle} 
where $\delta \rightarrow 1$. 
%
\begin{equation} \label{eq:CESkle}
  y = \gamma \: A \: \left\{\delta \left[\delta_1 k^{-\rho_1} 
      + (1-\delta_1)l^{-\rho_1} \right]^{\rho/\rho_1} 
      + (1-\delta) e^{-\rho} \right\}^{-1/\rho}; 
      A \equiv \mathrm{e}^{\lambda (t-t_0)}.
\end{equation}
%

In the CES production function, 
$\gamma$ is a fitting parameter that accounts for an atypical first year. 
The fitting parameters $\rho_1$ and $\rho$ indicate elasticities 
of substitution ($\sigma_1$ and $\sigma$).
The elasticity of substitution 
between capital ($k$) and labor ($l$) is given by 
$\sigma_1 = \frac{1}{1+\rho_1}$, and
the elasticity of substitution between ($kl$) and ($e$) is given by 
$\sigma = \frac{1}{1+\rho}$. 
As $\rho_1 \rightarrow 0$, $\sigma_1 \rightarrow 1$,
and the embedded CES production function for $k$ and $l$ degenerates 
to the Cobb-Douglas production function.
Similarly, as $\rho \rightarrow 0$, $\sigma \rightarrow 1$,
and the CES production function for $(kl)$ and $(e)$ degenerates 
to the Cobb-Douglas production function.
As $\sigma \rightarrow \infty$ ($\rho \rightarrow -1$), 
($kl$) and ($e$) are perfect substitutes. 
As $\sigma \rightarrow 0$ ($\rho \rightarrow \infty$), 
($kl$) and ($e$) are perfect complements: 
no substitution is possible. Similarly, 
as $\sigma_1 \rightarrow 0$ ($\rho_1 \rightarrow \infty$),
$k$ and $l$ are perfect complements.
$\delta_1$ describes the relative importance of capital ($k$)
and labor ($l$), and
$\delta$ describes the importance of ($kl$) relative to ($e$).

Constraints on the fitting parameters include 
$\delta_1 \in [0,1]$,
$\delta \in [0,1]$,
$\rho_1 \in [-1,0) \cup (0,\infty)$, and
$\rho \in [-1,0) \cup (0,\infty)$. 

Two other nestings of the factors of production ($k$, $l$, and $e$)
are possible with the CES model.
%
\begin{equation} \label{eq:CESlek}
  y = \gamma \: A \: \left\{\delta \left[\delta_1 l^{-\rho_1} 
      + (1-\delta_1) e^{-\rho_1} \right]^{\rho/\rho_1} 
      + (1-\delta) k^{-\rho} \right\}^{-1/\rho}; 
      A \equiv \mathrm{e}^{\lambda (t-t_0)}
\end{equation}
%
\begin{equation} \label{eq:CESekl}
  y = \gamma \: A \: \left\{\delta \left[\delta_1 e^{-\rho_1} 
      + (1-\delta_1) k^{-\rho_1} \right]^{\rho/\rho_1} 
      + (1-\delta) l^{-\rho} \right\}^{-1/\rho}; 
      A \equiv \mathrm{e}^{\lambda (t-t_0)}
\end{equation}
% 
Note that $\rho$ ($\sigma$), $\rho_1$ ($\sigma_1$), $\delta$, and $\delta_1$ have different
meanings depending upon the nesting of the factors of production.


%------------------------------
\subsubsection{Exponential Production Function} 
\label{sec:exp}
%------------------------------

We define an exponential-only reference model for economic growth
in Equation~\ref{eq:exponential-only-model}.
%
\begin{equation} \label{eq:exponential-only-model}
  y = \theta A \; ; \; A \equiv \mathrm{e}^{\lambda(t-t_0)} \; .
\end{equation}
%
The reference model is a degenerate case of the Cobb-Douglas production function
wherein all factor shares are zero~($\alpha_1 = \alpha_2 = \alpha_3 = 0$)
and the constant returns to scale constraint is not respected.

We expect that the reference model will have a larger fitted Solow residual term than
the Cobb-Douglas and CES models, because no factors of production are
included in the reference model to drive growth.
Indeed, in the reference model, 
all growth is attributed to the Solow residual.

In contrast, it is not necessarily true that 
Cobb-Douglas and CES models 
will exhibit lower $mse$ than the reference model
shown in Equation~\ref{eq:exponential-only-model}.
The Cobb-Douglas and CES models have more fitting parameters,
but they incorporate the factors of production
at constant returns to scale. 
The reference model has $\alpha_1 + \alpha_2 + \alpha_3 = 0$,
whereas our implementations of both the Cobb-Douglas and CES models
requires constant returns to scale, $\alpha_1 + \alpha_2 + \alpha_3 = 1$.
Thus, if the factors of production ($\alpha_1$, $\alpha_2$, and $\alpha_3$) 
are poorly correlated to output ($y$),
the constant returns to scale constraint may cause higher $mse$ for
the Cobb-Douglas or CES models relative to the reference model.

In the sections that follow, 
we assess the Cobb-Douglas and CES models relative to the reference model
in terms of goodness of both goodness of fit~($mse$) and Solow residual~($\lambda$).


%++++++++++++++++++++++++++++++
\subsection{Economies} 
\label{sec:Economies}
%++++++++++++++++++++++++++++++

Discuss economies here.

UK and Portugal.


%%%%%%%%%%%%%%%%%%%%%%%%%%%%%%%
\section{Sources of Data}
\label{sec:Sources_of_Data}
%%%%%%%%%%%%%%%%%%%%%%%%%%%%%%%

Discuss data sources here.


%++++++++++++++++++++++++++++++
\subsection{Historical Data} 
\label{sec:Historical_Data}
%++++++++++++++++++++++++++++++

Historical data are stored in the \texttt{IST} and \texttt{Leeds} data sets.





For unadjusted variables, we use the time series shown in Table~\ref{tab:unadjusted_variables}.

\begin{table} \caption{Unadjusted time-series variables.} 
\label{tab:unadjusted_variables} 
  \begin{center}
    \begin{tabular}{r l l} 
      \toprule
      Variable    & Portugal           & UK                 \\
      \midrule
      Output      & iGDP   
                                                             & iGDP\\
      Capital     & iKstkS.L & iKstkO.WwithRD \\
      Labor       & iL & iL \\
      Energy      & iXpMP & iXp \\
      \bottomrule
    \end{tabular}
  \end{center}
\end{table}

For quality-adjusted variables, we use the time series shown in Table~\ref{tab:quality_adjusted_variables}.

\begin{table} \caption{Quality-adjusted time series variables.} 
\label{tab:quality_adjusted_variables} 
  \begin{center}
    \begin{tabular}{r l l} 
      \toprule
      Variable    & Portugal         & UK               \\
      \midrule
      Output      & iGDP   
                                                           & iGDP \\
      Capital     & iKservS.L & iKservO.WwithRD \\
      Labor       & ihLest & ihLest \\
      Energy      & iUMP & iU \\
      \bottomrule
    \end{tabular}
  \end{center}
\end{table}

Unadjusted and quality-adjusted data for both Portugal and the United Kingdom 
are shown in Figure~\ref{fig:Historical Data Graph}.

\begin{knitrout}
\definecolor{shadecolor}{rgb}{0.969, 0.969, 0.969}\color{fgcolor}\begin{figure}[H]

{\centering \includegraphics[width=\maxwidth]{figure/Historical_Data_Graph-1} 

}

\caption[Historical data]{Historical data.}\label{fig:Historical Data Graph}
\end{figure}


\end{knitrout}

\begin{knitrout}
\definecolor{shadecolor}{rgb}{0.969, 0.969, 0.969}\color{fgcolor}\begin{figure}[H]

{\centering \includegraphics[width=\maxwidth]{figure/Pre-econometric_graph-1} 

}

\caption[Pre-econometric data]{Pre-econometric data.}\label{fig:Pre-econometric graph}
\end{figure}


\end{knitrout}


%%%%%%%%%%%%%%%%%%%%%%%%%%%%%%%
\section{Parameter Estimation (Without Cost-share Theorem)}
\label{sec:parameter_estimation_noCST}
%%%%%%%%%%%%%%%%%%%%%%%%%%%%%%%

% Parameter estimation is accomplished with \texttt{formula}s of the form
% %
% \begin{equation} \label{eq:formulas}
%   y \sim x_1 + x_2 + x_3 + time,
% \end{equation}
% %
% where $y$ is economic output and 
% $x_1$, $x_2$, and $x_3$ are factors of production
% and permutations of
% capital ($k$), labor ($l$), and energy ($e$).
% (The factors of production may be quality-adjusted.)
% $time$ is the time variable, usually \texttt{iYear}. 
% The \texttt{formula}s are used in a fitting function such as 
% \texttt{cdModel} to create a \texttt{model} object.
% The function \texttt{naturalCoef} extracts reasonable
% coefficients from \texttt{model} objects.
% 
% Note that in the fitting process for the Cobb-Douglas equation,
% $\texttt{scale} = \theta$, 
% $\alpha_1 = \alpha$,
% $\alpha_2 = \beta$, and
% $\alpha_3 = \gamma$,
% when the \texttt{formula} is specified as 
% %
% \begin{equation} \label{eq:formula_kle}
%   \texttt{y} \sim \texttt{k + l + e + time}.
% \end{equation}
% %
% Note that for the CES model, $\texttt{scale} = \gamma$ and is expected to be close to unity. 

The models being fit can be described by the algebraic form of the model 
(Cobb-Douglas, CES, etc.) and a formula that enumerates 
which data variables are being used in which roles to fit the model.  
A formula of the form
%
\begin{equation} \label{eq:formulas-3-factor}
y \sim x_1 + x_2 + x_3 + t,
\end{equation}
%
or 
%
\begin{equation} \label{eq:formulas-2-factor}
y \sim x_1 + x_2 + t,
\end{equation}
%
describes the economic output variable ($y$, usually \texttt{iGDP},
                                        indexed GDP) 
and the factors of production ($x_1$, $x_2$, and $x_3$,
                               which will be some measure of capital, labor, and energy, 
                               but perhaps not in that order), 
and a time variable ($t$, usually \texttt{iYear}, 
                     the number of years since the beginning of data collection).  
All of the models assume an error term that is additive on the 
logarithmic scale and are fit by the method of least squares.  
Model fitting provides estimates for all parameters in the model.

For the CES model, values of $\alpha_1$, $\alpha_2$, and $\alpha_3$ are calculated by
$\alpha_1 = \delta_1$, $\alpha_2 = 1 - \delta_1$, and $\alpha_3 = 0$ 
for the CES model with two factors of production
and by 
$\alpha_1 = \delta \delta_1$, $\alpha_2 = \delta(1 - \delta_1)$, and $\alpha_3 = 1 - \delta$
for the CES model with three factors of production and the ($kl$)$e$ nesting.
The values of $\alpha_1$, $\alpha_2$, and $\alpha_3$ are interpreted as
factor shares for capital, labor, and energy, respectively,
as shown in Table \ref{tab:CES_abg_equations}, 
assuming factors of production are specified in the fitting \texttt{formula} as
\texttt{y~$\sim$~k~+~l~+~e~+~time}.

\begin{table} \caption{Equations for $\alpha_1$, $\alpha_2$, and $\alpha_3$ 
                        (factor shares for capital, labor, and energy, respectively)
                        for the various CES nestings, 
                        provided that the \texttt{formula} is specified as \texttt{y~$\sim$~k~+~l~+~e~+~time}.} 
\label{tab:CES_abg_equations} 
  \begin{center}
    \begin{tabular}{c c c c c} 
      \toprule
      Nesting         & \texttt{nest}     & $\alpha_1$              & $\alpha_2$              & $\alpha_3$       \\
      \midrule
      ($kl$) + ()     & \texttt{c(1,2)}   & $\delta_1$              & $1 - \delta_1$          & 0          \\
      ($kl$) + ($e$)  & \texttt{c(1,2,3)} & $\delta \delta_1$       & $\delta (1 - \delta_1)$ & $1 - \delta$ \\   
      ($le$) + ($k$)  & \texttt{c(2,3,1)} & $1 - \delta$            & $\delta \delta_1$       & $\delta (1 - \delta_1)$ \\
      ($ek$) + ($l$)  & \texttt{c(3,1,2)} & $\delta (1 - \delta_1)$ & $1 - \delta$            & $\delta \delta_1$ \\
      \bottomrule
    \end{tabular}
  \end{center}
\end{table}

\begin{knitrout}
\definecolor{shadecolor}{rgb}{0.969, 0.969, 0.969}\color{fgcolor}\begin{kframe}


{\ttfamily\noindent\bfseries\color{errorcolor}{Error in saveRDS(models, file = fileName): object 'fileName' not found}}\end{kframe}
\end{knitrout}




%%%%%%%%%%%%%%%%%%%%%%%%%%%%%%%
\section{Results}
\label{sec:results}
%%%%%%%%%%%%%%%%%%%%%%%%%%%%%%%

%++++++++++++++++++++++++++++++
\subsection{Fits to historical data} 
\label{sec:fits_to_historical_data}
%++++++++++++++++++++++++++++++



Both historical GDP and fitted GDP are shown in 
Figures~\ref{fig:Graphs CD-fitted}--\ref{fig:Graphs CESekl-fitted}.
%
\begin{knitrout}
\definecolor{shadecolor}{rgb}{0.969, 0.969, 0.969}\color{fgcolor}\begin{figure}[H]

{\centering \includegraphics[width=\maxwidth]{figure/Graphs_CD-fitted-1} 

}

\caption[Cobb-Douglas models that reject the cost-share theorem]{Cobb-Douglas models that reject the cost-share theorem. Solid line is historical GDP.}\label{fig:Graphs CD-fitted}
\end{figure}


\end{knitrout}
%
\begin{knitrout}
\definecolor{shadecolor}{rgb}{0.969, 0.969, 0.969}\color{fgcolor}\begin{figure}[H]

{\centering \includegraphics[width=\maxwidth]{figure/Graphs_CESkle-fitted-1} 

}

\caption[CES models with (]{CES models with ($kl$)$e$ nesting. Solid line is historical GDP.}\label{fig:Graphs CESkle-fitted}
\end{figure}


\end{knitrout}
%
\begin{knitrout}
\definecolor{shadecolor}{rgb}{0.969, 0.969, 0.969}\color{fgcolor}\begin{figure}[H]

{\centering \includegraphics[width=\maxwidth]{figure/Graphs_CESlek-fitted-1} 

}

\caption[CES models with (]{CES models with ($le$)$k$ nesting. Solid line is historical GDP.}\label{fig:Graphs CESlek-fitted}
\end{figure}


\end{knitrout}
%
\begin{knitrout}
\definecolor{shadecolor}{rgb}{0.969, 0.969, 0.969}\color{fgcolor}\begin{figure}[H]

{\centering \includegraphics[width=\maxwidth]{figure/Graphs_CESekl-fitted-1} 

}

\caption[CES models with (]{CES models with ($ek$)$l$ nesting. Solid line is historical GDP.}\label{fig:Graphs CESekl-fitted}
\end{figure}


\end{knitrout}


%++++++++++++++++++++++++++++++
\subsection{Fitting residuals} 
\label{sec:fitting_residuals}
%++++++++++++++++++++++++++++++



Because we fit in log-space, fitting residuals ($r_i$) are defined as
%
\begin{equation} \label{eq:log-residuals}
  r_i \equiv \ln(y_i) - \ln(\hat{y}_i) =\ln\left( \frac{y_i}{\hat{y}_i} \right), 
\end{equation}
%
where $r_i$ will be zero when there is agreement between 
historical~($y_i$) and fitted~($\hat{y}_i$) economic output.

The mean squared error~($mse$) for any fitted model can be calculated by
%
\begin{equation} \label{eq:mse}
  mse \equiv \frac{1}{N} \sum_{i=1}^N{r_i^2} .
\end{equation}

Figures~\ref{fig:Graphs CD-resid}--\ref{fig:Graphs CESekl-resid}
show fitting residuals for all models.
%
\begin{knitrout}
\definecolor{shadecolor}{rgb}{0.969, 0.969, 0.969}\color{fgcolor}\begin{figure}[H]

{\centering \includegraphics[width=\maxwidth]{figure/Graphs_CD-resid-1} 

}

\caption[Fitting residuals for Cobb-Douglas models]{Fitting residuals for Cobb-Douglas models.}\label{fig:Graphs CD-resid}
\end{figure}


\end{knitrout}
%
\begin{knitrout}
\definecolor{shadecolor}{rgb}{0.969, 0.969, 0.969}\color{fgcolor}\begin{figure}[H]

{\centering \includegraphics[width=\maxwidth]{figure/Graphs_CESkle-resid-1} 

}

\caption[Fitting residuals for CES models with ]{Fitting residuals for CES models with $(kl$)$e$ nesting.}\label{fig:Graphs CESkle-resid}
\end{figure}


\end{knitrout}
%
\begin{knitrout}
\definecolor{shadecolor}{rgb}{0.969, 0.969, 0.969}\color{fgcolor}\begin{figure}[H]

{\centering \includegraphics[width=\maxwidth]{figure/Graphs_CESlek-resid-1} 

}

\caption[Fitting residuals for CES models with (]{Fitting residuals for CES models with ($le$)$k$ nesting.}\label{fig:Graphs CESlek-resid}
\end{figure}


\end{knitrout}
%
\begin{knitrout}
\definecolor{shadecolor}{rgb}{0.969, 0.969, 0.969}\color{fgcolor}\begin{figure}[H]

{\centering \includegraphics[width=\maxwidth]{figure/Graphs_CESekl-resid-1} 

}

\caption[Fitting residuals for CES models with ]{Fitting residuals for CES models with $(ek$)$l$ nesting.}\label{fig:Graphs CESekl-resid}
\end{figure}


\end{knitrout}





%%%%%%%%%%%%%%%%%%%%%%%%%%%%%%%
\section{Analysis}
\label{sec:analysis}
%%%%%%%%%%%%%%%%%%%%%%%%%%%%%%%

For now, I've organized the analysis around questions that we would like to answer.


%++++++++++++++++++++++++++++++
\subsection{Does rejecting the cost-share theorem 
            decrease the mean squared error and/or the Solow residual?} 
\label{sec:cst_and_solow_residual}
%++++++++++++++++++++++++++++++

Our hypothesis is that rejecting the cost-share theorem~(CST) will decrease
both mean squared error~($mse$) and
the Solow residual~($\lambda$).
The mean squared error~($mse$) is expected to decrease 
because rejecting the CST adds one or more fitting parameters
to the model.
The Solow residual~($\lambda$) is expected to decrease 
because the factor shares for capital~($\alpha_1$) and labor~($\alpha_2 = 1 - \alpha_1$)
are free to float, 
allowing each factor of production 
to contribute optimally toward production and
bringing the model's prediction for economic output closer to 
historical output. 
In so doing, we expect less ``unexplained'' economic growth 
and a decreased Solow residual~($\lambda$).

Table~\ref{tab:Effect_of_CST_CD} shows the effect of the CST 
on fitted parameters for the Cobb-Douglas model and the exponential-only model.
The Cobb-Douglas models show lower $mse$ than the exponential-only models.
The Cobb-Douglas models also reduce solow residual ($\lambda$) relative
to the exponential-only case.
%
% latex table generated in R 3.2.0 by xtable 1.7-4 package
% Wed Jul  8 15:43:56 2015
\begin{table}[ht]
\centering
\caption{Model parameters for Cobb-Douglas models with unadjusted factors of production, without energy.} 
\label{tab:Effect_of_CST_CD}
{\tiny
\begin{tabular}{lllrrrrr}
  \hline
Country & model & cst & scale & lambda & alpha\_1 & alpha\_2 & mse \\ 
  \hline
PT & exp & Reject CST & 1.283537 & 0.034507 & 0.000000 & 0.000000 & 0.013359 \\ 
  PT & CD & Adhere to CST & 1.328501 & 0.016080 & 0.300000 & 0.700000 & 0.011724 \\ 
  PT & CD & Reject CST & 1.113489 & $-$0.005539 & 1.000000 & 0.000000 & 0.003190 \\ 
  UK & exp & Reject CST & 1.021833 & 0.023565 & 0.000000 & 0.000000 & 0.001212 \\ 
  UK & CD & Adhere to CST & 1.032978 & 0.015510 & 0.300000 & 0.700000 & 0.000775 \\ 
  UK & CD & Reject CST & 1.000925 & 0.006654 & 0.544956 & 0.455044 & 0.000494 \\ 
   \hline
\end{tabular}
}
\end{table}

%
Table~\ref{tab:Effect_of_CST_CES} shows the effect of the CST 
on fitted parameters for the CES model.
The CES models show lower $mse$ than the exponential-only models.
The CES models also reduce solow residual ($\lambda$) relative
to the exponential-only case.
%
% latex table generated in R 3.2.0 by xtable 1.7-4 package
% Wed Jul  8 15:43:56 2015
\begin{table}[ht]
\centering
\caption{Model parameters for CES models with unadjusted factors of production, without energy.} 
\label{tab:Effect_of_CST_CES}
{\tiny
\begin{tabular}{lllrrrrrr}
  \hline
Country & model & cst & gamma & lambda & alpha\_1 & alpha\_2 & sigma\_1 & mse \\ 
  \hline
PT & exp & Reject CST &  & 0.034507 & 0.000000 & 0.000000 &  & 0.013359 \\ 
  PT & CES & Adhere to CST & 1.318284 & 0.018825 & 0.300000 & 0.700000 & 0.665163 & 0.011450 \\ 
  PT & CES & Reject CST & 0.998729 & 0.012404 & 0.983352 & 0.016648 & 0.173623 & 0.001113 \\ 
  UK & exp & Reject CST &  & 0.023565 & 0.000000 & 0.000000 &  & 0.001212 \\ 
  UK & CES & Adhere to CST & 1.006734 & 0.020914 & 0.300000 & 0.700000 & 0.509524 & 0.000491 \\ 
  UK & CES & Reject CST & 0.977872 & 0.017904 & 0.436298 & 0.563702 & 0.500000 & 0.000326 \\ 
   \hline
\end{tabular}
}
\end{table}

%
Both Cobb-Douglas and CES models do a better job of fitting 
historical data than an exponential-only model
as evidenced by decreased $mse$ values.
Both Cobb-Douglas and CES models exhibit reduced Solow residual~($\lambda$
compared to the exponential-only model.
For both Cobb-Douglas and CES, rejecting the Cost Share Theorem yields
decreased mean squared error~($mse$) and
smaller Solow residual ($\lambda$, on an absolute-value basis),
as expected.

***************

Note that in Table~\ref{tab:Effect_of_CST_CES}, we see $\sigma_1 = 0.5$ 
for UK when the cost share theorem is rejected. 
This even-numbered result is coming from the grid search in $\rho_1$. 
For an unknown reason, the gradient search from the best grid search point
fails.
I will investigate.

This result means that we have found a set of fitting coefficients
for this situation 
that is \emph{close} to providing the minimum possible $mse$,
but it may not be the \emph{exact} minimum.
Regardless, the point above (that $mse$ and $\lambda$ always decrease 
when rejecting the cost share theorem) remains valid.
Investigating and fixing this problem
will only serve to \emph{further} reduce $mse$ and $\lambda$ 
from the values reported in Table~\ref{tab:Effect_of_CST_CES}.

************


%++++++++++++++++++++++++++++++
\subsection{Does quality-adjusting the factors of production 
            decrease the mean squared error and/or the Solow residual?} 
\label{sec:quality_adj_energy_and_solow_residual}
%++++++++++++++++++++++++++++++

We can test our hypothesis that 
quality-adjusting the factors of production
and including energy will both reduce the Solow residuals~($\lambda$)
and improve the fit to historical data
(thereby reducing the fitting residuals, $r_i$, and mean squared error, $mse$)
by calculating 
$\Delta \lambda$ and $\Delta mse$, where
%
\begin{equation} \label{eq:delta_lambda}
  \Delta \lambda \equiv \left| \lambda \right| - \left| \lambda_{Unadjusted, \, Without \, energy, \, CST} \right|
\end{equation}
%
and
%
\begin{equation} \label{eq:delta_mse}
  \Delta mse \equiv mse - mse_{Unadjusted, \, Without \, energy, \, CST}.
\end{equation}
%
where~$N$ is the number of years of data.
On a percentage basis,
%
\begin{equation} \label{eq:delta_lambda_perc}
  \Delta \lambda \, [\%] = \frac{100 \, \Delta \lambda}{\left| \lambda_{Unadjusted, \, Without \, energy, \, CST} \right|} 
            = \frac{100 \, \left| \lambda \right|}{\left| \lambda_{Unadjusted, \, Without \, energy, \, CST} \right|} - 1
\end{equation}
%
When $\Delta \lambda$, $\Delta \lambda \, [\%]$, or $\Delta mse$ are negative, 
we observe reduction in the Solow residual ($\lambda$) or
the fitting residuals ($mse$)
relative to the unadjusted, no-energy, CST case.

To calculate $\Delta \lambda$ and $\Delta mse$, we need to fit all combinations of 
country, model, flavor, energy, nest, and CST.
Table~\ref{tab:CD_coeffs} shows coefficients for all fitted Cobb-Douglas models
and the reference model (exponential-only).
%
% latex table generated in R 3.2.0 by xtable 1.7-4 package
% Wed Jul  8 15:43:56 2015
\begin{table}[ht]
\centering
\caption{Model parameters for all Cobb-Douglas models.} 
\label{tab:CD_coeffs}
{\tiny
\begin{tabular}{lllllrrrrrr}
  \hline
Country & model & flavor & energy & cst & scale & lambda & alpha\_1 & alpha\_2 & alpha\_3 & mse \\ 
  \hline
PT & exp &  & Without energy & Reject CST & 1.283537 & 0.034507 & 0.000000 & 0.000000 & 0.000000 & 0.013359 \\ 
  PT & CD & Unadjusted & Without energy & Adhere to CST & 1.328501 & 0.016080 & 0.300000 & 0.700000 & 0.000000 & 0.011724 \\ 
  PT & CD & Unadjusted & Without energy & Reject CST & 1.113489 & $-$0.005539 & 1.000000 & 0.000000 & 0.000000 & 0.003190 \\ 
  PT & CD & Unadjusted & With energy & Reject CST & 1.130790 & $-$0.004439 & 0.755978 & 0.000000 & 0.244022 & 0.002628 \\ 
  PT & CD & Quality-adjusted & Without energy & Adhere to CST & 1.364226 & 0.004183 & 0.300000 & 0.700000 & 0.000000 & 0.011340 \\ 
  PT & CD & Quality-adjusted & Without energy & Reject CST & 1.231989 & $-$0.015079 & 1.000000 & 0.000000 & 0.000000 & 0.008937 \\ 
  PT & CD & Quality-adjusted & With energy & Reject CST & 1.111839 & $-$0.004569 & 0.000000 & 0.000000 & 1.000000 & 0.002268 \\ 
  UK & exp &  & Without energy & Reject CST & 1.021833 & 0.023565 & 0.000000 & 0.000000 & 0.000000 & 0.001212 \\ 
  UK & CD & Unadjusted & Without energy & Adhere to CST & 1.032978 & 0.015510 & 0.300000 & 0.700000 & 0.000000 & 0.000775 \\ 
  UK & CD & Unadjusted & Without energy & Reject CST & 1.000925 & 0.006654 & 0.544956 & 0.455044 & 0.000000 & 0.000494 \\ 
  UK & CD & Unadjusted & With energy & Reject CST & 0.995030 & 0.018134 & 0.162500 & 0.460325 & 0.377175 & 0.000331 \\ 
  UK & CD & Quality-adjusted & Without energy & Adhere to CST & 1.038384 & 0.008725 & 0.300000 & 0.700000 & 0.000000 & 0.001564 \\ 
  UK & CD & Quality-adjusted & Without energy & Reject CST & 1.015146 & $-$0.008149 & 0.751058 & 0.248942 & 0.000000 & 0.000720 \\ 
  UK & CD & Quality-adjusted & With energy & Reject CST & 0.970614 & 0.001981 & 0.403433 & 0.330248 & 0.266319 & 0.000453 \\ 
   \hline
\end{tabular}
}
\end{table}

%
Tables~\ref{tab:CES_coeffs} and~\ref{tab:CES_coeffs_sigmas} show 
coefficients for all fitted CES models and the reference model (exponential-only).
%
% latex table generated in R 3.2.0 by xtable 1.7-4 package
% Wed Jul  8 15:43:56 2015
\begin{table}[ht]
\centering
\caption{Model parameters for all CES models.} 
\label{tab:CES_coeffs}
{\tiny
\begin{tabular}{llllllrrrrr}
  \hline
Country & model & flavor & energy & nest & cst & gamma & lambda & alpha\_1 & alpha\_2 & alpha\_3 \\ 
  \hline
PT & exp &  & Without energy &  & Reject CST &  & 0.034507 & 0.000000 & 0.000000 & 0.000000 \\ 
  PT & CES & Unadjusted & Without energy & kl & Adhere to CST & 1.318284 & 0.018825 & 0.300000 & 0.700000 & 0.000000 \\ 
  PT & CES & Unadjusted & Without energy & kl & Reject CST & 0.998729 & 0.012404 & 0.983352 & 0.016648 & 0.000000 \\ 
  PT & CES & Unadjusted & With energy & kle & Reject CST & 1.022631 & 0.009738 & 0.727922 & 0.000064 & 0.272014 \\ 
  PT & CES & Unadjusted & With energy & lek & Reject CST & 1.005212 & 0.008318 & 1.000000 & 0.000000 & 0.000000 \\ 
  PT & CES & Unadjusted & With energy & ekl & Reject CST & 1.009966 & 0.010742 & 0.983292 & 0.016708 & 0.000000 \\ 
  PT & CES & Quality-adjusted & Without energy & kl & Adhere to CST & 1.335143 & 0.009131 & 0.300000 & 0.700000 & 0.000000 \\ 
  PT & CES & Quality-adjusted & Without energy & kl & Reject CST & 1.017672 & 0.005739 & 1.000000 & 0.000000 & 0.000000 \\ 
  PT & CES & Quality-adjusted & With energy & kle & Reject CST & 1.027582 & 0.004508 & 0.902835 & 0.000000 & 0.097165 \\ 
  PT & CES & Quality-adjusted & With energy & lek & Reject CST & 1.016454 & 0.005790 & 1.000000 & 0.000000 & 0.000000 \\ 
  PT & CES & Quality-adjusted & With energy & ekl & Reject CST & 1.017327 & 0.005789 & 1.000000 & 0.000000 & 0.000000 \\ 
  UK & exp &  & Without energy &  & Reject CST &  & 0.023565 & 0.000000 & 0.000000 & 0.000000 \\ 
  UK & CES & Unadjusted & Without energy & kl & Adhere to CST & 1.006734 & 0.020914 & 0.300000 & 0.700000 & 0.000000 \\ 
  UK & CES & Unadjusted & Without energy & kl & Reject CST & 0.977872 & 0.017904 & 0.436298 & 0.563702 & 0.000000 \\ 
  UK & CES & Unadjusted & With energy & kle & Reject CST & 0.989712 & 0.013857 & 0.445243 & 0.554757 & 0.000000 \\ 
  UK & CES & Unadjusted & With energy & lek & Reject CST & 0.982919 & 0.020230 & 0.570049 & 0.134387 & 0.295564 \\ 
  UK & CES & Unadjusted & With energy & ekl & Reject CST & 0.985088 & 0.022682 & 0.254298 & 0.738243 & 0.007460 \\ 
  UK & CES & Quality-adjusted & Without energy & kl & Adhere to CST & 1.008638 & 0.013592 & 0.300000 & 0.700000 & 0.000000 \\ 
  UK & CES & Quality-adjusted & Without energy & kl & Reject CST & 0.983725 & 0.001956 & 0.628728 & 0.371272 & 0.000000 \\ 
  UK & CES & Quality-adjusted & With energy & kle & Reject CST & 0.980315 & 0.007103 & 0.426684 & 0.421804 & 0.151512 \\ 
  UK & CES & Quality-adjusted & With energy & lek & Reject CST & 0.968799 & 0.007115 & 0.469167 & 0.144017 & 0.386817 \\ 
  UK & CES & Quality-adjusted & With energy & ekl & Reject CST & 0.981695 & 0.011338 & 0.305749 & 0.642022 & 0.052229 \\ 
   \hline
\end{tabular}
}
\end{table}

%
% latex table generated in R 3.2.0 by xtable 1.7-4 package
% Wed Jul  8 15:43:56 2015
\begin{table}[ht]
\centering
\caption{Model parameters for all CES models.} 
\label{tab:CES_coeffs_sigmas}
{\tiny
\begin{tabular}{llllllrrr}
  \hline
Country & model & flavor & energy & nest & cst & sigma\_1 & sigma & mse \\ 
  \hline
PT & exp &  & Without energy &  & Reject CST &  &  & 0.013359 \\ 
  PT & CES & Unadjusted & Without energy & kl & Adhere to CST & 0.665163 &  & 0.011450 \\ 
  PT & CES & Unadjusted & Without energy & kl & Reject CST & 0.173623 &  & 0.001113 \\ 
  PT & CES & Unadjusted & With energy & kle & Reject CST & 0.078512 & 1.747120 & 0.000746 \\ 
  PT & CES & Unadjusted & With energy & lek & Reject CST & 0.398887 & 0.013645 & 0.000588 \\ 
  PT & CES & Unadjusted & With energy & ekl & Reject CST & 0.005193 & 0.186281 & 0.000985 \\ 
  PT & CES & Quality-adjusted & Without energy & kl & Adhere to CST & 0.399944 &  & 0.009694 \\ 
  PT & CES & Quality-adjusted & Without energy & kl & Reject CST & 0.017511 &  & 0.000804 \\ 
  PT & CES & Quality-adjusted & With energy & kle & Reject CST & 0.017991 &     Inf & 0.000792 \\ 
  PT & CES & Quality-adjusted & With energy & lek & Reject CST & 0.017595 & 0.005358 & 0.000812 \\ 
  PT & CES & Quality-adjusted & With energy & ekl & Reject CST & 0.017013 &  & 0.000817 \\ 
  UK & exp &  & Without energy &  & Reject CST &  &  & 0.001212 \\ 
  UK & CES & Unadjusted & Without energy & kl & Adhere to CST & 0.509524 &  & 0.000491 \\ 
  UK & CES & Unadjusted & Without energy & kl & Reject CST & 0.500000 &  & 0.000326 \\ 
  UK & CES & Unadjusted & With energy & kle & Reject CST & 0.720815 & 0.009391 & 0.000223 \\ 
  UK & CES & Unadjusted & With energy & lek & Reject CST &     Inf & 0.465188 & 0.000289 \\ 
  UK & CES & Unadjusted & With energy & ekl & Reject CST & 0.103338 & 1.340508 & 0.000281 \\ 
  UK & CES & Quality-adjusted & Without energy & kl & Adhere to CST & 0.545219 &  & 0.001207 \\ 
  UK & CES & Quality-adjusted & Without energy & kl & Reject CST & 0.607601 &  & 0.000571 \\ 
  UK & CES & Quality-adjusted & With energy & kle & Reject CST & 0.611818 & 0.014357 & 0.000267 \\ 
  UK & CES & Quality-adjusted & With energy & lek & Reject CST &     Inf & 0.611994 & 0.000380 \\ 
  UK & CES & Quality-adjusted & With energy & ekl & Reject CST & 0.190138 & 1.247783 & 0.000301 \\ 
   \hline
\end{tabular}
}
\end{table}

%
Table~\ref{tab:DSRmse_CD} shows $\Delta \lambda$ and $\Delta mse$ results for the Cobb-Douglas models.
%
% latex table generated in R 3.2.0 by xtable 1.7-4 package
% Wed Jul  8 15:43:56 2015
\begin{table}[ht]
\centering
\caption{$\Delta \lambda$ and $\Delta mse$ for Cobb-Douglas models.} 
\label{tab:DSRmse_CD}
{\tiny
\begin{tabular}{lllllrr}
  \hline
Country & model & flavor & energy & cst & Dlambda & Dmse \\ 
  \hline
PT & CD & Unadjusted & Without energy & Adhere to CST & $-$0.018427 & $-$0.001635 \\ 
  PT & CD & Unadjusted & Without energy & Reject CST & $-$0.028968 & $-$0.010168 \\ 
  PT & CD & Unadjusted & With energy & Reject CST & $-$0.030068 & $-$0.010730 \\ 
  PT & CD & Quality-adjusted & Without energy & Adhere to CST & $-$0.030324 & $-$0.002019 \\ 
  PT & CD & Quality-adjusted & Without energy & Reject CST & $-$0.019428 & $-$0.004422 \\ 
  PT & CD & Quality-adjusted & With energy & Reject CST & $-$0.029938 & $-$0.011091 \\ 
  UK & CD & Unadjusted & Without energy & Adhere to CST & $-$0.008055 & $-$0.000437 \\ 
  UK & CD & Unadjusted & Without energy & Reject CST & $-$0.016911 & $-$0.000717 \\ 
  UK & CD & Unadjusted & With energy & Reject CST & $-$0.005432 & $-$0.000880 \\ 
  UK & CD & Quality-adjusted & Without energy & Adhere to CST & $-$0.014840 & 0.000352 \\ 
  UK & CD & Quality-adjusted & Without energy & Reject CST & $-$0.015416 & $-$0.000492 \\ 
  UK & CD & Quality-adjusted & With energy & Reject CST & $-$0.021584 & $-$0.000758 \\ 
   \hline
\end{tabular}
}
\end{table}

% 
Tables~\ref{tab:DSRmse_CESkle}--\ref{tab:DSRmse_CESekl} show $\Delta \lambda$ and $\Delta mse$ 
results for CES models with various nestings.
%
% latex table generated in R 3.2.0 by xtable 1.7-4 package
% Wed Jul  8 15:43:56 2015
\begin{table}[ht]
\centering
\caption{$\Delta \lambda$ and $\Delta mse$ for CES models with ($kl$)$e$ nesting.} 
\label{tab:DSRmse_CESkle}
{\tiny
\begin{tabular}{llllllrr}
  \hline
Country & model & flavor & energy & nest & cst & Dlambda & Dmse \\ 
  \hline
PT & exp &  & Without energy &  & Reject CST & 0.000000 & 0.000000 \\ 
  PT & CES & Unadjusted & Without energy & kl & Adhere to CST & $-$0.015683 & $-$0.001908 \\ 
  PT & CES & Unadjusted & Without energy & kl & Reject CST & $-$0.022103 & $-$0.012246 \\ 
  PT & CES & Unadjusted & With energy & kle & Reject CST & $-$0.024769 & $-$0.012613 \\ 
  PT & CES & Quality-adjusted & Without energy & kl & Adhere to CST & $-$0.025377 & $-$0.003665 \\ 
  PT & CES & Quality-adjusted & Without energy & kl & Reject CST & $-$0.028768 & $-$0.012555 \\ 
  PT & CES & Quality-adjusted & With energy & kle & Reject CST & $-$0.029999 & $-$0.012566 \\ 
  UK & exp &  & Without energy &  & Reject CST & 0.000000 & 0.000000 \\ 
  UK & CES & Unadjusted & Without energy & kl & Adhere to CST & $-$0.002651 & $-$0.000721 \\ 
  UK & CES & Unadjusted & Without energy & kl & Reject CST & $-$0.005661 & $-$0.000886 \\ 
  UK & CES & Unadjusted & With energy & kle & Reject CST & $-$0.009708 & $-$0.000989 \\ 
  UK & CES & Quality-adjusted & Without energy & kl & Adhere to CST & $-$0.009973 & $-$0.000004 \\ 
  UK & CES & Quality-adjusted & Without energy & kl & Reject CST & $-$0.021609 & $-$0.000641 \\ 
  UK & CES & Quality-adjusted & With energy & kle & Reject CST & $-$0.016463 & $-$0.000944 \\ 
   \hline
\end{tabular}
}
\end{table}

%
% latex table generated in R 3.2.0 by xtable 1.7-4 package
% Wed Jul  8 15:43:56 2015
\begin{table}[ht]
\centering
\caption{$\Delta \lambda$ and $\Delta mse$ for CES models with ($le$)$k$ nesting.} 
\label{tab:DSRmse_CESlek}
{\tiny
\begin{tabular}{llllllrr}
  \hline
Country & model & flavor & energy & nest & cst & Dlambda & Dmse \\ 
  \hline
PT & exp &  & Without energy &  & Reject CST & 0.000000 & 0.000000 \\ 
  PT & CES & Unadjusted & Without energy & kl & Adhere to CST & $-$0.015683 & $-$0.001908 \\ 
  PT & CES & Unadjusted & Without energy & kl & Reject CST & $-$0.022103 & $-$0.012246 \\ 
  PT & CES & Unadjusted & With energy & lek & Reject CST & $-$0.026189 & $-$0.012771 \\ 
  PT & CES & Quality-adjusted & Without energy & kl & Adhere to CST & $-$0.025377 & $-$0.003665 \\ 
  PT & CES & Quality-adjusted & Without energy & kl & Reject CST & $-$0.028768 & $-$0.012555 \\ 
  PT & CES & Quality-adjusted & With energy & lek & Reject CST & $-$0.028717 & $-$0.012546 \\ 
  UK & exp &  & Without energy &  & Reject CST & 0.000000 & 0.000000 \\ 
  UK & CES & Unadjusted & Without energy & kl & Adhere to CST & $-$0.002651 & $-$0.000721 \\ 
  UK & CES & Unadjusted & Without energy & kl & Reject CST & $-$0.005661 & $-$0.000886 \\ 
  UK & CES & Unadjusted & With energy & lek & Reject CST & $-$0.003335 & $-$0.000923 \\ 
  UK & CES & Quality-adjusted & Without energy & kl & Adhere to CST & $-$0.009973 & $-$0.000004 \\ 
  UK & CES & Quality-adjusted & Without energy & kl & Reject CST & $-$0.021609 & $-$0.000641 \\ 
  UK & CES & Quality-adjusted & With energy & lek & Reject CST & $-$0.016450 & $-$0.000831 \\ 
   \hline
\end{tabular}
}
\end{table}

%
% latex table generated in R 3.2.0 by xtable 1.7-4 package
% Wed Jul  8 15:43:56 2015
\begin{table}[ht]
\centering
\caption{$\Delta \lambda$ and $\Delta mse$ for CES models with ($ek$)$l$ nesting.} 
\label{tab:DSRmse_CESekl}
{\tiny
\begin{tabular}{llllllrr}
  \hline
Country & model & flavor & energy & nest & cst & Dlambda & Dmse \\ 
  \hline
PT & exp &  & Without energy &  & Reject CST & 0.000000 & 0.000000 \\ 
  PT & CES & Unadjusted & Without energy & kl & Adhere to CST & $-$0.015683 & $-$0.001908 \\ 
  PT & CES & Unadjusted & Without energy & kl & Reject CST & $-$0.022103 & $-$0.012246 \\ 
  PT & CES & Unadjusted & With energy & ekl & Reject CST & $-$0.023766 & $-$0.012374 \\ 
  PT & CES & Quality-adjusted & Without energy & kl & Adhere to CST & $-$0.025377 & $-$0.003665 \\ 
  PT & CES & Quality-adjusted & Without energy & kl & Reject CST & $-$0.028768 & $-$0.012555 \\ 
  PT & CES & Quality-adjusted & With energy & ekl & Reject CST & $-$0.028718 & $-$0.012542 \\ 
  UK & exp &  & Without energy &  & Reject CST & 0.000000 & 0.000000 \\ 
  UK & CES & Unadjusted & Without energy & kl & Adhere to CST & $-$0.002651 & $-$0.000721 \\ 
  UK & CES & Unadjusted & Without energy & kl & Reject CST & $-$0.005661 & $-$0.000886 \\ 
  UK & CES & Unadjusted & With energy & ekl & Reject CST & $-$0.000883 & $-$0.000930 \\ 
  UK & CES & Quality-adjusted & Without energy & kl & Adhere to CST & $-$0.009973 & $-$0.000004 \\ 
  UK & CES & Quality-adjusted & Without energy & kl & Reject CST & $-$0.021609 & $-$0.000641 \\ 
  UK & CES & Quality-adjusted & With energy & ekl & Reject CST & $-$0.012227 & $-$0.000910 \\ 
   \hline
\end{tabular}
}
\end{table}

%
Figures~\ref{fig:DSRmse_CD_graph} and~\ref{fig:percDSRmse_CD_graph} summarize 
$\Delta \lambda$ and $\Delta mse$ 
results for the Cobb-Douglas model.
%
\begin{knitrout}
\definecolor{shadecolor}{rgb}{0.969, 0.969, 0.969}\color{fgcolor}\begin{figure}[H]

{\centering \includegraphics[width=\maxwidth]{figure/DSRmse_CD_graph-1} 

}

\caption[Change in Solow Residuals~(]{Change in Solow Residuals~($\Delta\lambda$) and mean squared error~($\Delta mse$) for the Cobb-Douglas model relative to the exponential-only model.}\label{fig:DSRmse_CD_graph}
\end{figure}


\end{knitrout}
%
\begin{knitrout}
\definecolor{shadecolor}{rgb}{0.969, 0.969, 0.969}\color{fgcolor}\begin{figure}[H]

{\centering \includegraphics[width=\maxwidth]{figure/percDSRmse_CD_graph-1} 

}

\caption[Percentage change in Solow Residuals~(]{Percentage change in Solow Residuals~($\Delta\lambda \, [\%]$) and mean squared error~($\Delta mse \, [\%]$) for the Cobb-Douglas model relative to exponential-only models.}\label{fig:percDSRmse_CD_graph}
\end{figure}


\end{knitrout}
%
Figures~\ref{fig:DSRmse_CES_graph} and~\ref{fig:percDSRmse_CES_graph} summarize 
$\Delta \lambda$ and $\Delta mse$ 
results for CES models.
%
\begin{knitrout}
\definecolor{shadecolor}{rgb}{0.969, 0.969, 0.969}\color{fgcolor}\begin{figure}[H]

{\centering \includegraphics[width=\maxwidth]{figure/DSRmse_CES_graph-1} 

}

\caption[Change in Solow Residuals~(]{Change in Solow Residuals~($\Delta\lambda$) and mean squared error~($\Delta mse$) for the CES modles relative to the exponential-only model.}\label{fig:DSRmse_CES_graph}
\end{figure}


\end{knitrout}
%
\begin{knitrout}
\definecolor{shadecolor}{rgb}{0.969, 0.969, 0.969}\color{fgcolor}\begin{figure}[H]

{\centering \includegraphics[width=\maxwidth]{figure/percDSRmse_CES_graph-1} 

}

\caption[Percentage change in Solow Residuals~(]{Percentage change in Solow Residuals~($\Delta\lambda \, [\%]$) and mean squared error~($\Delta mse \, [\%]$) for the CES models relative to the exponential-only model.}\label{fig:percDSRmse_CES_graph}
\end{figure}


\end{knitrout}


%++++++++++++++++++++++++++++++
\subsection{Are the means of the fitting residuals 
            significantly different from zero?} 
\label{sec:fitting_residuals_different_from_zero}
%++++++++++++++++++++++++++++++

The set of fitting residual values~($r_i$)
for each combination of 
country~(PT or UK), 
model~(Cobb-Douglas or CES),
flavor~(Unadjusted or Quality-adjusted), 
energy~(with or without), and
nest~[($kl$), ($kl$)$e$, ($le$)$k$, or ($ek$)$l$, for the CES models]
can be assessed for statistical significance
compared to zero by a t-test.
The null hypothesis for the t-test is that the mean of each group is
equal to zero, and 
the alternative hypothesis is that the mean is not equal to zero.
Smaller p-values indicate increasing confidence that the 
mean is different from zero. 
We expect that p-values will be large (close to unity), 
because the mean of the residuals is expected to be small (close to zero).
Indeed, that is the case.
Table~\ref{tab:r_p_values} summarizes the mean 
and associated p-values values for the fitting residuals~($r_i$).
%
% latex table generated in R 3.2.0 by xtable 1.7-4 package
% Wed Jul  8 15:43:57 2015
\begin{table}[ht]
\centering
\caption{Means and p-values for fitting residuals~($r_i$).} 
\label{tab:r_p_values}
{\tiny
\begin{tabular}{lllllrr}
  \hline
Country & model & flavor & energy & nest & Residuals.mean & Residuals.p.value \\ 
  \hline
PT & CD & Unadjusted & Without energy &  & $-$0.000000 & 1.000000 \\ 
  PT & CD & Unadjusted & With energy &  & $-$0.000000 & 1.000000 \\ 
  PT & CD & Quality-adjusted & Without energy &  & 0.000000 & 1.000000 \\ 
  PT & CD & Quality-adjusted & With energy &  & 0.000000 & 1.000000 \\ 
  PT & CES & Unadjusted & Without energy & kl & 0.000000 & 0.999999 \\ 
  PT & CES & Unadjusted & With energy & kle & $-$0.000000 & 0.999999 \\ 
  PT & CES & Unadjusted & With energy & lek & $-$0.000162 & 0.962887 \\ 
  PT & CES & Unadjusted & With energy & ekl & $-$0.000080 & 0.985890 \\ 
  PT & CES & Quality-adjusted & Without energy & kl & $-$0.000025 & 0.997238 \\ 
  PT & CES & Quality-adjusted & With energy & kle & $-$0.000083 & 0.983608 \\ 
  PT & CES & Quality-adjusted & With energy & lek & $-$0.000007 & 0.998657 \\ 
  PT & CES & Quality-adjusted & With energy & ekl & 0.000001 & 0.999715 \\ 
  PT & exp &  & Without energy &  & $-$0.000000 & 1.000000 \\ 
  UK & CD & Unadjusted & Without energy &  & 0.000000 & 1.000000 \\ 
  UK & CD & Unadjusted & With energy &  & $-$0.000000 & 1.000000 \\ 
  UK & CD & Quality-adjusted & Without energy &  & 0.000000 & 1.000000 \\ 
  UK & CD & Quality-adjusted & With energy &  & $-$0.000000 & 1.000000 \\ 
  UK & CES & Unadjusted & Without energy & kl & $-$0.000000 & 1.000000 \\ 
  UK & CES & Unadjusted & With energy & kle & $-$0.000037 & 0.986099 \\ 
  UK & CES & Unadjusted & With energy & lek & $-$0.000000 & 1.000000 \\ 
  UK & CES & Unadjusted & With energy & ekl & 0.000000 & 0.999997 \\ 
  UK & CES & Quality-adjusted & Without energy & kl & 0.000000 & 0.999999 \\ 
  UK & CES & Quality-adjusted & With energy & kle & $-$0.000000 & 0.999998 \\ 
  UK & CES & Quality-adjusted & With energy & lek & 0.000000 & 1.000000 \\ 
  UK & CES & Quality-adjusted & With energy & ekl & 0.000000 & 0.999998 \\ 
  UK & exp &  & Without energy &  & $-$0.000000 & 1.000000 \\ 
   \hline
\end{tabular}
}
\end{table}



%++++++++++++++++++++++++++++++
\subsection{Does quality-adjusting the factors of production or including energy
            decrease the fitting residuals in a statistically-significant manner?} 
\label{sec:quality_adj_energy_and_fitting_residual}
%++++++++++++++++++++++++++++++

The set of annual $\Delta r$ values for each combination of 
country~(PT or UK), 
model~(Cobb-Douglas or CES),
flavor~(Unadjusted or Quality-adjusted), 
energy~(with or without), and
nest~[($kl$), ($kl$)$e$, ($le$)$k$, or ($ek$)$l$, for the CES models]
can be assessed for statistical significance
compared to zero by a t-test.
The null hypothesis for the t-test is that the mean of each group is
equal to zero, and 
the alternative hypothesis is that the mean is less than zero.
Smaller p-values indicate increasing confidence that the 
mean is actually less than zero.
Table~\ref{tab:Dr_p_values} summarizes the mean 
and associated p-values values for $\Delta r$.
%
% latex table generated in R 3.2.0 by xtable 1.7-4 package
% Wed Jul  8 15:43:57 2015
\begin{table}[ht]
\centering
\caption{Means and p-values for $\Delta r$.} 
\label{tab:Dr_p_values}
{\tiny
\begin{tabular}{lllllrr}
  \hline
Country & model & flavor & energy & nest & Dr.mean & Dr.p.value \\ 
  \hline
PT & CD & Unadjusted & Without energy &  & $-$0.025733 & 0.000000 \\ 
  PT & CD & Unadjusted & With energy &  & $-$0.048181 & 0.000000 \\ 
  PT & CD & Quality-adjusted & Without energy &  & $-$0.009587 & 0.003533 \\ 
  PT & CD & Quality-adjusted & With energy &  & $-$0.051804 & 0.000000 \\ 
  PT & CES & Unadjusted & Without energy & kl & $-$0.038049 & 0.000000 \\ 
  PT & CES & Unadjusted & With energy & kle & $-$0.069223 & 0.000000 \\ 
  PT & CES & Unadjusted & With energy & lek & $-$0.071483 & 0.000000 \\ 
  PT & CES & Unadjusted & With energy & ekl & $-$0.065928 & 0.000000 \\ 
  PT & CES & Quality-adjusted & Without energy & kl & $-$0.042697 & 0.000000 \\ 
  PT & CES & Quality-adjusted & With energy & kle & $-$0.067761 & 0.000000 \\ 
  PT & CES & Quality-adjusted & With energy & lek & $-$0.067419 & 0.000000 \\ 
  PT & CES & Quality-adjusted & With energy & ekl & $-$0.067335 & 0.000000 \\ 
  PT & exp &  & Without energy &  & 0.000000 &  \\ 
  UK & CD & Unadjusted & Without energy &  & $-$0.011254 & 0.000003 \\ 
  UK & CD & Unadjusted & With energy &  & $-$0.014450 & 0.000000 \\ 
  UK & CD & Quality-adjusted & Without energy &  & $-$0.004538 & 0.055557 \\ 
  UK & CD & Quality-adjusted & With energy &  & $-$0.012454 & 0.000072 \\ 
  UK & CES & Unadjusted & Without energy & kl & $-$0.013655 & 0.000000 \\ 
  UK & CES & Unadjusted & With energy & kle & $-$0.017349 & 0.000000 \\ 
  UK & CES & Unadjusted & With energy & lek & $-$0.015158 & 0.000000 \\ 
  UK & CES & Unadjusted & With energy & ekl & $-$0.015557 & 0.000000 \\ 
  UK & CES & Quality-adjusted & Without energy & kl & $-$0.007222 & 0.003270 \\ 
  UK & CES & Quality-adjusted & With energy & kle & $-$0.016547 & 0.000000 \\ 
  UK & CES & Quality-adjusted & With energy & lek & $-$0.013366 & 0.000007 \\ 
  UK & CES & Quality-adjusted & With energy & ekl & $-$0.014973 & 0.000000 \\ 
  UK & exp &  & Without energy &  & 0.000000 &  \\ 
   \hline
\end{tabular}
}
\end{table}



%++++++++++++++++++++++++++++++
\subsection{What trends exist in factors shares?} 
\label{sec:factor_shares}
%++++++++++++++++++++++++++++++

Figures~\ref{fig:Factor_shares_graph_PT} and~\ref{fig:Factor_shares_graph_UK}
show factor shares ($\alpha$ values) for CES models for
Portugal and the UK, respectively.
See Table~\ref{tab:CES_abg_equations} for details.



\begin{knitrout}
\definecolor{shadecolor}{rgb}{0.969, 0.969, 0.969}\color{fgcolor}\begin{figure}[H]

{\centering \includegraphics[width=\maxwidth]{figure/Factor_shares_graph_PT-1} 

}

\caption[Factor shares (]{Factor shares ($\alpha$ values) for CES models for Portugal.}\label{fig:Factor_shares_graph_PT}
\end{figure}


\end{knitrout}

\begin{knitrout}
\definecolor{shadecolor}{rgb}{0.969, 0.969, 0.969}\color{fgcolor}\begin{figure}[H]

{\centering \includegraphics[width=\maxwidth]{figure/Factor_shares_graph_UK-1} 

}

\caption[Factor shares (]{Factor shares ($\alpha$ values) for CES models for the UK.}\label{fig:Factor_shares_graph_UK}
\end{figure}


\end{knitrout}





























%%%%%%%%%%%%%%%%%%%%%%%%%%%%%%%
\section{Conclusion}
\label{sec:Conclusion}
%%%%%%%%%%%%%%%%%%%%%%%%%%%%%%%


%%%%%%%%%%%%%%%%%%%%%%%%%%%%%%%
\section{Future Work}
\label{sec:FutureWork}
%%%%%%%%%%%%%%%%%%%%%%%%%%%%%%%




%%%%%%%%%%%%%%%%%%%%%%%%%%%%%%%
\section*{Acknowledgements}
\label{sec:Acknowledgements}
%%%%%%%%%%%%%%%%%%%%%%%%%%%%%%%




%%%%%%%%%%%%%%%%%%%%%%%%%%%%%%%
% \section*{Reproducible Research}
%%%%%%%%%%%%%%%%%%%%%%%%%%%%%%%

% In the spirit of Reproducible Research \citep{Gandrud:2013vx}, 
% all data, spreadsheets, \texttt{R} code \citep{R}, and other materials 
% associated with this paper can be found at\\*
% \protect\url{https://github.com/MatthewHeun/Econ-Growth-R-Analysis}.



%% References
%%
%% Following citation commands can be used in the body text:
%%
%%  \citet{key}  ==>>  Jones et al. (1990)
%%  \citep{key}  ==>>  (Jones et al., 1990)
%%
%% Multiple citations as normal:
%% \citep{key1,key2}         ==>> (Jones et al., 1990; Smith, 1989)
%%                            or  (Jones et al., 1990, 1991)
%%                            or  (Jones et al., 1990a,b)
%% \cite{key} is the equivalent of \citet{key} in author-year mode
%%
%% Full author lists may be forced with \citet* or \citep*, e.g.
%%   \citep*{key}            ==>> (Jones, Baker, and Williams, 1990)
%%
%% Optional notes as:
%%   \citep[chap. 2]{key}    ==>> (Jones et al., 1990, chap. 2)
%%   \citep[e.g.,][]{key}    ==>> (e.g., Jones et al., 1990)
%%   \citep[see][pg. 34]{key}==>> (see Jones et al., 1990, pg. 34)
%%  (Note: in standard LaTeX, only one note is allowed, after the ref.
%%   Here, one note is like the standard, two make pre- and post-notes.)
%%
%%   \citealt{key}          ==>> Jones et al. 1990
%%   \citealt*{key}         ==>> Jones, Baker, and Williams 1990
%%   \citealp{key}          ==>> Jones et al., 1990
%%   \citealp*{key}         ==>> Jones, Baker, and Williams, 1990
%%
%% Additional citation possibilities
%%   \citeauthor{key}       ==>> Jones et al.
%%   \citeauthor*{key}      ==>> Jones, Baker, and Williams
%%   \citeyear{key}         ==>> 1990
%%   \citeyearpar{key}      ==>> (1990)
%%   \citetext{priv. comm.} ==>> (priv. comm.)
%%   \citenum{key}          ==>> 11 [non-superscripted]
%% Note: full author lists depends on whether the bib style supports them;
%%       if not, the abbreviated list is printed even when full requested.
%%
%% For names like della Robbia at the start of a sentence, use
%%   \Citet{dRob98}         ==>> Della Robbia (1998)
%%   \Citep{dRob98}         ==>> (Della Robbia, 1998)
%%   \Citeauthor{dRob98}    ==>> Della Robbia


%% References with bibTeX database:

%%%%%%%%%%%%%%%%%%%%%%%%%%%%%%%
\section*{References}
%%%%%%%%%%%%%%%%%%%%%%%%%%%%%%%

% \bibliographystyle{model2-names}
%%\bibliography{<your-bib-database>}
% \bibliography{Paper2.bib}

%% Authors are advised to submit their bibtex database files. They are
%% requested to list a bibtex style file in the manuscript if they do
%% not want to use model2-names.bst.

%% References without bibTeX database:

% \begin{thebibliography}{00}

%% \bibitem must have one of the following forms:
%%   \bibitem[Jones et al.(1990)]{key}...
%%   \bibitem[Jones et al.(1990)Jones, Baker, and Williams]{key}...
%%   \bibitem[Jones et al., 1990]{key}...
%%   \bibitem[\protect\citeauthoryear{Jones, Baker, and Williams}{Jones
%%       et al.}{1990}]{key}...
%%   \bibitem[\protect\citeauthoryear{Jones et al.}{1990}]{key}...
%%   \bibitem[\protect\astroncite{Jones et al.}{1990}]{key}...
%%   \bibitem[\protect\citename{Jones et al., }1990]{key}...
%%   \harvarditem[Jones et al.]{Jones, Baker, and Williams}{1990}{key}...
%%

% \bibitem[ ()]{}

% \end{thebibliography}


%% The Appendices part is started with the command \appendix;
%% appendix sections are then done as normal sections
%% \appendix
%% \section{}
%% \label{}


%%%%%%%%%%%%%%%%%%%%%%%%%%%%%%%
\appendix
%%%%%%%%%%%%%%%%%%%%%%%%%%%%%%%


%%%%%%%%%%%%%%%%%%%%%%%%%%%%%%%
\section{Derivation of dynamic Solow residual for the CD equation}
\label{sec:dynamic_sr_CD}
%%%%%%%%%%%%%%%%%%%%%%%%%%%%%%%

Assuming that parameters $\theta$, $\alpha$, and $\beta$
in Equation~\ref{eq:CDe} are known from parameter estimation,
we can estimate the value of $\lambda$ as follows.
First, assume constant returns to scale such that 
$\alpha + \beta + \gamma = 1$, 
and calculate $\gamma = 1 - \alpha - \beta$.
Then, take the natural logarithm ($\ln$) of Equation~\ref{eq:CDe} to obtain
%
\begin{equation} \label{eq:lnCD}
  \ln y = \ln \theta + \ln A + \alpha \ln k + \beta \ln l + (1 - \alpha - \beta) \ln e \;.
\end{equation}
%
By taking the derivative of Equation~\ref{eq:lnCD} with respect to time ($t$) 
and noting that model parameters $\theta$, $\alpha$, and $\beta$ are 
constant with respect to time, we obtain
%
\begin{equation} \label{eq:ddtlnCD}
  \frac{1}{y}\frac{\mathrm{d}y}{\mathrm{d}t} 
      = \frac{1}{A}\frac{\mathrm{d}A}{\mathrm{d}t} 
      + \alpha \frac{1}{k}\frac{\mathrm{d}k}{\mathrm{d}t} 
      + \beta \frac{1}{l}\frac{\mathrm{d}l}{\mathrm{d}t} 
      + (1 - \alpha - \beta) \frac{1}{e}\frac{\mathrm{d}e}{\mathrm{d}t} \;.
\end{equation}
%
Solving for the total factor productivity term gives
%
\begin{equation} \label{eq:solow_resid}
  \frac{1}{A}\frac{\mathrm{d}A}{\mathrm{d}t} 
      = \frac{1}{y}\frac{\mathrm{d}y}{\mathrm{d}t} 
      - \left[
              \alpha \frac{1}{k}\frac{\mathrm{d}k}{\mathrm{d}t} 
              + \beta \frac{1}{l}\frac{\mathrm{d}l}{\mathrm{d}t} 
              + (1 - \alpha - \beta) \frac{1}{e}\frac{\mathrm{d}e}{\mathrm{d}t} 
        \right] \;.
\end{equation}
%
Recognizing that $A \equiv \mathrm{e}^{\lambda (t-t_0)}$ gives
%
\begin{equation} \label{eq:lambda_calc}
  \frac{1}{A}\frac{\mathrm{d}A}{\mathrm{d}t} 
    = \frac{1}{\mathrm{e}^{\lambda (t-t_0)}} \; \lambda \mathrm{e}^{\lambda (t-t_0)}
    = \lambda \; ,
\end{equation}
%
which can be substituted into Equation~\ref{eq:solow_resid} to find
%
\begin{equation} \label{eq:lambda}
  \lambda 
    = \frac{1}{y}\frac{\mathrm{d}y}{\mathrm{d}t} 
    - \left[
              \alpha \frac{1}{k}\frac{\mathrm{d}k}{\mathrm{d}t} 
              + \beta \frac{1}{l}\frac{\mathrm{d}l}{\mathrm{d}t} 
              + (1 - \alpha - \beta) \frac{1}{e}\frac{\mathrm{d}e}{\mathrm{d}t} 
      \right] \;.
\end{equation}
%
Equation~\ref{eq:lambda} applies for any instant in time.

If we approximate derivatives in Equation~\ref{eq:lambda} with forward differences
between times $i$ and $j$, we find
%
\begin{equation} \label{eq:lambdaij}
  \lambda_{i,j}
    = \frac{1}{y_i}\frac{y_j - y_i}{t_j - t_i} 
    - \left[
            \alpha \frac{1}{k_i}\frac{k_j - k_i}{t_j - t_i} 
            + \beta \frac{1}{l_i}\frac{l_j - l_i}{t_j - t_i} 
            + (1 - \alpha - \beta) \frac{1}{e_i}\frac{e_j - e_i}{t_j - t_i} 
      \right] \;,
\end{equation}
% 
where $\lambda_{i,j}$ approximates the true, instantaneous value of $\lambda$
from Equation~\ref{eq:lambda} between times $i$ and $j$.

Interestingly, 
%
\begin{equation}
  \lambda_{1,n}
    \neq \sum_{i=2}^{n} \lambda_{i,i-1}  \;.
\end{equation}
% 
Rather,
%
\begin{equation} \label{eq:lambda1n}
  \lambda_{1,n}
    = \frac{1}{y_1}\frac{y_n - y_1}{t_n - t_1} 
      - \left[ 
            \alpha \frac{1}{k_1}\frac{k_n - k_1}{t_n - t_1} 
            + \beta \frac{1}{l_1}\frac{l_n - l_1}{t_n - t_1} 
            + (1 - \alpha - \beta) \frac{1}{e_1}\frac{e_n - e_1}{t_n - t_1} 
      \right] \; ,
\end{equation}
% 
which will become increasingly inaccurate over large time spans,
because $y_1$, $k_1$, $l_1$, and $e_1$ will be less representative
of the average value of $y$, $k$, $l$, and $e$ in the time span, respectively.
This suggests that an averaging approach such as
%
\begin{equation}
  \lambda_{1,n}
    = \frac{\sum_{i=2}^{n} \lambda_{i,i-1}}{n-1}
\end{equation}
%
is a better approximation of Equation~\ref{eq:lambda1n},
which is, itself, an approximation of the true value of $\lambda$ 
given by Equation~\ref{eq:lambda}.

The fraction~($f$) of GDP growth explained by the Solow residual
for the CD model
can be given as
%
\begin{equation} \label{eq:frac_growth_explained_by_solow}
  f \equiv \frac{\lambda}{\frac{1}{y}\frac{\mathrm{d}y}{\mathrm{d}t}} 
            = 1 - \frac{\alpha \frac{1}{k}\frac{\mathrm{d}k}{\mathrm{d}t} 
                      + \beta \frac{1}{l}\frac{\mathrm{d}l}{\mathrm{d}t} 
                      + (1 - \alpha - \beta) \frac{1}{e}\frac{\mathrm{d}e}{\mathrm{d}t}}
                  {\frac{1}{y}\frac{\mathrm{d}y}{\mathrm{d}t}} \; .
\end{equation}
%
Using forward differences to estimate $f$ yields
%
\begin{equation} \label{eq:f_finite_diff_1}
  f_{i,j} = \frac{\lambda_{i,j}}{\frac{1}{y_i}\frac{y_j - y_i}{t_j - t_i}}
       = 1 - \frac{\alpha \frac{1}{k_i}\frac{k_j - k_i}{t_j - t_i} 
            + \beta \frac{1}{l_i}\frac{l_j - l_i}{t_j - t_i} 
            + (1 - \alpha - \beta) \frac{1}{e_i}\frac{e_j - e_i}{t_j - t_i} }
            {\frac{1}{y_i}\frac{y_j - y_i}{t_j - t_i}} \; .
\end{equation}
%
Cancelling $t_j - t_i$ terms and simplifying gives
%
\begin{equation} \label{eq:f_finite_diff_2}
  f_{i,j} = 1 - \frac{\alpha \left( \frac{k_j}{k_i} - 1 \right) 
                      + \beta \left( \frac{l_j}{l_i} - 1 \right)
                      + (1 - \alpha - \beta) \left( \frac{e_j}{e_i} - 1 \right)}
            {\left( \frac{y_j}{y_i} - 1 \right)} \; ,
\end{equation}
%
which is an estimate of the instantaneous value of $f$ 
given by Equation~\ref{eq:frac_growth_explained_by_solow}.


%%%%%%%%%%%%%%%%%%%%%%%%%%%%%%%
\section{Derivation of dynamic Solow residual for the CES equation}
\label{sec:dynamic_sr_CES}
%%%%%%%%%%%%%%%%%%%%%%%%%%%%%%%

In this section, we derive the dynamic Solow residual ($\lambda$) 
for the CES equation with three generic factors of production, namely
$x_1$, $x_2$, and $x_3$.

We can define
%
\begin{equation} \label{eq:a_def}
  a \equiv \delta b^{\rho/\rho_1} + (1-\delta) x_3^{-\rho}
\end{equation}
%
and
%
\begin{equation} \label{eq:b_def}
  b \equiv \delta_1 x_1^{-\rho_1} + (1 - \delta_1) x_2^{-\rho_1} \; ,
\end{equation}
%
such that Equation~\ref{eq:CESgeneric} can be restated as
%
\begin{equation} \label{eq:simple_y} 
  y = \gamma A a^{-1/\rho} \; .
\end{equation}
%
Taking the natural logarithm of Equation~\ref{eq:simple_y}
and realizing that $\gamma$ is not a function of time, we find
%
\begin{equation}
  \frac{1}{y} \frac{\mathrm{d} y}{\mathrm{d} t} = \frac{1}{A} \frac{\mathrm{d} A}{\mathrm{d} t} 
                      + \left( - \frac{1}{\rho} \right) \frac{1}{a} \frac{\mathrm{d} a}{\mathrm{d} t} \; .
\end{equation}
%
By Equation~\ref{eq:lambda_calc}, and after rearranging, 
the dynamic Solow residual for the CES equation can be stated as
%
\begin{equation} \label{eq:lambda_for_CES}
  \lambda = \frac{1}{y} \frac{\mathrm{d} y}{\mathrm{d} t} + \frac{1}{a} \frac{1}{\rho} \frac{\mathrm{d} a}{\mathrm{d} t} \; .
\end{equation}
%
To find $\frac{\mathrm{d} a}{\mathrm{d} t}$ and $\frac{\mathrm{d} b}{\mathrm{d} t}$, 
we take the time derivatives 
of Equations~\ref{eq:a_def} and~\ref{eq:b_def}
and rearrange slightly to obtain
%
\begin{equation} \label{eq:da_dt}
  \frac{1}{\rho} \frac{\mathrm{d} a}{\mathrm{d} t} = \delta b^{(\rho/\rho_1 - 1)} \frac{1}{\rho_1} \frac{\mathrm{d} b}{\mathrm{d} t}
                                      - (1-\delta) x_3^{-(\rho + 1)} \frac{\mathrm{d} x_3}{\mathrm{d} t}
\end{equation}
%
and
%
\begin{equation} \label{eq:db_dt}
  \frac{1}{\rho_1} \frac{\mathrm{d} b}{\mathrm{d} t} = - \delta_1 x_1^{-(\rho_1 + 1)} \frac{\mathrm{d} x_1}{\mathrm{d} t}
                                      - (1 - \delta_1) x_2^{-(\rho_1 + 1)} \frac{\mathrm{d} x_2}{\mathrm{d} t} \; .
\end{equation}
%
Equations~\ref{eq:lambda_for_CES}--\ref{eq:db_dt} can be used to 
calculate a dynamic time series for $\lambda$ given 
fitted parameters from the CES model
($\rho, \rho_1, \delta, \, \mathrm{and} \; \delta_1$)
and time series for economic output~($y$) and
factors of production 
($x_1, x_2, \, \mathrm{and} \; x_3$).
The time derivatives $\left( 
\frac{\mathrm{d} y}{\mathrm{d} t}, 
\frac{\mathrm{d} x_1}{\mathrm{d} t},
\frac{\mathrm{d} x_2}{\mathrm{d} t}, \, \mathrm{and} \;
\frac{\mathrm{d} x_3}{\mathrm{d} t} \right)$
can be approximated from historical data
by forward finite differences in a manner similar to Equation~\ref{eq:lambdaij}.


%%%%%%%%%%%%%%%%%%%%%%%%%%%%%%%
\section{Thoughts on the ``dynamic'' Solow residual}
\label{sec:thoughts_dynamic_sr}
%%%%%%%%%%%%%%%%%%%%%%%%%%%%%%%

\ref{sec:dynamic_sr_CD} and \ref{sec:dynamic_sr_CES} derived expressions
for a ``dynamic'' Solow residual 
for Cobb-Douglas and CES production functions, respectively.
However, these approaches conflate 
stochastic error terms and the time-dependent contribution of ``technology'' 
to economic growth
in one variable~($\lambda_{i,j}$).
In this appendix, we let the Solow residual 
(expressed as time-independent $\lambda$)
stand on its own and try to separate the stochastic and time-dependent
aspects of the fitting residuals~($r_i$).
Note: this section is speculative, and we need feedback from Marco and Randy 
before moving ahead with a presentation or publication based on this work.

We can re-state the models (in general) as
%
\begin{equation} \label{eq:general_y}
  y_i = \hat{y}_i \epsilon_i
\end{equation}
%
where $\epsilon_i$ is a multiplicative error term, 
$\hat{y}_i$ is the predicted economic output, and
$\hat{y}_i \equiv f(\theta, \lambda, \alpha, \beta, \gamma; \; t_i)$.
(For the Cobb-Douglas model, $f = \theta A k^\alpha l^\beta e^\gamma$.)

Taking the natural logarithm~($\ln$) of Equation~\ref{eq:general_y} yields
%
\begin{equation} \label{eq:ln_y}
  \ln y_i = \ln \hat{y}_i + \ln \epsilon_i \; ,
\end{equation}
%
which can be rearranged to show
%
\begin{equation} \label{eq:lny_prep_for_r}
  \ln y_i - \ln \hat{y}_i = \ln \epsilon_i \; .
\end{equation}
%
Using Equation~\ref{eq:log-residuals}, we can write
%
\begin{equation} \label{eq:r_epsilon}
  r_i = \ln \epsilon_i \; .
\end{equation}
%
Our concern is that the residuals~($r_i$) may have a time-dependent component 
in addition to a stochastic component.
To decompose, we can define
%
\begin{equation} \label{eq:epsilon_definition}
  \epsilon_i = \mathrm{e}^{m(t_i - t_0)} \mathrm{e}^{\varepsilon_i^*} \; .
\end{equation}
%
Substituting Equation~\ref{eq:epsilon_definition} into \ref{eq:r_epsilon} yields
%
\begin{equation} \label{eq:r_for_correlation}
  r_i = m (t_i - t_0) + \varepsilon_i^* \; .
\end{equation}
%
where
$m$ is a slope and captures the linear (in natural logarithmic space), 
time-dependent component of $r_i$. 
$\varepsilon_i^*$ is a variable that captures a stochastic component of $\epsilon_i$.
Equation~\ref{eq:r_for_correlation} can be fitted via linear regression in time~($t_i$) as
%
\begin{equation} \label{eq:r_regression_eqn}
  r_i = m (t_i - t_0) + b + \varepsilon_i^* \; .
\end{equation}
%
In Equation~\ref{eq:r_regression_eqn},
$b$ is the $y$-intercept of the regression and captures a time-independent offset of $r_i$.
Recall that $\epsilon_i$ is related to the fitting residuals~($r_i$)
by Equation~\ref{eq:r_epsilon}.
Thus, obtaining estimates of $m$ and $b$ and values for $\varepsilon_i^*$
is equivalent to estimating the time-dependent, time-independent, and stochastic
components, respectively, of the multiplicative errors~($\epsilon_i$).
This regression is equivalent to performing a linear fit to data in 
Figures~\ref{fig:Graphs CD-resid}--\ref{fig:Graphs CESekl-resid}.

For PT, Cobb-Douglas, unadjusted, without energy, we obtain the following:
%
\begin{knitrout}
\definecolor{shadecolor}{rgb}{0.969, 0.969, 0.969}\color{fgcolor}\begin{kframe}
\begin{alltt}
\hlstd{mod} \hlkwb{<-} \hlstd{models}\hlopt{$}\hlstd{PT}\hlopt{$}\hlstd{unadjusted}\hlopt{$}\hlstd{noE}\hlopt{$}\hlstd{CD}
\hlstd{r_regression} \hlkwb{<-} \hlkwd{lm}\hlstd{(}\hlkwc{formula} \hlstd{=} \hlkwd{resid}\hlstd{(mod)} \hlopt{~} \hlstd{mod}\hlopt{$}\hlstd{data}\hlopt{$}\hlstd{iYear)}
\hlkwd{print}\hlstd{(}\hlkwd{summary}\hlstd{(r_regression))}
\end{alltt}
\begin{verbatim}

Call:
lm(formula = resid(mod) ~ mod$data$iYear)

Residuals:
      Min        1Q    Median        3Q       Max 
-0.107498 -0.040098  0.009557  0.044471  0.109683 

Coefficients:
                 Estimate Std. Error t value Pr(>|t|)
(Intercept)    -1.366e-17  1.575e-02       0        1
mod$data$iYear  3.847e-19  5.322e-04       0        1

Residual standard error: 0.0576 on 50 degrees of freedom
Multiple R-squared:  5.069e-32,	Adjusted R-squared:  -0.02 
F-statistic: 2.535e-30 on 1 and 50 DF,  p-value: 1
\end{verbatim}
\end{kframe}
\end{knitrout}
%
Note that the estimate for the coefficient in front of \texttt{mod\$data\$iYear}
($m$ in Equation~\ref{eq:r_for_correlation})
is incredibly small relative to the scale of the residuals~($r_i$),
\ensuremath{3.8468693\times 10^{-19}}, 
indicating that there is little 
linear time-dependent information in the residuals~($r_i$).
This result is the same as saying that 
the slope of a line fitted through the 
PT, Unadjusted, Without energy data 
in Figure~\ref{fig:Graphs CD-resid}
is (effectively) zero.
There is (effectively) no linear time trend in the residuals.

Furthermore, the intercept ($b$ in Equation~\ref{eq:r_for_correlation})
is very small relative to the scale of the residuals~($r_i$),
\ensuremath{-1.3658522\times 10^{-17}}, 
indicating that there is little time-independent information in the residuals~($r_i$).
This result is the same as saying that the line fitted to the
PT, Unadjusted, Without energy 
data in Figure~\ref{fig:Graphs CD-resid}
is nearly coincident with the $x$-axis.

Almost all of the information in the residuals~($r_i$)
is stochastic and captured by the $\varepsilon_i^*$ term.

Perhaps Marco can help us interpret the rest of the
output here (\texttt{F-statistic}, \texttt{p-value}, etc.).
(An aside: what about fitting a quadratic curve,
instead of a line, 
through the $r_i$ values in Figure~\ref{fig:Graphs CD-resid}?)

We can compare $r_i$ and $\varepsilon_i^*$, graphically,
but the result is uninteresting. 
We would expect $r_i$ and $\varepsilon_i^*$
to be equal to each other by Equation~\ref{eq:r_regression_eqn}, 
because
(a) $m (t_i - t_0) \ll \max(r_i)$
for any value of $t_i$
and
(b) $b \ll \max(r_i)$.
Indeed, Figure~\ref{fig:Residuals_comparison}
shows this is the case.
%
\begin{knitrout}
\definecolor{shadecolor}{rgb}{0.969, 0.969, 0.969}\color{fgcolor}\begin{figure}[H]

{\centering \includegraphics[width=\maxwidth]{figure/Residuals_comparison-1} 

}

\caption[Comparison between ]{Comparison between $\varepsilon_i^*$ and $r_i$}\label{fig:Residuals_comparison}
\end{figure}


\end{knitrout}
%
I'm not sure where this leaves us.
Marco and I agreed that I would re-derive the ``dynamic'' Solow residual
equation while properly including the multiplicative error term~($\epsilon_i$).
I think I have done that, but the results aren't terribly interesting.
There is little (linear) time-dependent~($m$) or 
time-independent~($b$) 
information in the residuals~($r_i$),
as evidenced by Figures~\ref{fig:Graphs CD-resid}--\ref{fig:Graphs CESekl-resid}.

Perhaps there is another direction in which I should go with this analysis?


%++++++++++++++++++++++++++++++
\section{Correlograms} 
\label{sec:correlograms}
%++++++++++++++++++++++++++++++

This appendix contains correlograms for all models.

%++++++++++++++++++++++++++++++
\subsection{Correlograms for models respecting the cost-share theorem} 
\label{sec:correlograms_CST}
%++++++++++++++++++++++++++++++

In this section we present correlograms for all models that respect the cost-share theorem.

\begin{knitrout}
\definecolor{shadecolor}{rgb}{0.969, 0.969, 0.969}\color{fgcolor}
\includegraphics[width=\maxwidth]{figure/Correlograms_CST-1} 

\includegraphics[width=\maxwidth]{figure/Correlograms_CST-2} 

\includegraphics[width=\maxwidth]{figure/Correlograms_CST-3} 

\includegraphics[width=\maxwidth]{figure/Correlograms_CST-4} 

\includegraphics[width=\maxwidth]{figure/Correlograms_CST-5} 

\includegraphics[width=\maxwidth]{figure/Correlograms_CST-6} 

\includegraphics[width=\maxwidth]{figure/Correlograms_CST-7} 

\includegraphics[width=\maxwidth]{figure/Correlograms_CST-8} 
\begin{kframe}\begin{verbatim}
$PT.unadjusted.noE.CD

Autocorrelations of series 'resid(model)', by lag

     0      1      2      3      4      5      6      7      8      9 
 1.000  0.865  0.726  0.601  0.485  0.371  0.260  0.157  0.078  0.028 
    10     11     12     13     14     15     16     17 
-0.041 -0.110 -0.161 -0.174 -0.144 -0.150 -0.177 -0.190 

$PT.unadjusted.noE.CES.kl

Autocorrelations of series 'resid(model)', by lag

     0      1      2      3      4      5      6      7      8      9 
 1.000  0.865  0.727  0.605  0.492  0.379  0.267  0.164  0.086  0.036 
    10     11     12     13     14     15     16     17 
-0.034 -0.107 -0.162 -0.175 -0.144 -0.152 -0.179 -0.194 

$PT.adjusted.noE.CD

Autocorrelations of series 'resid(model)', by lag

     0      1      2      3      4      5      6      7      8      9 
 1.000  0.852  0.696  0.566  0.445  0.326  0.219  0.116  0.039 -0.003 
    10     11     12     13     14     15     16     17 
-0.069 -0.139 -0.176 -0.186 -0.148 -0.154 -0.186 -0.200 

$PT.adjusted.noE.CES.kl

Autocorrelations of series 'resid(model)', by lag

     0      1      2      3      4      5      6      7      8      9 
 1.000  0.842  0.676  0.543  0.422  0.299  0.186  0.083  0.008 -0.030 
    10     11     12     13     14     15     16     17 
-0.098 -0.174 -0.211 -0.213 -0.162 -0.163 -0.188 -0.198 

$UK.unadjusted.noE.CD

Autocorrelations of series 'resid(model)', by lag

     0      1      2      3      4      5      6      7      8      9 
 1.000  0.711  0.401  0.255  0.211  0.158  0.053 -0.011 -0.009  0.016 
    10     11     12     13     14     15     16     17 
 0.018  0.017 -0.029 -0.014  0.029  0.047  0.002 -0.015 

$UK.unadjusted.noE.CES.kl

Autocorrelations of series 'resid(model)', by lag

     0      1      2      3      4      5      6      7      8      9 
 1.000  0.634  0.259  0.125  0.119  0.063 -0.072 -0.121 -0.067 -0.032 
    10     11     12     13     14     15     16     17 
 0.000  0.024 -0.033 -0.006  0.029  0.049 -0.018 -0.030 

$UK.adjusted.noE.CD

Autocorrelations of series 'resid(model)', by lag

     0      1      2      3      4      5      6      7      8      9 
 1.000  0.784  0.548  0.427  0.373  0.304  0.203  0.123  0.086  0.068 
    10     11     12     13     14     15     16     17 
 0.043  0.029 -0.011 -0.024 -0.024 -0.040 -0.092 -0.129 

$UK.adjusted.noE.CES.kl

Autocorrelations of series 'resid(model)', by lag

     0      1      2      3      4      5      6      7      8      9 
 1.000  0.740  0.470  0.344  0.298  0.226  0.120  0.054  0.041  0.021 
    10     11     12     13     14     15     16     17 
 0.009  0.016 -0.018 -0.023 -0.031 -0.044 -0.093 -0.116 
\end{verbatim}
\end{kframe}
\end{knitrout}


%++++++++++++++++++++++++++++++
\subsection{Correlograms for models reject the cost-share theorem} 
\label{sec:correlograms_noCST}
%++++++++++++++++++++++++++++++

In this section we present correlograms for all models that do not adhere to the cost-share theorem.

\begin{knitrout}
\definecolor{shadecolor}{rgb}{0.969, 0.969, 0.969}\color{fgcolor}
\includegraphics[width=\maxwidth]{figure/Correlograms_noCST-1} 

\includegraphics[width=\maxwidth]{figure/Correlograms_noCST-2} 

\includegraphics[width=\maxwidth]{figure/Correlograms_noCST-3} 

\includegraphics[width=\maxwidth]{figure/Correlograms_noCST-4} 

\includegraphics[width=\maxwidth]{figure/Correlograms_noCST-5} 

\includegraphics[width=\maxwidth]{figure/Correlograms_noCST-6} 

\includegraphics[width=\maxwidth]{figure/Correlograms_noCST-7} 

\includegraphics[width=\maxwidth]{figure/Correlograms_noCST-8} 

\includegraphics[width=\maxwidth]{figure/Correlograms_noCST-9} 

\includegraphics[width=\maxwidth]{figure/Correlograms_noCST-10} 

\includegraphics[width=\maxwidth]{figure/Correlograms_noCST-11} 

\includegraphics[width=\maxwidth]{figure/Correlograms_noCST-12} 

\includegraphics[width=\maxwidth]{figure/Correlograms_noCST-13} 

\includegraphics[width=\maxwidth]{figure/Correlograms_noCST-14} 

\includegraphics[width=\maxwidth]{figure/Correlograms_noCST-15} 

\includegraphics[width=\maxwidth]{figure/Correlograms_noCST-16} 

\includegraphics[width=\maxwidth]{figure/Correlograms_noCST-17} 

\includegraphics[width=\maxwidth]{figure/Correlograms_noCST-18} 

\includegraphics[width=\maxwidth]{figure/Correlograms_noCST-19} 

\includegraphics[width=\maxwidth]{figure/Correlograms_noCST-20} 

\includegraphics[width=\maxwidth]{figure/Correlograms_noCST-21} 

\includegraphics[width=\maxwidth]{figure/Correlograms_noCST-22} 

\includegraphics[width=\maxwidth]{figure/Correlograms_noCST-23} 

\includegraphics[width=\maxwidth]{figure/Correlograms_noCST-24} 
\begin{kframe}\begin{verbatim}
$PT.unadjusted.noE.CD

Autocorrelations of series 'resid(model)', by lag

     0      1      2      3      4      5      6      7      8      9 
 1.000  0.813  0.597  0.388  0.185  0.045 -0.042 -0.125 -0.137 -0.133 
    10     11     12     13     14     15     16     17 
-0.210 -0.270 -0.304 -0.327 -0.270 -0.191 -0.155 -0.073 

$PT.unadjusted.noE.CES.kl

Autocorrelations of series 'resid(model)', by lag

     0      1      2      3      4      5      6      7      8      9 
 1.000  0.651  0.285  0.104 -0.064 -0.121 -0.178 -0.113  0.085  0.133 
    10     11     12     13     14     15     16     17 
-0.061 -0.278 -0.413 -0.472 -0.461 -0.324 -0.116  0.013 

$PT.unadjusted.withE.CD

Autocorrelations of series 'resid(model)', by lag

     0      1      2      3      4      5      6      7      8      9 
 1.000  0.780  0.551  0.386  0.215  0.058 -0.057 -0.161 -0.167 -0.119 
    10     11     12     13     14     15     16 
-0.184 -0.233 -0.276 -0.294 -0.213 -0.139 -0.151 

$PT.unadjusted.withE.CES.kle

Autocorrelations of series 'resid(model)', by lag

     0      1      2      3      4      5      6      7      8      9 
 1.000  0.514  0.042 -0.070 -0.098 -0.123 -0.275 -0.225  0.085  0.219 
    10     11     12     13     14     15     16 
 0.005 -0.129 -0.217 -0.298 -0.213 -0.110 -0.043 

$PT.unadjusted.withE.CES.lek

Autocorrelations of series 'resid(model)', by lag

     0      1      2      3      4      5      6      7      8      9 
 1.000  0.404 -0.109 -0.163 -0.199 -0.253 -0.384 -0.175  0.255  0.386 
    10     11     12     13     14     15     16 
 0.165  0.018 -0.107 -0.282 -0.286 -0.166 -0.028 

$PT.unadjusted.withE.CES.ekl

Autocorrelations of series 'resid(model)', by lag

     0      1      2      3      4      5      6      7      8      9 
 1.000  0.539  0.134  0.043 -0.038 -0.133 -0.285 -0.218  0.033  0.139 
    10     11     12     13     14     15     16 
-0.077 -0.203 -0.265 -0.336 -0.348 -0.259 -0.105 

$PT.adjusted.noE.CD

Autocorrelations of series 'resid(model)', by lag

     0      1      2      3      4      5      6      7      8      9 
 1.000  0.877  0.728  0.599  0.473  0.360  0.267  0.170  0.101  0.066 
    10     11     12     13     14     15     16     17 
-0.004 -0.064 -0.118 -0.162 -0.167 -0.175 -0.216 -0.232 

$PT.adjusted.noE.CES.kl

Autocorrelations of series 'resid(model)', by lag

     0      1      2      3      4      5      6      7      8      9 
 1.000  0.428 -0.104 -0.252 -0.335 -0.273 -0.252 -0.072  0.305  0.463 
    10     11     12     13     14     15     16     17 
 0.184 -0.140 -0.289 -0.346 -0.284 -0.079  0.175  0.265 

$PT.adjusted.withE.CD

Autocorrelations of series 'resid(model)', by lag

     0      1      2      3      4      5      6      7      8      9 
 1.000  0.757  0.530  0.401  0.237  0.040 -0.133 -0.343 -0.420 -0.411 
    10     11     12     13     14     15     16 
-0.435 -0.397 -0.355 -0.274 -0.145 -0.004  0.065 

$PT.adjusted.withE.CES.kle

Autocorrelations of series 'resid(model)', by lag

     0      1      2      3      4      5      6      7      8      9 
 1.000  0.476 -0.023 -0.152 -0.239 -0.221 -0.255 -0.143  0.188  0.332 
    10     11     12     13     14     15     16 
 0.077 -0.169 -0.286 -0.329 -0.245 -0.077  0.133 

$PT.adjusted.withE.CES.lek

Autocorrelations of series 'resid(model)', by lag

     0      1      2      3      4      5      6      7      8      9 
 1.000  0.410 -0.108 -0.246 -0.342 -0.275 -0.259 -0.080  0.305  0.465 
    10     11     12     13     14     15     16 
 0.184 -0.113 -0.262 -0.332 -0.288 -0.094  0.165 

$PT.adjusted.withE.CES.ekl

Autocorrelations of series 'resid(model)', by lag

     0      1      2      3      4      5      6      7      8      9 
 1.000  0.422 -0.106 -0.249 -0.332 -0.268 -0.252 -0.075  0.302  0.457 
    10     11     12     13     14     15     16 
 0.172 -0.130 -0.278 -0.338 -0.282 -0.084  0.173 

$UK.unadjusted.noE.CD

Autocorrelations of series 'resid(model)', by lag

     0      1      2      3      4      5      6      7      8      9 
 1.000  0.595  0.130 -0.075 -0.107 -0.130 -0.225 -0.260 -0.193 -0.121 
    10     11     12     13     14     15     16     17 
-0.048 -0.012 -0.045  0.001  0.084  0.177  0.199  0.222 

$UK.unadjusted.noE.CES.kl

Autocorrelations of series 'resid(model)', by lag

     0      1      2      3      4      5      6      7      8      9 
 1.000  0.509  0.021 -0.117 -0.085 -0.133 -0.294 -0.323 -0.198 -0.164 
    10     11     12     13     14     15     16     17 
-0.068 -0.001 -0.040  0.027  0.068  0.132  0.095  0.112 

$UK.unadjusted.withE.CD

Autocorrelations of series 'resid(model)', by lag

     0      1      2      3      4      5      6      7      8      9 
 1.000  0.527  0.105 -0.064  0.004 -0.051 -0.249 -0.339 -0.321 -0.272 
    10     11     12     13     14     15     16     17 
-0.172 -0.133 -0.109 -0.052  0.078  0.147  0.207  0.213 

$UK.unadjusted.withE.CES.kle

Autocorrelations of series 'resid(model)', by lag

     0      1      2      3      4      5      6      7      8      9 
 1.000  0.449  0.055 -0.100 -0.028 -0.044 -0.257 -0.376 -0.268 -0.179 
    10     11     12     13     14     15     16     17 
-0.099 -0.055 -0.128 -0.010  0.111  0.227  0.136  0.099 

$UK.unadjusted.withE.CES.lek

Autocorrelations of series 'resid(model)', by lag

     0      1      2      3      4      5      6      7      8      9 
 1.000  0.521  0.074 -0.053  0.014 -0.059 -0.291 -0.378 -0.309 -0.279 
    10     11     12     13     14     15     16     17 
-0.161 -0.091 -0.099 -0.027  0.058  0.136  0.156  0.157 

$UK.unadjusted.withE.CES.ekl

Autocorrelations of series 'resid(model)', by lag

     0      1      2      3      4      5      6      7      8      9 
 1.000  0.513  0.080 -0.037  0.017 -0.036 -0.304 -0.416 -0.346 -0.311 
    10     11     12     13     14     15     16     17 
-0.174 -0.085 -0.089  0.000  0.092  0.165  0.195  0.176 

$UK.adjusted.noE.CD

Autocorrelations of series 'resid(model)', by lag

     0      1      2      3      4      5      6      7      8      9 
 1.000  0.645  0.192 -0.020 -0.083 -0.109 -0.213 -0.302 -0.266 -0.140 
    10     11     12     13     14     15     16     17 
-0.063 -0.045 -0.072 -0.056  0.035  0.132  0.165  0.194 

$UK.adjusted.noE.CES.kl

Autocorrelations of series 'resid(model)', by lag

     0      1      2      3      4      5      6      7      8      9 
 1.000  0.615  0.186  0.039  0.032 -0.010 -0.146 -0.237 -0.207 -0.172 
    10     11     12     13     14     15     16     17 
-0.122 -0.085 -0.096 -0.048  0.003  0.067  0.067  0.085 

$UK.adjusted.withE.CD

Autocorrelations of series 'resid(model)', by lag

     0      1      2      3      4      5      6      7      8      9 
 1.000  0.604  0.196  0.023  0.025 -0.023 -0.162 -0.258 -0.222 -0.182 
    10     11     12     13     14     15     16     17 
-0.084 -0.063 -0.077 -0.056 -0.055 -0.031  0.011  0.023 

$UK.adjusted.withE.CES.kle

Autocorrelations of series 'resid(model)', by lag

     0      1      2      3      4      5      6      7      8      9 
 1.000  0.453  0.039 -0.098 -0.030 -0.034 -0.212 -0.332 -0.243 -0.142 
    10     11     12     13     14     15     16     17 
-0.053  0.008 -0.022  0.071  0.130  0.161  0.007 -0.086 

$UK.adjusted.withE.CES.lek

Autocorrelations of series 'resid(model)', by lag

     0      1      2      3      4      5      6      7      8      9 
 1.000  0.562  0.151  0.006  0.023 -0.021 -0.189 -0.288 -0.243 -0.236 
    10     11     12     13     14     15     16     17 
-0.138 -0.093 -0.094 -0.039 -0.025  0.007  0.052  0.068 

$UK.adjusted.withE.CES.ekl

Autocorrelations of series 'resid(model)', by lag

     0      1      2      3      4      5      6      7      8      9 
 1.000  0.500  0.067 -0.062 -0.024 -0.039 -0.255 -0.376 -0.316 -0.289 
    10     11     12     13     14     15     16     17 
-0.169 -0.080 -0.093 -0.006  0.061  0.136  0.182  0.188 
\end{verbatim}
\end{kframe}
\end{knitrout}


%++++++++++++++++++++++++++++++
\section{Statistical details on all models} 
\label{sec:staistical_details}
%++++++++++++++++++++++++++++++

In this section, we present statistical details of all models.

At the moment, this is a simple example of two models.
I hope to develop a table later, after I figure out some things.
Both models are CES models with energy fitted to Quality-adjusted data using the ($kl$)($e$) nesting.
In Table~\ref{tab:stat_details_table}, Model 1 is for Portugal, and Model 2 is for the UK.


\begin{table}[hb]
\caption{Example output from \texttt{texreg}.}
\begin{center}
\begin{tiny}
\begin{tabular}{l D{.}{.}{4.9}@{} D{.}{.}{4.9}@{} }
\toprule
           & \multicolumn{1}{c}{Model 1} & \multicolumn{1}{c}{Model 2} \\
\midrule
gamma      & 1.027582^{***} & 0.980315^{***} \\
           & (0.011759)     & (0.012946)     \\
lambda     & 0.004508^{***} & 0.007103^{***} \\
           & (0.001169)     & (0.001737)     \\
delta\_1   & 1.000000^{***} & 0.502875^{***} \\
           & (0.000000)     & (0.036155)     \\
delta      & 0.902835^{***} & 0.848488       \\
           & (0.161689)     & (0.661918)     \\
rho\_1     & 54.581844      & 0.634472^{***} \\
           & (119.786669)   & (0.138784)     \\
rho        & -1.000000      & 68.650078      \\
           & (6.852668)     & (115.484907)   \\
\midrule
R$^2$      & 0.997138       & 0.997823       \\
Num.\ obs. & 50             & 51             \\
\bottomrule
\multicolumn{3}{l}{\tiny{$^{***}p<0.001$, $^{**}p<0.01$, $^*p<0.05$}}
\end{tabular}
\end{tiny}
\label{tab:stat_details_table}
\end{center}
\end{table}





% %++++++++++++++++++++++++++++++
% \subsection{What effect does including energy and quality-adjustment have 
%             on the Solow residual in the Cobb-Douglas model?} 
% \label{sec:quality_adj_energy_and_solow_residual}
% %++++++++++++++++++++++++++++++
% 
% We'll first look at Portugal.
% 
% %------------------------------
% \subsubsection{Portugal} 
% %------------------------------
% 
% First, use non-quality-adjusted data for Portugal and the Cobb-Douglas model.
% 
% <<Portugal C-D undadjusted without energy>>=
% naturalCoef(models$PT$unadjusted$noE$CD)
% @
% 
% <<Portugal C-D undadjusted with energy>>=
% naturalCoef(models$PT$unadjusted$withE$CD)
% @
% 
% The Solow residual (\texttt{lambda}) is always negative. 
% However, the absolute value of \texttt{lambda} decreases from 
% naturalCoef(models$PT$unadjusted$noE$CD)$lambda to 
% naturalCoef(models$PT$unadjusted$withE$CD)$lambda
% when adding energy.
% So it appears that for Portugal, 
% using unadjusted inputs and including energy as a factor of production
% reduces the absolute value of the Solow residual somewhat.
% 
% What about for quality-adjusted data and Portugal?
% 
% <<Portugal C-D adjusted without energy>>=
% naturalCoef(models$PT$adjusted$noE$CD)
% @
% 
% <<Portugal C-D adjusted with energy>>=
% naturalCoef(models$PT$adjusted$withE$CD)
% @
% 
% Using quality-adjusted inputs, the Solow residual (\texttt{lambda}) moves from 
% naturalCoef(models$PT$adjusted$noE$CD)$lambda to
% naturalCoef(models$PT$adjusted$withE$CD)$lambda when including energy.
% The absolute value of the Solow residual again decreases when adding energy.
% But, with quality-adjusted inputs, the effect is much stronger.
% 
% Table~\ref{tab:lambda_CD_PT} shows
% the effect of both quality-corrected factors of production and energy.
% We find that including energy as a factor of production 
% reduces the absolute value of the Solow residual (\texttt{lambda})
% whether or not the factors of production are quality-adjusted. 
% Quality-adjustment gets part of the way toward eliminating the Solow residual.
% 
% \begin{table} \caption{Portugal Solow residual results with Cobb-Douglas.} 
% \label{tab:lambda_CD_PT} 
%   \begin{center}
%     \begin{tabular}{r c c} 
%       \toprule
%                       & Unadjusted     & Quality-adjusted   \\
%       \midrule
%       Without energy  & naturalCoef(models$PT$unadjusted$noE$CD)$lambda    & naturalCoef(models$PT$adjusted$noE$CD)$lambda \\
%       With energy     & naturalCoef(models$PT$unadjusted$withE$CD)$lambda  & naturalCoef(models$PT$adjusted$withE$CD)$lambda \\
%       \bottomrule
%     \end{tabular}
%   \end{center}
% \end{table}
% 
% Note that for Portugal the contribution of capital (\texttt{alpha\_1})
% always shifts to energy (\texttt{alpha\_3}) when energy is included 
% as a factor of production.
% 
% What about for the UK?
% 
% %------------------------------
% \subsubsection{United Kingdom} 
% %------------------------------
% 
% Table~\ref{tab:lambda_CD_UK} shows fitted Cobb-Douglas coefficients for the UK. 
% 
% \begin{table} \caption{UK Solow residual results with Cobb-Douglas.} 
% \label{tab:lambda_CD_UK} 
%   \begin{center}
%     \begin{tabular}{r c c} 
%       \toprule
%                       & Unadjusted     & Quality-adjusted   \\
%       \midrule
%       Without energy  & naturalCoef(models$UK$unadjusted$noE$CD)$lambda   
%                                                 & naturalCoef(models$UK$adjusted$noE$CD)$lambda \\
%       With energy     & naturalCoef(models$UK$unadjusted$withE$CD)$lambda 
%                                                 & naturalCoef(models$UK$adjusted$withE$CD)$lambda \\
%       \bottomrule
%     \end{tabular}
%   \end{center}
% \end{table}
% 
% For the UK, the quality-adjusted, with energy Solow residual is 
% naturalCoef(models$UK$adjusted$withE$CD)$lambda * 100\%,
% a very small number indeed.
% For the UK, it appears that \emph{both} quality-adjustment and addition of energy as a factor of production
% are required to effectively eliminate the Solow residual,
% reducing it to naturalCoef(models$UK$adjusted$withE$CD)$lambda.
% 
% Adding energy alone actually \emph{in}creases the Solow residual from 
% naturalCoef(models$UK$unadjusted$noE$CD)$lambda to
% naturalCoef(models$UK$unadjusted$withE$CD)$lambda.
% The details for each model with unadjusted factors of production are shown below.
% 
% <<UK CD Unadjusted Coefficients>>=
% naturalCoef(models$UK$unadjusted$noE$CD)
% naturalCoef(models$UK$unadjusted$withE$CD)
% @
% 
% When energy is included, much of the effect of capital (\texttt{alpha\_1}) is 
% attributed to energy (\texttt{alpha\_3}), 
% as \texttt{alpha\_1} moves from 
% naturalCoef(models$UK$unadjusted$noE$CD)$alpha_1 to
% naturalCoef(models$UK$unadjusted$withE$CD)$alpha_1.
% 
% When we look at quality-adjusted inputs, we find similar results.
% 
% <<UK CD Quality-adjusted Coefficients>>=
% naturalCoef(models$UK$adjusted$noE$CD)
% naturalCoef(models$UK$adjusted$withE$CD)
% @
% 
% \texttt{alpha\_1} moves from 
% naturalCoef(models$UK$adjusted$noE$CD)$alpha_1 to
% naturalCoef(models$UK$adjusted$withE$CD)$alpha_1.
% 
% Now, let's compare the fits to historical data.
% 
% 
% 
% %++++++++++++++++++++++++++++++
% \subsection{Do we obtain a negative Solow residual for Portugal with the CES model?} 
% \label{sec:Solow_negative_for_PT_in_CES?}
% %++++++++++++++++++++++++++++++
% 
% First, look at the CES model without energy.
% Nesting is unabmiguous, so this fit does not need to be repeated
% for each nesting option. 
% (This means that the top rows of Tables~\ref{tab:lambda_CES-kle_PT}--\ref{tab:lambda_CES-ekl_PT}
% are identical.)
% 
% <<PT CES-kl, warning=FALSE>>=
% naturalCoef(models$PT$unadjusted$noE$CES$kl)
% naturalCoef(models$PT$adjusted$noE$CES$kl)
% @
% 
% Both unadjusted and quality-adjusted variables give positive Solow residuals,
% naturalCoef(models$PT$unadjusted$noE$CES$kl)$lambda for unadjusted inputs and
% naturalCoef(models$PT$adjusted$noE$CES$kl)$lambda for quality-adjusted inputs.
% Furthermore, quality-adjustment reduces the magnitude of the Solow residual.
% 
% %------------------------------
% \subsubsection{($kl$)$e$ nesting} 
% %------------------------------
% 
% Now, look at unadjusted inputs using the ($kl$)$e$ nesting.
% 
% <<PT CES-kle unadjusted, warning=FALSE>>=
% naturalCoef(models$PT$unadjusted$withE$CES$kle)
% @
% 
% Now look at quality-adjusted inputs using the ($kl$)$e$ nesting.
% 
% <<PT CES-kle quality-adjusted, warning=FALSE>>=
% naturalCoef(models$PT$adjusted$withE$CES$kle)
% @
% 
% Table~\ref{tab:lambda_CES-kle_PT} summarizes the Portugal Solow residual results with CES and the ($kl$)$e$ nesting.
% 
% \begin{table} \caption{Portugal Solow residual results with CES and ($kl$)$e$ nesting.} 
% \label{tab:lambda_CES-kle_PT} 
%   \begin{center}
%     \begin{tabular}{r c c} 
%       \toprule
%                       & Unadjusted     & Quality-adjusted   \\
%       \midrule
%       Without energy  & naturalCoef(models$PT$unadjusted$noE$CES$kl)$lambda 
%                                                             & naturalCoef(models$PT$adjusted$noE$CES$kl)$lambda \\
%       With energy     & naturalCoef(models$PT$unadjusted$withE$CES$kle)$lambda 
%                                                             & naturalCoef(models$PT$adjusted$withE$CES$kle)$lambda \\
%       \bottomrule
%     \end{tabular}
%   \end{center}
% \end{table}
% 
% Table~\ref{tab:sse_CES-kle_PT} shows the $sse$ values for each of these models.
% 
% \begin{table} \caption{PT $sse$ with CES and ($kl$)$e$ nesting.} 
% \label{tab:sse_CES-kle_PT} 
%   \begin{center}
%     \begin{tabular}{r c c} 
%       \toprule
%                       & Unadjusted     & Quality-adjusted   \\
%       \midrule
%       Without energy  & EconModels:::sse(models$PT$unadjusted$noE$CES$kl)$sse 
%                                                             & EconModels:::sse(models$PT$adjusted$noE$CES$kl)$sse \\
%       With energy     & EconModels:::sse(models$PT$unadjusted$withE$CES$kle)$sse 
%                                                             & EconModels:::sse(models$PT$adjusted$withE$CES$kle)$sse \\
%       \bottomrule
%     \end{tabular}
%   \end{center}
% \end{table}
% 
% In contrast to the Cobb-Douglas model,
% the CES model with ($kl$)$e$ nesting
% produces positive Solow residuals in all cases. 
% We also see that both 
% quality-adjusting the inputs and
% including energy
% reduce the Solow residual (\texttt{lambda}).
% 
% %------------------------------
% \subsubsection{($le$)$k$ nesting} 
% %------------------------------
% 
% Next, look at unadjusted inputs using the ($le$)$k$ nesting.
% 
% <<PT CES-lek unadjusted, warning=FALSE>>=
% naturalCoef(models$PT$unadjusted$withE$CES$lek)
% @
% 
% Now look at quality-adjusted inputs using the ($le$)$k$ nesting.
% 
% <<PT CES-lek quality-adjusted, warning=FALSE>>=
% naturalCoef(models$PT$adjusted$withE$CES$lek)
% @
% 
% Table~\ref{tab:lambda_CES-lek_PT} summarizes the Portugal Solow residual results with CES and the ($le$)$k$ nesting.
% 
% \begin{table} \caption{Portugal Solow residual results with CES and ($le$)$k$ nesting.} 
% \label{tab:lambda_CES-lek_PT} 
%   \begin{center}
%     \begin{tabular}{r c c} 
%       \toprule
%                       & Unadjusted     & Quality-adjusted   \\
%       \midrule
%       Without energy  & naturalCoef(models$PT$unadjusted$noE$CES$kl)$lambda 
%                                         & naturalCoef(models$PT$adjusted$noE$CES$kl)$lambda \\
%       With energy     & naturalCoef(models$PT$unadjusted$withE$CES$lek)$lambda 
%                                         & naturalCoef(models$PT$adjusted$withE$CES$lek)$lambda \\
%       \bottomrule
%     \end{tabular}
%   \end{center}
% \end{table}
% 
% Table~\ref{tab:sse_CES-lek_PT} shows the $sse$ values for each of these models.
% 
% \begin{table} \caption{PT $sse$ with CES and ($le$)$k$ nesting.} 
% \label{tab:sse_CES-lek_PT} 
%   \begin{center}
%     \begin{tabular}{r c c} 
%       \toprule
%                       & Unadjusted     & Quality-adjusted   \\
%       \midrule
%       Without energy  & EconModels:::sse(models$PT$unadjusted$noE$CES$kl)$sse 
%                                         & EconModels:::sse(models$PT$adjusted$noE$CES$kl)$sse \\
%       With energy     & EconModels:::sse(models$PT$unadjusted$withE$CES$lek)$sse 
%                                         & EconModels:::sse(models$PT$adjusted$withE$CES$lek)$sse \\
%       \bottomrule
%     \end{tabular}
%   \end{center}
% \end{table}
% 
% Similar to the CES model with ($kl$)$e$ nesting,
% the CES model with ($le$)$k$ nesting
% produces positive Solow residuals in all cases. 
% After energy has been added, quality-adjustment 
% has very little effect on the Solow residual,
% creating a slight increase from 
% naturalCoef(models$PT$unadjusted$withE$CES$lek)$lambda to
% naturalCoef(models$PT$adjusted$withE$CES$lek)$lambda.
% 
% 
% %------------------------------
% \subsubsection{($ek$)$l$ nesting} 
% %------------------------------
% 
% Next, look at unadjusted inputs using the ($ek$)$l$ nesting.
% 
% <<PT CES-ekl unadjusted, warning=FALSE>>=
% naturalCoef(models$PT$unadjusted$withE$CES$ekl)
% @
% 
% Now look at quality-adjusted inputs using the ($ek$)$l$ nesting.
% 
% <<PT CES-ekl quality-adjusted, warning=FALSE>>=
% naturalCoef(models$PT$adjusted$withE$CES$ekl)
% @
% 
% Table~\ref{tab:lambda_CES-ekl_PT} summarizes the Portugal Solow residual results with CES and the ($ek$)$l$ nesting.
% 
% \begin{table} \caption{Portugal Solow residual results with CES and ($ek$)$l$ nesting.} 
% \label{tab:lambda_CES-ekl_PT} 
%   \begin{center}
%     \begin{tabular}{r c c} 
%       \toprule
%                       & Unadjusted     & Quality-adjusted   \\
%       \midrule
%       Without energy  & naturalCoef(models$PT$unadjusted$noE$CES$kl)$lambda 
%                                         & naturalCoef(models$PT$adjusted$noE$CES$kl)$lambda \\
%       With energy     & naturalCoef(models$PT$unadjusted$withE$CES$ekl)$lambda 
%                                         & naturalCoef(models$PT$adjusted$withE$CES$ekl)$lambda \\
%       \bottomrule
%     \end{tabular}
%   \end{center}
% \end{table}
% 
% Table~\ref{tab:sse_CES-ekl_PT} shows the $sse$ values for each of these models.
% 
% \begin{table} \caption{PT $sse$ with CES and ($ek$)$l$ nesting.} 
% \label{tab:sse_CES-ekl_PT} 
%   \begin{center}
%     \begin{tabular}{r c c} 
%       \toprule
%                       & Unadjusted     & Quality-adjusted   \\
%       \midrule
%       Without energy  & EconModels:::sse(models$PT$unadjusted$noE$CES$kl)$sse 
%                                         & EconModels:::sse(models$PT$adjusted$noE$CES$kl)$sse \\
%       With energy     & EconModels:::sse(models$PT$unadjusted$withE$CES$ekl)$sse 
%                                         & EconModels:::sse(models$PT$adjusted$withE$CES$ekl)$sse \\
%       \bottomrule
%     \end{tabular}
%   \end{center}
% \end{table}
% 
% For the ($ek$)$l$ nesting, 
% the Solow residual is positive, and 
% quality-adjusting the inputs 
% reduces the Solow residual (\texttt{lambda}).
% Adding energy decreases the Solow residual 
% (from naturalCoef(models$PT$unadjusted$withE$CES$ekl)$lambda to 
% naturalCoef(models$PT$adjusted$withE$CES$ekl)$lambda)
% when enegy is included in the model.
% But, adding energy slightly increases the Solow residual
% (from naturalCoef(models$PT$adjusted$noE$CES$kl)$lambda to naturalCoef(models$PT$adjusted$withE$CES$ekl)$lambda)
% when quality-adjusted inputs are used.
% 
% 
% %++++++++++++++++++++++++++++++
% \subsection{Do we obtain a positive Solow residual for the UK with the CES model?} 
% \label{sec:Solow_positive_for_UK_in_CES?}
% %++++++++++++++++++++++++++++++
% 
% First, look at the CES model without energy.
% Nesting is unabmiguous, so this fit does not need to be repeated
% for each nesting option. 
% (This means that the top rows of Tables~\ref{tab:lambda_CES-kle_UK}--\ref{tab:lambda_CES-ekl_UK}
% are identical.)
% 
% <<UK CES-kl, warning=FALSE>>=
% naturalCoef(models$UK$unadjusted$noE$CES$kl)
% naturalCoef(models$UK$adjusted$noE$CES$kl)
% @
% 
% Both unadjusted and quality-adjusted variables give positive Solow residuals for the UK,
% naturalCoef(models$UK$unadjusted$noE$CES$kl)$lambda for unadjusted inputs and
% naturalCoef(models$UK$adjusted$noE$CES$kl)$lambda for quality-adjusted inputs.
% Furthermore, quality-adjustment reduces the magnitude of the UK Solow residual.
% 
% %------------------------------
% \subsubsection{($kl$)$e$ nesting} 
% %------------------------------
% 
% Now, look at unadjusted inputs using the ($kl$)$e$ nesting.
% 
% <<UK CES-kle unadjusted>>=
% naturalCoef(models$UK$unadjusted$withE$CES$kle)
% @
% 
% Now look at quality-adjusted inputs using the ($kl$)$e$ nesting.
% 
% <<UK CES-kle quality-adjusted>>=
% naturalCoef(models$UK$adjusted$withE$CES$kle)
% @
% 
% Table~\ref{tab:lambda_CES-kle_UK} summarizes the UK Solow residual results with CES and the ($kl$)$e$ nesting.
% 
% \begin{table} \caption{UK Solow residual results with CES and ($kl$)$e$ nesting.} 
% \label{tab:lambda_CES-kle_UK} 
%   \begin{center}
%     \begin{tabular}{r c c} 
%       \toprule
%                       & Unadjusted     & Quality-adjusted   \\
%       \midrule
%       Without energy  & naturalCoef(models$UK$unadjusted$noE$CES$kl)$lambda & naturalCoef(models$UK$adjusted$noE$CES$kl)$lambda \\
%       With energy     & naturalCoef(models$UK$unadjusted$withE$CES$kle)$lambda 
%                                                                 & naturalCoef(models$UK$adjusted$withE$CES$kle)$lambda \\
%       \bottomrule
%     \end{tabular}
%   \end{center}
% \end{table}
% 
% Table~\ref{tab:sse_CES-kle_UK} shows the $sse$ values for each of these models.
% 
% \begin{table} \caption{UK $sse$ with CES and ($kl$)$e$ nesting.} 
% \label{tab:sse_CES-kle_UK} 
%   \begin{center}
%     \begin{tabular}{r c c} 
%       \toprule
%                       & Unadjusted     & Quality-adjusted   \\
%       \midrule
%       Without energy  & EconModels:::sse(models$UK$unadjusted$noE$CES$kl)$sse & EconModels:::sse(models$UK$adjusted$noE$CES$kl)$sse \\
%       With energy     & EconModels:::sse(models$UK$unadjusted$withE$CES$kle)$sse 
%                                                             & EconModels:::sse(models$UK$adjusted$withE$CES$kle)$sse \\
%       \bottomrule
%     \end{tabular}
%   \end{center}
% \end{table}
% 
% Like the Cobb-Douglas model, the CES model produces positive Solow residuals
% with the ($kl$)$e$ nesting.
% We also see that both 
% quality-adjusting the inputs and
% including energy
% reduce the Solow residual (\texttt{lambda}).
% 
% %------------------------------
% \subsubsection{($le$)$k$ nesting}
% %------------------------------
% 
% Next, look at unadjusted inputs using the ($le$)$k$ nesting.
% 
% <<UK CES-lek unadjusted>>=
% naturalCoef(models$UK$unadjusted$withE$CES$lek)
% @
% 
% Now look at quality-adjusted inputs using the ($le$)$k$ nesting.
% 
% <<UK CES-lek quality-adjusted>>=
% naturalCoef(models$UK$adjusted$withE$CES$lek)
% @
% 
% Table~\ref{tab:lambda_CES-lek_UK} summarizes the UK Solow residual results with CES and the ($le$)$k$ nesting.
% 
% \begin{table} \caption{UK Solow residual results with CES and ($le$)$k$ nesting.} 
% \label{tab:lambda_CES-lek_UK} 
%   \begin{center}
%     \begin{tabular}{r c c} 
%       \toprule
%                       & Unadjusted     & Quality-adjusted   \\
%       \midrule
%       Without energy  & naturalCoef(models$UK$unadjusted$noE$CES$kl)$lambda & naturalCoef(models$UK$adjusted$noE$CES$kl)$lambda \\
%       With energy     & naturalCoef(models$UK$unadjusted$withE$CES$lek)$lambda 
%                                                             & naturalCoef(models$UK$adjusted$withE$CES$lek)$lambda \\
%       \bottomrule
%     \end{tabular}
%   \end{center}
% \end{table}
% 
% Table~\ref{tab:sse_CES-lek_UK} shows the $sse$ values for each of these models.
% 
% \begin{table} \caption{UK $sse$ with CES and ($le$)$k$ nesting.} 
% \label{tab:sse_CES-lek_UK} 
%   \begin{center}
%     \begin{tabular}{r c c} 
%       \toprule
%                       & Unadjusted     & Quality-adjusted   \\
%       \midrule
%       Without energy  & EconModels:::sse(models$UK$unadjusted$noE$CES$kl)$sse & EconModels:::sse(models$UK$adjusted$noE$CES$kl)$sse \\
%       With energy     & EconModels:::sse(models$UK$unadjusted$withE$CES$lek)$sse 
%                                                             & EconModels:::sse(models$UK$adjusted$withE$CES$lek)$sse \\
%       \bottomrule
%     \end{tabular}
%   \end{center}
% \end{table}
% 
% 
% Similar to the CES model with ($kl$)$e$ nesting,
% the CES model with ($le$)$k$ nesting
% produces positive Solow residuals in all cases. 
% Both quality-adjustment and inclusion of energy 
% decrease the magnitude of the Solow residual.
% 
% %------------------------------
% \subsubsection{($ek$)$l$ nesting}
% %------------------------------
% 
% Next, look at unadjusted inputs using the ($el$)$l$ nesting.
% 
% <<UK CES-ekl unadjusted>>=
% naturalCoef(models$UK$unadjusted$withE$CES$ekl)
% @
% 
% Now look at quality-adjusted inputs using the ($ek$)$l$ nesting.
% 
% <<UK CES-ekl quality-adjusted>>=
% naturalCoef(models$UK$adjusted$withE$CES$ekl)
% @
% 
% Table~\ref{tab:lambda_CES-ekl_UK} summarizes the UK Solow residual results with CES and the ($ek$)$l$ nesting.
% 
% \begin{table} \caption{UK Solow residual results with CES and ($ek$)$l$ nesting.} 
% \label{tab:lambda_CES-ekl_UK} 
%   \begin{center}
%     \begin{tabular}{r c c} 
%       \toprule
%                       & Unadjusted     & Quality-adjusted   \\
%       \midrule
%       Without energy  & naturalCoef(models$UK$unadjusted$noE$CES$kl)$lambda & naturalCoef(models$UK$adjusted$noE$CES$kl)$lambda \\
%       With energy     & naturalCoef(models$UK$unadjusted$withE$CES$ekl)$lambda 
%                                                             & naturalCoef(models$UK$adjusted$withE$CES$ekl)$lambda \\
%       \bottomrule
%     \end{tabular}
%   \end{center}
% \end{table}
% 
% Table~\ref{tab:sse_CES-ekl_UK} shows the $sse$ values for each of these models.
% 
% \begin{table} \caption{UK $sse$ with CES and ($ek$)$l$ nesting.} 
% \label{tab:sse_CES-ekl_UK} 
%   \begin{center}
%     \begin{tabular}{r c c} 
%       \toprule
%                       & Unadjusted     & Quality-adjusted   \\
%       \midrule
%       Without energy  & EconModels:::sse(models$UK$unadjusted$noE$CES$kl)$sse & EconModels:::sse(models$UK$adjusted$noE$CES$kl)$sse \\
%       With energy     & EconModels:::sse(models$UK$unadjusted$withE$CES$ekl)$sse 
%                                                             & EconModels:::sse(models$UK$adjusted$withE$CES$ekl)$sse \\
%       \bottomrule
%     \end{tabular}
%   \end{center}
% \end{table}
% 
% 
% Similar to the CES model with ($ek$)$l$ nesting,
% the CES model with ($ek$)$l$ nesting
% produces positive Solow residuals in all cases. 
% Both quality-adjustment and inclusion of energy 
% decrease the magnitude of the Solow residual.




%' %++++++++++++++++++++++++++++++
%' \subsection{Dynamic Solow residuals} 
%' \label{sec:dynamic_solow_residuals}
%' %++++++++++++++++++++++++++++++
%' 
%' <<Dynamic Solow Residuals>>=
%' # Calculate finite difference approximations of time derivatives for output and all factors of production
%' dynamicSR <- function(model, id, lag=1, pad="tail"){
%'   id_list <- strsplit(id, split = "[.]")[[1]]
%'   energy <- id_list[[3]]
%'   modelStr <- id_list[[4]]
%'   
%'   f <- model$formula
%'   denom <- summands(f, n=length(summands(f))) # time variable
%'   nums <- all.vars(f) [1:(length(all.vars(f))-1)] # y, k, l, and e (if present)
%'   df <- getData(model)[c(denom, nums)] # columns are: iYear, y, k, l, and e (if present)
%'   
%'   # Calculate names for the columns of finite difference derivatives
%'   derivNames <- lapply(1:length(nums), function(i){paste0("d", nums[[i]], ".d", denom)}) %>% unlist
%'   derivCols <- lapply(1:length(nums), function(i){
%'     # Calculate finite difference derivative columns
%'     ediff(df[[i+1]], pad=pad, lag=lag) / ediff(df[[1]], pad=pad, lag=lag)
%'   }) %>% do.call(cbind, .) %>% as.data.frame
%'   names(derivCols) <- derivNames  
%'   df <- cbind(derivCols, df[ , c(nums, denom)]) # puts derivatives first, then raw numbers, then time
%'   df[["dydtOvery"]] <- 1 / df[[ nums[[1]] ]] * df[[ derivNames[[1]] ]]
%'   coeffs <- naturalCoef(model)
%'   df <- cbind(df, coeffs)
%'   
%'   if (modelStr == "CD"){
%'     lambda_fd <- 
%'       df[["dydtOvery"]] - 
%'       df[["alpha_1"]] / df[[nums[[2]]]] * df[[derivNames[[2]]]] - 
%'       df[["alpha_2"]] / df[[nums[[3]]]] * df[[derivNames[[3]]]]
%'     if (energy == "withE"){
%'       # Subtract the energy term also
%'       lambda_fd <- lambda_fd - 
%'         df[["alpha_3"]] / df[[nums[[4]]]] * df[[derivNames[[4]]]]
%'     }
%'     # Fraction of growth explained by Solow Residual
%'     solowFrac <- lambda_fd / df[["dydtOvery"]]
%'     # Add to data frame
%'     df$lambda_fd <- lambda_fd
%'     df$solowFrac <- solowFrac
%'     return(df)
%'   } else if (modelStr == "CES"){
%'     # Assign nice variable names so that the equations below are easier to read (and debug, and maintain)
%'     # This code is nest-aware.
%'     n <- model$nest
%'     for (i in 1:length(model$nest)){
%'       df[[paste0("x", i)]] <- df[[ nums[ model$nest[i]+1 ] ]] # GDP is 1st col, add 1 to get to the factors of production
%'       df[[paste0("dx", i, "dt")]] <- df[[ derivNames[ model$nest[i]+1 ] ]]
%'     }
%'     if (energy == "noE"){
%'       # Calculate terms for the finite-difference Solow residual (lambda_fd)
%'       df <- df %>% 
%'         transform(b = delta_1 * x1^(-rho_1) + (1 - delta_1) * x2^(-rho_1)) %>%
%'         transform(dbdtOverrho1 = -delta_1 * x1^(-(rho_1+1))*dx1dt - (1 - delta_1) * x2^(-(rho_1+1))*dx2dt) %>%
%'         transform(lambda_fd = dydtOvery + 1/b * dbdtOverrho1) %>%
%'         transform(solowFrac = lambda_fd / dydtOvery)
%'       return(df)
%'     } else if (energy == "withE"){
%'       df <- df %>% 
%'         transform(b = delta_1 * x1^(-rho_1) + (1 - delta_1) * x2^(-rho_1)) %>%
%'         transform(a = delta * b^(rho/rho_1) + (1 - delta) * x3^(-rho)) %>%
%'         transform(dbdtOverrho1 = -delta_1 * x1^(-(rho_1+1))*dx1dt - (1 - delta_1) * x2^(-(rho_1+1))*dx2dt) %>%
%'         transform(dadtOverrho = delta * b^(rho/rho_1 - 1) * dbdtOverrho1 - (1 - delta) * x3^(-(rho+1))*dx3dt) %>%
%'         transform(lambda_fd = dydtOvery + 1/a * dadtOverrho) %>%
%'         transform(solowFrac = lambda_fd / dydtOvery)
%'       return(df)
%'     } else {
%'       stop (paste("Unknown energy:", energy))
%'     }
%'   } else {
%'     stop(paste("Unknown modelStr", modelStr))
%'   }
%' }
%' 
%' lag.to.use <- 1 # years
%' temp_data <- leaf_apply(models, class=c("CDEmodel", "cesModel", "LINEXmodel"), 
%'                         f = function(model, id, lag=lag.to.use, pad="tail"){
%'   df <- dynamicSR(model, id=id, lag=lag, pad=pad)
%'   id_list <- strsplit(id, split = "[.]")[[1]]
%'   df$Country <- id_list[[1]]
%'   df$flavor <- if (id_list[[2]] == "unadjusted") "Unadjusted" else "Quality-adjusted"
%'   df$energy <- if (id_list[[3]] == "noE") "Without energy" else "With energy"
%'   df$model <- id_list[[4]]
%'   df$nest <- if (length(id_list) > 4) id_list[[5]] else NA
%'   df$Year <- getData(model)[["Year"]]
%'   return(df)  
%' }) %>% do.call(plyr::rbind.fill, .)
%' 
%' # Get the unadjusted, without-energy Solow residuals and store them in a data frame
%' sr_unadj_noE <- filter(temp_data, flavor=="Unadjusted" & energy=="Without energy") %>% 
%'                 select(Year, Country, lambda_fd, model)
%' colnames(sr_unadj_noE)[colnames(sr_unadj_noE)=="lambda_fd"] <- "sr_unadj_noE"
%' # Merge to get the unadjusted, without-energy values next to each country, year, and model
%' solowResiduals <- merge(temp_data, sr_unadj_noE)
%' 
%' # Create a column containing ∆lambda_df, the difference between a lambda_df value
%' # and the lambda_df value in the same year for the unadjusted, no-energy case
%' solowResiduals[["Dlambda_fd"]] <- abs(solowResiduals[["lambda_fd"]]) - abs(solowResiduals[["sr_unadj_noE"]])
%' 
%' # Relevel factors before plotting
%' solowResiduals$Country <- relevelFactor(as.factor(solowResiduals$Country), c("PT", "UK"))
%' solowResiduals$flavor <- relevelFactor(as.factor(solowResiduals$flavor), c("Unadjusted", "Quality-adjusted"))
%' solowResiduals$energy <- relevelFactor(as.factor(solowResiduals$energy), c("Without energy", "With energy"))
%' solowResiduals$nest <- relevelFactor(as.factor(solowResiduals$nest), c("kl", "kle", "lek", "ekl"))
%' 
%' # Make data viewing easier
%' solowResiduals <- arrange(solowResiduals, Country, model, flavor, energy, nest)
%' @
%' 
%' <<DynamicSR_Graph>>=
%' dynamicSRGraph <- function(data){
%' ggplot(data=data) +
%'   geom_line(data=filter(data, energy=="Without energy"), mapping=aes(x=Year, y=dydtOvery)) +
%'   geom_line(mapping=aes(x=Year, y=lambda_fd, linetype=energy)) +
%'   scale_linetype_manual(values=c(2,3), name="") +
%'   facet_grid(Country ~ flavor) +
%'   scale_x_continuous(breaks=c(1960, 1980, 2000)) +
%'   xlab("") + 
%'   ylab(expression(lambda[ij])) +
%'   xy_theme()  
%' }
%' @
%' 
%' ``Dynamic'' Solow residuals~($\lambda_{i,j}$) can be calculated 
%' for the Cobb-Douglas and CES models
%' as shown in Appendices~\ref{sec:dynamic_sr_CD} and~\ref{sec:dynamic_sr_CES},
%' respectively.
%' 
%' Figure~\ref{fig:DynamicSR_CD_Graph} shows the evolution 
%' of the Solow residual~($\lambda_{i,j}$) over time
%' as calculated by Equation~\ref{eq:lambdaij} 
%' with the timespan for finite differences ($t_j - t_i$) set to 
%' lag.to.use~if (lag.to.use == 1) "year" else "years".
%' %
%' <<DynamicSR_CD_Graph, fig.pos="H", fig.align='center', fig.width=6.5, fig.height=4, fig.cap="Dynamic Solow residuals for the Cobb-Douglas model. Dashed lines are $\\lambda$ values. Solid lines are $\\frac{1}{y}\\frac{\\mathrm{d} y}{\\mathrm{d} t}$.">>=
%' dynamicSRGraph(filter(solowResiduals, model=="CD"))
%' @
%' %
%' <<DynamicSR_CESkle_Graph, fig.pos="H", fig.align='center', fig.width=6.5, fig.height=4, fig.cap="Dynamic Solow residuals for the CES model with ($kl$)$e$ nesting. Dashed lines are $\\lambda$ values. Solid lines are $\\frac{1}{y}\\frac{\\mathrm{d} y}{\\mathrm{d} t}$.">>=
%' dynamicSRGraph(filter(solowResiduals, model=="CES" & (nest=="kle" | nest=="kl")))
%' @
%' %
%' <<DynamicSR_CESlek_Graph, fig.pos="H", fig.align='center', fig.width=6.5, fig.height=4, fig.cap="Dynamic Solow residuals for the CES model with ($le$)$k$ nesting. Dashed lines are $\\lambda$ values. Solid lines are $\\frac{1}{y}\\frac{\\mathrm{d} y}{\\mathrm{d} t}$.">>=
%' dynamicSRGraph(filter(solowResiduals, model=="CES" & (nest=="lek" | nest=="kl")))
%' @
%' %
%' <<DynamicSR_CESekl_Graph, fig.pos="H", fig.align='center', fig.width=6.5, fig.height=4, fig.cap="Dynamic Solow residuals for the CES model with ($ek$)$l$ nesting. Dashed lines are $\\lambda$ values. Solid lines are $\\frac{1}{y}\\frac{\\mathrm{d} y}{\\mathrm{d} t}$.">>=
%' dynamicSRGraph(filter(solowResiduals, model=="CES" & (nest=="ekl" | nest=="kl")))
%' @


%' %++++++++++++++++++++++++++++++
%' \subsection{Does quality-adjusting the factors of production 
%'             decrease fitting residuals and/or the dynamic Solow residual?} 
%' \label{sec:quality_adj_energy_and_dynamic_solow_residual}
%' %++++++++++++++++++++++++++++++
%' 
%' <<SR and R trajectory data>>=
%' traj.data <- merge(solowResiduals[c("Country", "Year", "model", "flavor", "energy", "nest", "Dlambda_fd")], 
%'                    fitted_and_resid_data[c("Country", "Year", "model", "flavor", "energy", "nest", "Dr")]) %>%
%'                    arrange(Country, model, flavor, energy, nest)
%' traj.data.stats <- traj.data %>% 
%'   group_by(model, Country, flavor, energy, nest) %>%
%'   summarise(Dlambda_fd.mean = mean(Dlambda_fd, na.rm=TRUE), Dr.mean = mean(Dr))
%' 
%' ############################################
%' # Dlambda_fd.p <- traj.data %>% 
%' #   group_by(model, Country, flavor, energy, nest) %>% 
%' #   select(Dlambda_fd) %>%
%' #   t.test(., alternative="less")[["p.value"]]
%' ############################################  
%' @
%' 
%' <<SR and R trajectory plot function>>=
%' residualsTrajGraph <- function(data){
%'   plot <- ggplot(data=data) + 
%'     geom_point(mapping=aes(x=Dr, y=Dlambda_fd, color=energy)) + 
%'     facet_grid(Country ~ flavor) +
%'     xlab(expression(Delta*r[i])) +
%'     ylab(expression(Delta*lambda[ij])) +
%'     xy_theme() + 
%'     theme(legend.title=element_blank())
%'   return(plot)
%' }
%' @
%' 
%' We can test our hypothesis that quality-adjusting the factors of production
%' and including energy will reduce the Solow residuals~($\lambda$)
%' and the fitting residuals~($r$) by plotting
%' $\Delta \lambda_i$ vs. $\Delta r_i$, where
%' %
%' \begin{equation} \label{eq:delta_lambda_i}
%'   \Delta \lambda_i \equiv \left| \lambda_i \right| - \left| \lambda_{i, \, Unadjusted, \, Without \, energy} \right|
%' \end{equation}
%' %
%' and
%' %
%' \begin{equation} \label{eq:delta_r}
%'   \Delta r_i \equiv \left| r_i \right| - \left| r_{i, \,Unadjusted, \, Without \, energy} \right|.
%' \end{equation}
%' %
%' If most of the points fall in the lower-left quadrant of a graph, 
%' we can conclude that both Solow residual~($\lambda$) and
%' fitting residual~($r$) have decreased due to either 
%' quality-adjustment of the factors of production or 
%' including energy.
%' 
%' Figure~\ref{fig:Solow_residual_trajectories_CD} indicates that
%' for the Cobb-Douglas model, 
%' neither 
%' quality-adjusting the factors of production
%' nor including energy 
%' consistently reduces
%' the Solow residual or the fitting residual.
%' 
%' <<Solow_residual_trajectories_CD, fig.pos="H", fig.align='center', fig.width=6.5, fig.height=4, fig.cap="Dynamic Solow residual and fitting residual trajectories relative to the Unadjusted, Without-energy condition for the Cobb-Douglas model.">>=
%' residualsTrajGraph(filter(traj.data, model=="CD"))
%' @
%' 
%' We can also calculate the average values for $\Delta \lambda_{i,j}$ and $\Delta r_{i}$ 
%' as shown in Table~\ref{tab:Trajectory_Means_CD}.
%' %
%' <<Trajectory_Means_CD, results="asis">>=
%' myXTable(data = filter(traj.data.stats, model=="CD") %>% select(-nest), 
%'          caption = "Average $\\Delta\\lambda$ and $\\Delta r$ for Cobb-Douglas models.",
%'          label=paste0("tab:", opts_current$get("label")),
%'          digits=6)
%' @
%' %
%' The following tables and figures show the same analysis for the CES model.
%' %
%' <<Solow_residual_trajectories_CESkle, fig.pos="H", fig.align='center', fig.width=6.5, fig.height=4, fig.cap="Dynamic Solow residual and fitting residual trajectories relative to the Unadjusted, Without-energy condition for the CES model with ($kl$)$e$ nesting.">>=
%' residualsTrajGraph(filter(traj.data, model=="CES" & (nest=="kl" | nest=="kle")))
%' @
%' %
%' <<Trajectory_Means_CESkle, results="asis">>=
%' myXTable(data = filter(traj.data.stats, model=="CES" & (nest=="kl" | nest=="kle")), 
%'          caption = "Average $\\Delta\\lambda$ and $\\Delta r$ for CES models with ($kl$)$e$ nesting.",
%'          label=paste0("tab:", opts_current$get("label")),
%'          digits=6)
%' @
%' 
%' <<Solow_residual_trajectories_CESlek, fig.pos="H", fig.align='center', fig.width=6.5, fig.height=4, fig.cap="Dynamic Solow residual and fitting residual trajectories relative to the Unadjusted, Without-energy condition for the CES model with ($le$)$k$ nesting.">>=
%' residualsTrajGraph(filter(traj.data, model=="CES" & (nest=="kl" | nest=="lek")))
%' @
%' 
%' <<Trajectory_Means_CESlek, results="asis">>=
%' myXTable(data = filter(traj.data.stats, model=="CES" & (nest=="kl" | nest=="lek")), 
%'          caption = "Average $\\Delta\\lambda$ and $\\Delta r$ for CES models with ($le$)$k$ nesting.",
%'          label=paste0("tab:", opts_current$get("label")),
%'          digits=6)
%' @
%' 
%' <<Solow_residual_trajectories_CESekl, fig.pos="H", fig.align='center', fig.width=6.5, fig.height=4, fig.cap="Dynamic Solow residual and fitting residual trajectories relative to the Unadjusted, Without-energy condition for the CES model with ($ek$)$l$ nesting.">>=
%' residualsTrajGraph(filter(traj.data, model=="CES" & (nest=="kl" | nest=="ekl")))
%' @
%' 
%' <<Trajectory_Means_CESekl, results="asis">>=
%' myXTable(data = filter(traj.data.stats, model=="CES" & (nest=="kl" | nest=="ekl")), 
%'          caption = "Average $\\Delta\\lambda$ and $\\Delta r$ for CES models with ($ek$)$l$ nesting.",
%'          label=paste0("tab:", opts_current$get("label")),
%'          digits=6)
%' @
%' 


\end{document}

%%
%% End of file `elsarticle-template-2-harv.tex'.
